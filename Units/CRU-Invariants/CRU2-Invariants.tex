\documentclass{article}
\usepackage[mast]{dennis}

%\usepackage{epigraph}
%\setlength{\epigraphwidth}{0.8\textwidth}
\title{Invariants and Symmetry}
\author{Dennis Chen}
\date{CRU}

\begin{document}
\maketitle

\section{Invariants}
When we want to prove something isn't possible, we can use things that don't change across all operations - otherwise known as \textit{invariants}. We recommend reading the \href{https://www.math.cmu.edu/~mlavrov/arml/12-13/invariants-12-09-12.pdf}{ARML 2012 presentation} on invariants.

\begin{exam}[Parity]
Bob starts with $15$ coins. He may either deposit $4$ coins, or withdraw $6$ coins. Can he ever get $0$ coins?
\end{exam}

\begin{sol}The answer is quite obviously no - $15$ is not an odd number. The operations preserve the parity of Bob's coins, and $15$ and $0$ have different parities.\end{sol}

This idea of \textit{something} that is preserved - it doesn't necessarily have to be parity - can be very powerful for proving non-existence.

\begin{exam}[Not the Other Way]
Claire starts with $16$ coins. He may either deposit $6$ coins or withdraw $12$ coins. Can he ever get $2$ coins?
\end{exam}

Once again, an obvious no (consider mod $6$). But this problem is perfect for making a point: \textit{invariants can only disprove existence, not prove existence}. If we considered mod $2$ the invariant would seem fine as $16\equiv 2\pmod {2}.$ But existence (as is obvious in this case) is not possible. Thus a problem that asks you to find whether something is possible or not usually has an answer of "no," unless finding the construction itself is interesting.

The thing doesn't even necessarily have to be preserved. As long as its behavior is predictable enough in some manner, the idea of invariants applies.

\begin{exam}[Knight Attaching Piece]
Say a knight is currently attacking a piece on a chessboard. Prove that, after it moves, it is no longer attacking the piece.
\end{exam}

\begin{sol}
Note that the knight always changes color as it moves, and that it can only attack squares of one color at any time. So after moving, the knight is no longer attacking that piece because it is no longer attacking the square of the color the piece is on.
\end{sol}

Consult a chess rulebook if you do not know how a knight moves or how the board is colored.

\section{Symmetry}
What stays the same?

This is sort of similar to invariants - look for things that stay identical.

\begin{exam}[HMMT 2019]
Reimu and Sanae play a game using $4$ fair coins. Initially both sides of each coin are white. Starting with Reimu, they take turns to color one of the white sides either red or green. After all sides are colored, the $4$ coins are tossed. If there are more red sides showing up, then Reimu wins, and if there are more green sides showing up, then Sanae wins. However, if there is an equal number of red sides and green sides, then neither of them wins. Given that both of them play optimally to maximize the probability of winning, what is the probability that Reimu wins?
\end{exam}

\begin{sol}Notice that the chance Reimu wins is necessarily the same as Sanae's chance of winning - any coin with both sides red must be able to be paired with a coin with both sides green. Thus we want to minimize the probability of a tie. It's quite intuitive from here - we want the largest possible amount of coins to have an undecided outcome, so we will just have four fair coins. The probability of a tie is $T=\frac{\binom{4}{2}}{16}=\frac{6}{16}.$ Thus, the probability that Reimu wins is $p=\frac{1-T}{2}=\frac{\frac{10}{16}}{2}=\frac{5}{16}.$\end{sol}

\pagebreak

\section{Problems}

%\epigraph{\textit{Harry had always been frightened of ending up as one of those child prodigies that never amounted to anything and spent the rest of their
%lives boasting about how far ahead they’d been at age ten. But then most
%adult geniuses never amounted to anything either. There were probably
%a thousand people as intelligent as Einstein for every actual Einstein in
%history. Because those other geniuses hadn’t gotten their hands on the one thing you absolutely needed to achieve greatness. They’d never found an important problem.} 
%}{\emph{Harry Potter and the Methods of Rationality} \\ \textbf{Eliezer Yudkowski}}

\minpt{TBD}

\prob{2}{USSR Problem Book}{Is it possible for a knight on a standard chessboard to go from the bottom left corner of the board to the upper right corner of the board, while visiting each square on the board exactly once?}

\prob{2}{USAMTS 2019}{Let $n>1$ be an integer. There are $n$ orangutoads, conveniently numbered $1,2,\dots{},n$, each sitting at an integer position on the number line. They take turns moving in the order $1,2,\dots{},n$, and then going back to $1$ to start the process over; they stop if any orangutoad is ever unable to move. To move, an orangutoad chooses another orangutoad who is at least $2$ units away from her towards them by a a distance of $1$ unit. (Multiple orangutoads can be at the same position.) Show that eventually some orangutoad will be unable to move.}

\req{2}{PUMaC 2020}{Given the graph $G$ and cycle $C$ in it, we can perform the following operation: add another vertex $v$ to the graph, connect it to all vertices in $C$ and erase all the edges from $C.$ Prove that we cannot perform the operation indefinitely on a given graph.}

\prob{3}{NIMO Winter 2014/3}{The numbers $1, 2,\ldots , 10$ are written on a board. Every minute, one can select three numbers $a, b, c$ on the board, erase them, and write $\sqrt{a^{2}+b^{2}+c^{2}}$ in their place. This process continues until no more numbers can be erased. What is the largest possible number that can remain on the board at this point?}


\prob{3}{ELMO 1999/2}{Mr. Fat moves around on the lattice points according to the following rules: From point $(x,y)$ he may move to any of the points $(y,x)$, $(3x,-2y)$, $(-2x,3y)$, $(x+1,y+4)$ and $(x-1,y-4)$. Show that if he starts at $(0,1)$ he can never get to $(0,0)$.}


\req{3}{ELMO SL 2019/C1}{Elmo and Elmo's clone are playing a game. Initially, $n\geq 3$ points are given on a circle. On a player's turn, that player must draw a triangle using three unused points as vertices, without creating any crossing edges. The first player who cannot move loses. If Elmo's clone goes first and players alternate turns, who wins? (Your answer may be in terms of $n$.)}


\prob{6}{ISL 2017/C1}{A rectangle $\mathcal{R}$ with odd integer side lengths is divided into small rectangles with integer side lengths. Prove that there is at least one among the small rectangles whose distances from the four sides of $\mathcal{R}$ are either all odd or all even.}

\prob{13}{USAMO 2015/4}{Steve is piling $m \geq 1$ indistinguishable stones on the squares of an $n \times n$ grid. Each square can have an arbitrarily high pile of stones. After he finished piling his stones in some manner, he can then perform stone moves, defined as follows. Consider any four grid squares, which are corners of a rectangle, i.e. in positions $(i, k),(i, l),(j, k),(j, l)$ for some $1 \leq i, j, k, l \leq n,$ such that $i<j$ and $k<l .$ A stone move consists of either removing one stone from each of $(i, k)$ and $(j, l)$ and moving them to $(i, l)$ and $(j, k)$ respectively, or removing one stone from each of $(i, l)$ and $(j, k)$ and moving them to $(i, k)$ and $(j, l)$ respectively.

Two ways of piling the stones are equivalent if they can be obtained from one another by a sequence of stone moves. How many different non-equivalent ways can Steve pile the stones on the grid?} 

\end{document}
