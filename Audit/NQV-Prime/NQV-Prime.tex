\documentclass{article}

\usepackage[mast]{dennis}

\title{Prime Factorization}
\author{Dennis Chen}
\date{NQV}

\begin{document}
\maketitle

These are problems where you want to look at the highest power of a prime that divides a number. This is the $\nu_p$ function.

\section{Divisiblity}

Here is the formal definition of divisibility.

\begin{defi}[Divisibility]
For integers $a,b,$ we say $a$ divides $b$ if and only if there exists some integer $c$ such that $ac=b.$

We denote this as $a\mid b.$ % please use mid, not the \mid  symbol!
\end{defi}
This implies the following three facts.

\begin{fact}[Divisibility Results]
Given integers $a,b,c,$
\begin{itemize}
    \Item If $a\mid b$ and $b\mid c$ then $a\mid c.$ (This may be referred to as the “chain rule” of divisibility.)
    
    \Item If $a\mid b$ then $a\mid bc$ for all integer $c.$
    
    \Item If $a\mid b$ and $a\mid c,$ then $a\mid b+c$ and $a\mid b-c.$
\end{itemize}
\end{fact}

\subsection{P-adic Valuation}

P-adic valuation, or the $\nu_p$ function, asks for the largest power of $p$ that divides an integer.

\begin{defi}[P-adic Valuation]
For a positive integer $n,$ $\nu_p(n)$ is the largest integer that satisfies $p^{\nu_p(n)}\mid n.$
\end{defi}

Remember that $\nu_p(n)$ is only defined for prime $p.$

This implies the following obvious but very useful fact.

\begin{fact}[P-adic Inequality]
If $a\mid b,$ then for all primes $p,$ $\nu_p(a)\leq \nu_p(b).$
\end{fact}

\begin{pro}
We proceed by contradiction. Say $\nu_p(a)>\nu_p(b).$ Then $p^{\nu_p(a)}\mid a\mid b,$ implying that $p^{\nu_p(a)}\mid b$ by the chain rule of divisibility. But $\nu_p(b)$ is the largest power of $p$ that divides $b,$ contradiction.
\end{pro}

\subsection{GCD and LCM}
\begin{defi}[Greatest Common Divisor]We define $\gcd(a_1,a_2\dots a_n)$ as the largest positive integer such that \[\gcd(a_1,a_2\dots a_n)\mid a_i\] for all $1\leq i\leq n.$\end{defi}

\begin{defi}[Least Common Multiple]We define $\lcm(a_1,a_2\dots a_n)$ as the smallest \textbf{positive} integer such that \[a_i\mid \lcm(a_1,a_2\dots a_n)\] for all $1\leq i\leq n.$\end{defi}
As an exercise, list the divisors of $0,$ the numbers that $0$ divides, and find $\gcd(0,8).$

Now we take a look at the prime powers of $\gcd$ and $\lcm.$
\begin{fact}[P-adic Maximum and Minimum]
Given integers $a_1,a_2,\ldots,a_n,$
\begin{itemize}
\Item $\nu_p(\gcd(a_1,a_2,\ldots,a_n))=\min(\nu_p(a_1),\nu_p(a_2),\ldots,\nu_p(a_n))).$
\Item $\nu_p(\lcm(a_1,a_2,\ldots,a_n))=\max(\nu_p(a_1),\nu_p(a_2),\ldots,\nu_p(a_n)).$
\end{itemize}
\end{fact}

The proof is an obvious consequence of the P-adic Inequality.

\section{Well-known Divisor Tricks}
You should know everything here.

\begin{theo}[Fundamental Theorem of Arithmetic]
Every number greater than $1$ is either a prime or can be uniquely, up to order, expressed as a product of primes.
\end{theo}

If you are curious about the proof, you may check out \url{https://gowers.wordpress.com/2011/11/18/proving-the-fundamental-theorem-of-arithmetic/}.

\begin{theo}[Number of Divisors]
Say the prime factorization of $n$ is $p_1^{q_1}\cdot p_2^{q_2}\cdot\cdots\cdot p_k^{q_k}.$ Then $n$ has $(p_1+1)(p_2+1)\cdots(p_k+1)$ positive divisors.
\end{theo}

\begin{pro}
This is a simple combinatorics problem.

Note that there are $q_i+1$ numbers between $0$ and $q_i$ to pick from, and you choose the exponent of each prime $p_i$ for the divisor. So in total there are $(q_1+1)(q_2+1)\cdots(q_k+1)$ choices.
\end{pro}

\begin{theo}[Sum of Divisors]
Say the prime factorization of $n$ is $p_1^{q_1}\cdot p_2^{q_2}\cdot\cdots\cdot p_k^{q_k}.$ Then the sum of the factors of $n$ is
\[\left(\frac{p_1^{q_1+1}-1}{p_1-1}\right)\left(\frac{p_2^{q_1+1}-1}{p_2-1}\right)\cdots\left(\frac{p_k^{q_k+1}-1}{p_k-1}\right)=\]
\[\left(1+p_1+p_1^2+\cdots+p_1^{q_1}\right)\left(1+p_2+p_2^2+\cdots+p_2^{q_2}\right)\cdots\left(1+p_k+p_k^2+\cdots+p_k^{q_k}\right).\]
\end{theo}
The latter part of the proof is going to seem magical. Take some time to digest it.

\begin{pro}
By geometric series, $\frac{p_i^{q_i+1}-1}{p_i-1}=1+p_i+p_i^2+\cdots+p_i^{q_i}.$

Notice that expanding the product gives you every possible combination of powers of $p_1,p_2,\ldots,p_k.$ This means that summing all of these combinations together is equivalent to summing up all of the divisors of $n.$
\end{pro}

For concreteness, we present a few examples.
\begin{exam}
Find the prime factorization of $216.$
\end{exam}
\begin{sol}
The prime factorization is $2^3\cdot 3^3,$ and this is unique by the Fundamental Theorem of Arithmetic.
\end{sol}
\begin{exam}
Find the number of divisors of $216.$
\end{exam}
\begin{sol}
The prime factorization is $2^3\cdot 3^3,$ as we have established in the previous example. Now note that we can choose the power of $2$ of the divisor in $4$ ways, and we can choose the power of $3$ of the divisor in $4$ ways as well. Thus there are $4\cdot 4=16$ total divisors.
\end{sol}
\begin{exam}
Find the sum of the divisors of $216.$
\end{exam}
\begin{sol}
The sum of the divisors is $(2^0+2^1+2^2+2^3)(3^0+3^1+3^2+3^3)=15\cdot 40=600.$

Expand the sum out to convince yourself that all divisors are characterized exactly once.
\end{sol}
Here is a much harder example, motivated by an obvious fact.
\begin{fact}[Odd Number of Divisors]
A number has an odd number of positive divisors if and only if it is a perfect square.
\end{fact}
The proof is just looking at the number of divisors formula. Alternatively, each divisor $d$ gets paired off with $\frac{n}{d},$ except for $\sqrt{n}.$

\begin{exam}[104 NT]
Twenty bored students take turns walking down a hall that contains a row of closed lockers, numbered 1 to 20. The first student opens all the lockers; the second student closes all the lockers numbered 2, 4, 6, 8, 10, 12, 14, 16, 18, 20; the third student operates on the lockers numbered 3, 6, 9, 12, 15, 18: if a locker was closed, he opens it, and if a locker was open, he closes it; and so on. For the $i$th student, he works on the lockers numbered by multiples of $i$: if a locker was closed, he opens it, and if a locker was open, he closes it. What is the number of the lockers that remain open after all the students finish their walks?
\end{exam}

\begin{sol}
Note that a locker is only open if it is interacted with an odd number of times, and the number of times a locker is interacted with is the number of divisors it has. Since perfect squares are the only integers with an odd number of divisors, the open lockers are just $1,4,9,16.$ Thus $4$ lockers are open.
\end{sol}

\begin{exam}[AIME I 2005/12]
For positive integers $n,$ let $\tau (n)$ denote the number of positive integer divisors of $n,$ including 1 and $n.$ For example, $\tau (1)=1$ and $\tau(6) =4.$ Define $S(n)$ by $S(n)=\tau(1)+ \tau(2) + \cdots + \tau(n).$ Let $a$ denote the number of positive integers $n \leq 2005$ with $S(n)$ odd, and let $b$ denote the number of positive integers $n \leq 2005$ with $S(n)$ even. Find $|a-b|.$
\end{exam}
\begin{sol}
Note $\tau(n)$ is odd if and only if $n$ is a perfect square, implying that $S(n)$ is odd if $n$ is greater than an odd number of squares and even if $n$ is greater than an even number of squares.
    
We can explicitly characterize this as
\[S(n)\begin{cases}
\text{ is odd if } 1^2\leq n<2^2 \text{ or } 3^2\leq n<4^2 \text{ or } 5^2\leq n< 6^2 \text{ or} \ldots \\
\text{ is even if } 2^2\leq n<3^2 \text{ or } 4^2\leq n<5^2 \text{ or } 6^2\leq n< 7^2 \text{ or} \ldots \\
\end{cases}.\]
Now we use the difference of squares formula to find $a$ and $b.$ Note that the largest square smaller than $2005$ is $44^2=1936,$ so
\[a-b=(-1^2+2^2-3^2+4^2-\cdots-43^2+44^2)-(-2^2+3^2-4^2+5^2-\cdots+42^2-43^2)-(2005-1936+1)\]
\[a-b=(1+2+3+4+\cdots+43+44)-(2+3+4+\cdots+43)-70\]
\[a-b=1+44-70=-25.\]
Thus the answer is $25.$
\end{sol}
\section{Assorted Examples}
These are some examples of prime factorization analysis problems. Since this is a technique rather than a theorem, we will show, not tell.
\begin{exam}[Dennis' Mock AIME 2020/4]
Find the number of ordered pairs of positive integers $(a, b)$ such that $\gcd(a, b) = 20$ and $\lcm(a, b) = 19!$
\end{exam}
\begin{sol}
\hfill
\begin{enumerate}
    \item Note $\gcd(a,b)\cdot \lcm(a,b)=ab,$ so $ab=20!$
    \item Let $\frac{a}{20}=x$ and $\frac{b}{20}=y.$ Then note $\gcd(x,y)=1$ and $\lcm(a,b)=\lcm(20x,20y)=20\lcm(x,y)=20xy.$ Thus $xy=\frac{19!}{20}.$
    \item Note picking $(x,y)$ uniquely determines $(a,b).$
    \item Now note we have to either assign \textit{all} of the powers of a prime to $x$ or to $y.$
    \item Check how many primes divide $\frac{19!}{20}.$
\end{enumerate}
\end{sol}
\begin{exam}[AIME 1987/7]
Let $[r,s]$ denote the least common multiple of positive integers $r$ and $s$. Find the number of ordered triples $(a,b,c)$ of positive integers for which $[a,b] = 1000$, $[b,c] = 2000$, and $[c,a] = 2000$.
\end{exam}
\begin{sol}
\hfill
\begin{enumerate}
    \item Let $a=2^a\cdot 5^x,$ $b=2^b\cdot 5^y,$ $c=2^c\cdot 5^z.$
    
    \item Note that $\max(a,b)=3,$ $\max(b,c)=4,$ $\max(c,a)=4.$ This notably implies $c=4.$ (Why can't we have $a=4$ or $b=4?$)
    
    \item Note that $\max(x,y)=\max(y,z)=\max(z,x)=3.$ How many of $x,y,z$ can be less than $3$ at a time? How many ways can we do this?
    
    \item Multiply the number of ways to distribute the powers of $2$ by the number of ways to distribute the powers of $5.$
\end{enumerate}
\end{sol}
Here is a prime factorization analysis problem that doesn't have GCD or LCM in the problem statement.
\begin{exam}[ARML 2008]
If $n$ has $60$ positive factors, compute the largest number of positive factors that $n^2$ could have.
\end{exam}
\begin{sol}
Let the prime factorization of $n$ be $p_1^{q_1}\cdot p_2^{q_2}\cdot\cdots\cdot p_k^{q_k}.$ Then note
\[(p_1+1)(p_2+1)\cdots(p_k+1)=60.\]
We want to maximize
\[(2p_1+1)(2p_2+1)\cdots(2p_k+1).\]
This occurs when $p_1=p_2=1,p_3=2,p_4=4.$ Thus our answer is $3\cdot 3\cdot 5\cdot 9=405.$
\end{sol}

\pagebreak

\section{Problems}

\minpt{45}

\psetquote{I won't ask you to buy me curry bread anymore. Goodbye.}{Yugami-kun Has No Friends}

\prob{1}{}{Answer the following:
\begin{itemize}
\Item What are the divisors of $0?$

\Item What integers does $0$ divide?

\Item Find $\gcd(0,8).$

\Item Why isn't something like $\lcm(0,8)$ defined?
\end{itemize}
}

\prob{1}{AMC 10A 2005/15}{Find the number of positive cubes that divide $3! \cdot 5! \cdot 7!$}

\prob{1}{AMC 8 2013/10}{What is the ratio of the least common multiple of 180 and 594 to the greatest common factor of 180 and 594?}

\req{2}{PUMaC 2016}{What is the smallest positive integer $n$ such that $2016n$ is a perfect cube?}

\prob{2}{SMT 2018}{One of the six digits in the expression $435 \cdot 605$ can be changed so that the product is a perfect square $N^2.$ Compute $N.$}

\prob{2}{AIME I 2010/1}{Maya lists all the positive divisors of $2010^2$. She then randomly selects two distinct divisors from this list. Let $p$ be the probability that exactly one of the selected divisors is a perfect square. The probability $p$ can be expressed in the form $\frac {m}{n}$, where $m$ and $n$ are relatively prime positive integers. Find $m + n$.}

\prob{2}{AMC 12B 2002/12}{For which integers $n$ is $\dfrac n{20-n}$ the square of an integer?}

\req{3}{Scrabbler AMC 10}{Let $n$ be the smallest positive integer with the property that $\lcm(n, 2020!) = 2021!,$ where $\lcm(a, b)$ denotes the least common multiple of $a$ and $b.$ How many positive factors does $n$ have?}

\prob{3}{Switzerland Prelimiary Round 2018/N1}{Let $n \ge 2$ be a positive integer, and let $d_1,\cdots , d_r$ be all the positive divisors of $n$ that are smaller that $n$. Determine all $n$ for which
\[\lcm(d_1,\cdots , d_r) \neq n.\]}

\req{3}{AMC 12A 2016/22}{How many ordered triples $(x, y, z)$ of positive integers satisfy $\text{lcm}(x, y) = 72$, $\text{lcm}(x, z)= 600$, and $\text{lcm}(y, z) = 900$?}

\prob{4}{CMC 10A 2021/22}{For a certain positive integer $n$, there are exactly $2021$ ordered pairs of positive divisors $(d_1, d_2)$ of $n$ for which $d_1$ and $d_2$ are relatively prime. What is the sum of all possible values of the number of divisors of $n?$}

\prob{4}{AHSME 1987/23}{If $p$ is a prime and both roots of $x^2+px-444p=0$ are integers, then what is $p?$}

\prob{4}{HMMT 2018}{Distinct prime numbers $p,q,r$ satisfy the equation $$2pqr + 50pq = 7pqr + 55pr = 8pqr + 12qr  =A$$ for some positive integer $A$. What is $A$?}

\prob{6}{AMC 12B 2007/24}{Find all pairs of positive integers $(a,b)$ such that $\text{gcd}(a,b)=1$ and $\frac{a}{b} + \frac{14b}{9a}$ is an integer.}

\prob{6}{AIME I 2020/10}{Let $m$ and $n$ be positive integers satisfying the conditions
\begin{itemize}
\Item $\gcd(m+n,210)=1,$

\Item $m^m$ is a multiple of $n^n,$ and

\Item $m$ is not a multiple of $n.$
\end{itemize}
Find the least possible value of $m+n.$}

\prob{6}{ARML 2010}{Compute the smallest positive integer $n$ such that $n^n$ has at least $1,000,000$ positive divisors.}

%\sol If the roots of a polynomial are integers, the discriminant, $p^2 - 4 \cdot 1 \cdot (-444p) = p^2 + 1776p = p(p+1776)$, must be a perfect square. If $p$ does not divide $p + 1776$, there would be an extra factor of $p$, and so this expression could not possibly be a perfect square. Thus $p$ divides $p + 1776$, which means $p$ divides $1776 = 2^{4} \cdot 3 \cdot 37$, so because $p$ is a prime, it must be $2$, $3$, or $37$. It is easy to verify that neither $p = 2$ nor $p = 3$ make $p(p+1776)$ a perfect square, but $p = 37$ does, so the answer is $\boxed{37}$.

\prob{6}{AMC 10B 2018/23}{How many ordered pairs $(a, b)$ of positive integers satisfy the equation\[a\cdot b + 63 = 20\cdot \text{lcm}(a, b) + 12\cdot\text{gcd}(a,b),\]where $\text{gcd}(a,b)$ denotes the greatest common divisor of $a$ and $b$, and $\text{lcm}(a,b)$ denotes their least common multiple?}

\req{9}{AMC 12A 2021/25}{Let $d(n)$ denote the number of positive integers that divide $n$, including $1$ and $n$. For example, $d(1)=1,d(2)=2,$ and $d(12)=6$. (This function is known as the divisor function.) Let \[f(n)=\frac{d(n)}{\sqrt[3]{n}}.\]There is a unique positive integer $N$ such that $f(N)>f(n)$ for all positive integers $n\ne N$. What is the sum of the digits of $N?$}

\prob{9}{AMC 10A 2018/22}{Let $a, b, c,$ and $d$ be positive integers such that $\gcd(a, b)=24$, $\gcd(b, c)=36$, $\gcd(c, d)=54$, and $70<\gcd(d, a)<100$. Which of the following must be a divisor of $a$?

\begin{align*}\textbf{(A)} \text{ 5} \qquad \textbf{(B)} \text{ 7} \qquad \textbf{(C)} \text{ 11} \qquad \textbf{(D)} \text{ 13} \qquad \textbf{(E)} \text{ 17}\end{align*}

}

\prob{9}{AMC 12B 2010/25}{For every integer $n\ge2$, let $\text{pow}(n)$ be the largest power of the largest prime that divides $n$. For example $\text{pow}(144)=\text{pow}(2^4\cdot3^2)=3^2$. What is the largest integer $m$ such that $2010^m$ divides $\prod\limits_{n=2}^{5300}\text{pow}(n)?$}

\prob{13}{ISL 2007/N2}{Let $b,n > 1$ be integers. Suppose that for each $k > 1$ there exists an integer $a_k$ such that $b - a^n_k$ is divisible by $k$. Prove that $b = A^n$ for some integer $A$.}

\prob{13}{PUMaC 2016}{Let $k=2^6\cdot 3^5\cdot 5^2\cdot 7^3\cdot 53.$ let $S$ be the sum of $\frac{\gcd(m,n)}{\lcm(m,n)}$ over all ordered pairs of positive integers $(m,n)$ where $mn=k.$ If $S$ can be written in simplest form as $\frac{r}{s},$ compute $r+s.$}
\end{document}