\documentclass{article}
\usepackage[utf8]{inputenc}
\usepackage[]{dennis}
\title{Orders and Primitive Roots}
\author{Aprameya Tripathy, minor edits by Dennis Chen}
\date{NRU}

\begin{document}

\maketitle

Thanks to Raymond Feng for suggesting several of the problems in this handout.

\section{Orders}
We begin by reviewing some introductory theorems.

\begin{theo}[Fermat]
$a^{p-1} \equiv 1 \pmod{p}$ whenever $p$ is prime and $p \nmid a.$
\end{theo}

\begin{theo}[Euler]
$a^{\phi(n)} \equiv 1 \pmod{n}$ whenever $\gcd(a, n) = 1.$
\end{theo}

(If you have forgotten the proofs for these theorems, try to reprove them as an exercise or refer to \db{NQU-Mod}.) Notice that both Fermat and Euler are \emph{weak.}

\begin{exam}[Stronger Euler on 1000]
Show that there exists some $x < 400$ such that $$a^x \equiv 1 \pmod{1000}$$ for all $a$ relatively prime to $1000.$
\end{exam}

\begin{sol}
We claim that \ansbold{$x=100$} is a satisfactory value of $x.$

Notice that by Euler, $$a^{100} \equiv a^{100 \pmod{\phi(8)}} \equiv a^{100 \pmod{4}} \equiv a^0 \equiv 1 \pmod{8}$$ and that $$a^{100} \equiv a^{\phi (125)} \equiv 1 \pmod{125}$$ whenever $\gcd(a, 1000) = 1.$ Thus, by CRT, $$a^{100} \equiv 1 \pmod{1000}$$ for all $a$ relatively prime to $1000.$
\end{sol}

Often, there is a number $x < \phi(n)$ such that $a^x \equiv 1 \pmod{n}$ for some $a.$ In order to properly discuss this $x,$ we define \textbf{orders.}

\begin{defi}[Orders]
Let $a$ and $n$ be two relatively prime integers with $n>1.$ Then $\operatorname{ord}_n(a)$ (the order of $a$ mod $n$) is the smallest positive integer $x$ such that $a^x \equiv 1 \pmod{n}.$
\end{defi}

We immediately notice the following fact.

\begin{fact}
For relatively prime integers $a$ and $m,$ $a^m \equiv 1 \pmod{n}$ if and only if $\operatorname{ord}_n(a) \mid m.$
\end{fact}

\begin{pro}
Clearly, $a^m \equiv 1 \pmod{n}$ if $\operatorname{ord}_n(a) \mid m$ by definition, so the if condition holds.

Now, if $a^m \equiv 1 \pmod{n},$ we see that $m \ge \operatorname{ord}_n(a)$ by definition. Thus, from the division algorithm, there exist two integers $q$ and $r$ such that \[m = q \cdot \operatorname{ord}_n(a)+r,\] where $0 \le r \le \operatorname{ord}_n(a)-1$ and $q>0.$ Since $a^{\operatorname{ord}_n(a)} \equiv 1 \pmod{n},$ we see $a^{-q \cdot \operatorname{ord}_n(a)} \equiv (1)^{-q} \equiv 1.$

Thus, $$a^r \equiv a^{m - q \cdot \operatorname{ord}_n(a)} \equiv 1 \pmod{n},$$ so by the minimality of $\operatorname{ord}_n(a),$ $r$ can't be positive (as $r \le \operatorname{ord}_n(a)-1$). Thus, we must have $r=0,$ meaning $\operatorname{ord}_n(a) \mid m,$ completing the proof.
\end{pro}

This fact is one of the most useful facts relating to orders - it allows us to take the order of some random value and relate it to the overall modulus. In other words, it allows us to get \textbf{global} information from \textbf{local} information - something that is very powerful in many places. We will explore two of those examples.

\begin{exam}[Fermat's Christmas Theorem]
Show that if a prime $p>2$ can be written as the sum of two squares, we must have $p \equiv 1 \pmod{4}.$
\end{exam}

\begin{sol}
Suppose that $p = x^2+y^2$ for some positive integers $x$ and $y.$ Clearly, $p \nmid x$ and $p \nmid y,$ as if $p$ divided either $x$ or $y,$ we would have $x^2+y^2 > p.$

Since $x^2+y^2 = p,$ we see $x^2 + y^2 \equiv 0 \pmod{p}.$ Thus, $(x y^{-1})^2 \equiv -1 \pmod{p},$ and (by squaring), $(xy^{-1})^4 \equiv 1 \pmod{p}.$ Thus, $\operatorname{ord}_p(xy^{-1}) \mid 4.$

Notice that if $\operatorname{ord}_p(xy^{-1}) \mid 2,$ then we would have $(xy^{-1})^2 \equiv 1 \pmod{p},$ but as we showed earlier, $(xy^{-1})^2 \equiv -1 \pmod{p}.$ Since $p > 2,$ this is clearly absurd, so we must have $\operatorname{ord}_p(xy^{-1}) \nmid 2.$ Since the only factor of $4$ that doesn't divide $2$ is $4,$ we must have $\operatorname{ord}_p(xy^{-1}) = 4.$

Now, from Fermat, $(xy^{-1})^{p-1} \equiv 1 \pmod{p}.$ Thus, $\operatorname{ord}_p(xy^{-1}) = 4 \mid p-1,$ so $p \equiv 1 \pmod{4}.$
\end{sol}

Observe how we did not \emph{ever} try to find $x,$ or $y,$ or $xy^{-1}.$ We only tried to find $\operatorname{ord}_p(xy^{-1}).$ The idea of finding orders instead of variables is quite useful.

Oftentimes, this idea works, but sometimes, we need to use another idea -- exploiting minimality.

\begin{exam}[China TST 2006 Quiz]
Find all positive integers $a$ and $n$ such that $$\frac{(a+1)^n - a^n}{n}$$ is an integer.
\end{exam}

\begin{sol}
Assume for the sake of contradiction that $n>1.$ Let $p$ be the smallest prime that divides $n$ ($p$ exists as $n>1$). Since $\frac{(a+1)^n - a^n}{n} \in \mathbb{Z},$ we must have $(a+1)^{n} \equiv a^{n} \pmod{p}.$ Thus (since $a$ is clearly not a multiple of $p$), $((a+1)a^{-1})^{n} \equiv 1 \pmod{p},$ so $\operatorname{ord}_p((a+1)a^{-1}) \mid n.$

Observe that from Fermat, $\operatorname{ord}_p((a+1)a^{-1}) \mid p-1.$ Thus, $\operatorname{ord}_p((a+1)a^{-1}) \mid \gcd(p-1, n).$ Notice that if $\gcd(p-1, n) > 1,$ then there is some prime $q < p$ that divides $n,$ contradicting minimality of $p.$ Thus, we must have $\gcd(p-1, n) = 1,$ so $\operatorname{ord}_p((a+1)a^{-1}) = 1.$

Thus, $$(a+1)a^{-1} \equiv 1 \pmod{p} \Longleftrightarrow a+1 \equiv a \pmod{p} \Longleftrightarrow 1 \equiv 0 \pmod{p},$$ a contradiction.

Thus, $n = 1.$ Substituting that into the original expression, we see that $$\frac{(a+1)^n-a^n}{n} = \frac{a+1-a}{1} = 1,$$ so $\frac{(a+1)^n-a^n}{n}$ is an integer whenever \ansbold{$n = 1.$}
\end{sol}

There are two main takeaways from this problem that apply to most order problems.

\begin{itemize}
\Item The modulus is the most important: Like in the previous example, notice that the method of solving this problem was ``Find $n$" - not ``find $a$" (even with primitive roots, the idea is to pick $a$ instead of finding it) It's almost always a better idea to restrict the modulus than restrict the equivalent numbers in problems like these

\Item Minimality arguments: Notice how important the fact that $p$ was the smallest prime factor of $n$ was. Without it, the problem would be much more difficult.
\end{itemize}

\section{Primitive Roots}

We know that we always have $\operatorname{ord}_n(a) \le \phi(n),$ but can we ever achieve the maximum? In other words, does there exist a value $a$ for a certain $n$ such that $\operatorname{ord}_n(a) = \phi(n)?$ What properties might this $a$ have? In order to properly discuss these numbers, we define a \textbf{primitive root}.

\begin{defi}[Primitive Root]
Let $a$ and $n$ be two positive integers. $a$ is called a primitive root modulo $n$ if and only if $\operatorname{ord}_n(a) = \phi(n).$
\end{defi}

Before discussing the applications of primitive roots, we prove that they always exist modulo $p,$ where $p$ is prime.

\begin{defi}[Polynomial Ring]
Let $\mathbb{Z}[x]$ be the ring of polynomials with integer coefficients.
\end{defi}

\begin{theo}[Lagrange]
Let $f(x) \in \mathbb{Z}[x]$ such that not all coefficients of $f$ are multiples of some prime $p.$ Then the equation $$f(x) \equiv 0 \pmod{p}$$ has at most $\operatorname{deg} f$ incongruent solutions$\pmod{p}.$
\end{theo}

\begin{pro}
We proceed with induction on $\operatorname{deg} f.$

Consider when $\operatorname{deg} f=0.$ Then, by definition, $f(x) = c,$ where $p \nmid c.$ Thus, the equation $f(x) \equiv 0 \pmod{p}$ has no solutions, so the claim holds in the base case.

Now, assume the claim holds for all polynomials of degree $m$ for some $m \in \mathbb{N}.$ We will show it holds for all polynomials of degree $m+1.$

Consider some polynomial $f(x) \in \mathbb{Z}[x]$ with degree $m+1.$ If $f(x) \equiv 0 \pmod{p}$ has no solutions, the claim holds. Otherwise, assume that there exists some constant $a$ such that $f(a) \equiv 0 \pmod{p}.$ From the definition of modular arithmetic, there exists some integer $q$ such that $f(a) - pq = 0.$ From the remainder theorem, this means $x-a \mid f(x)-pq.$

Thus, there exists some $g(x) \in \mathbb{Z}[x]$ such that $f(x) = g(x) \cdot (x-a) + pq.$ Thus, $f(x) \equiv g(x)(x-a) \pmod{p},$ and since $\operatorname{deg} (x-a) = 1,$ we have $\operatorname{deg} g = m.$ Now, notice $g(x) \equiv 0 \pmod{p}$ has at most $m$ solutions (inductive hypothesis) and $x-a \equiv 0 \pmod{p}$ has one solution. Thus, $f(x) \equiv 0 \pmod{p}$ has at most $m+1$ solutions, completing the inductive step and finishing the proof.
\end{pro}

\begin{fact}[Summing the Euler Totient Function]
Over the positive integers,
$$\sum_{d \mid n} \phi(d) = n.$$
\end{fact}
We can now show that primitive roots always exist modulo $p$ where $p$ is prime. In fact, we an prove something much stronger.

\begin{theo}[Number of Repeating Orders]
Let $p$ be a prime, and $d \mid p-1.$ Then there are exactly $\phi(d)$ elements with order $d$ modulo $p.$
\end{theo}

\begin{pro}
Consider the polynomial $x^d - 1.$ Clearly, $x^{p-1} - 1 = (x^d - 1) \frac{x^{p-1}-1}{x^d-1}.$ From the geometric series formula, $\frac{x^{p-1}-1}{x^d-1} \in \mathbb{Z}[x],$ and from Fermat, $x^{p-1}-1 \equiv 0 \pmod{p}$ has $p-1$ solutions.

Now, from Lagrange, $x^d - 1 \equiv 0 \pmod{p}$ has at most $d$ non-congruent solutions$\pmod{p},$ and $\frac{x^{p-1}-1}{x^d-1} \equiv 0 \pmod{p}$ has at most $p - d$ non-congruent solutions$\pmod{p}.$ Since $(x^d - 1) \frac{x^{p-1}-1}{x^d-1} \equiv 0 \pmod{p}$ has exactly $p$ solutions$\pmod{p},$ $x^d-1$ and $\frac{x^{p-1}-1}{x^d-1}$ must each respectively have exactly $d$ and $p-d$ non-congruent solutions$\pmod{p}.$

Let $\Omega(q)$ be the number of prime factors of $q$ counted with multiplicity, where $q \mid p-1.$ We will show by strong induction on $\Omega(q)$ that there are $\phi(q)$ non-congruent numbers which have order $q$ modulo $p.$

If $\Omega(q) = 0,$ $q = 1.$ Clearly, there is only one number with order $\phi(1) = 1$ modulo $p,$ proving the first base case.

If $\Omega(q) = 1,$ $q$ would be prime. Consider the number of solutions to $x^q - 1 \equiv 0 \pmod{p}.$ From Fact 1, we know that $x^q - 1 \equiv 0 \pmod{p}$ if and only if $\operatorname{ord}_p(x) \mid q.$ Since $q$ is prime, the number of solutions to $x^q - 1 \equiv 0 \pmod{p}$ is equal to the number of solutions to $\operatorname{ord}_p(x) = 1$ plus the number of solutions to $\operatorname{ord}_p(x) = q.$ Since there is only one $x$ such that $\operatorname{ord}_p(x) = 1$ and $x^q - 1 \equiv 0 \pmod{p}$ has $q$ solutions, there are $q-1 = \phi(q)$ numbers with order $q$ modulo $p.$ Thus, the second base case is true.

Now, assume that for all $q \mid p-1$ with $\Omega(q) \le m,$ $\operatorname{ord}_p(x) = q$ has $\phi(q)$ solutions. We will show that for any $r \mid p-1$ and $\Omega(r) = m+1,$ there are $\phi(r)$ solutions to $\operatorname{ord}_p(x) = r.$

Let the proper divisors of $r$ be $1, r_1, r_2, \dots, r_n.$ Consider the number of solutions to $x^r - 1 \equiv 0 \pmod{p}.$ We know that the number of solutions to $x^r - 1 \equiv 0 \pmod{p}$ is equal to the number of solutions to $\operatorname{ord}_p(x) = 1$ plus the number of solutions to $\operatorname{ord}_p(x) = r_1,$ $\dots,$ plus the number of solutions to $\operatorname{ord}_p(x) = r.$

Clearly, $\Omega(r_i) \le m$ for all $1 \le i \le n.$ By the inductive hypothesis, there are $\phi(r_i)$ solutions to $\operatorname{ord}_p(x) = r_i$ for all $1 \le i \le n.$ From Fact 2, it follows that there are $\phi(r)$ solutions to $\operatorname{ord}_p(x) = r,$ completing the inductive step and finishing the proof.
\end{pro}

It turns out that primitive roots exist mod $n$ if and only if $n$ is either $2, 4, p^k,$ or $2p^k,$ where $p$ is an odd prime and $k$ is a positive integer. This will turn out to be very useful.

\begin{fact}[Primitive Root Residue System]
Let $p$ be a prime and $g$ a primitive root modulo $p.$ Show that $$\{g, g^2, g^3, \dots, g^{p-1}\} \equiv \{1, 2, 3, \dots, p-1 \} \pmod{p}.$$
\end{fact}

\begin{pro}
Let $g^m$ and $g^n$ be two distinct elements in $\{g, g^2, g^3, \dots, g^{p-1}\}.$ Notice that $g^m \not\equiv g^n \pmod{p},$ as if $g^m \equiv g^n \pmod{p},$ then we would have $p-1 \mid m-n.$ Thus, all the elements in $\{g, g^2, g^3, \dots, g^{p-1}\}$ are distinct modulo $p.$

Thus, since there are $p-1$ elements in $\{g, g^2, g^3, \dots, g^{p-1}\}$ and only $p-1$ non-zero residues modulo $p,$ non-zero residues modulo $p$ is equivalent to a certain element of the set $\{g, g^2, g^3, \dots, g^{p-1}\} \pmod{p}.$ Thus, $$\{g, g^2, g^3, \dots, g^{p-1}\} \equiv \{1, 2, 3, \dots, p-1 \} \pmod{p}.$$
\end{pro}

Primitive roots an often be used to convert questions dealing with the set $\{1, 2, 3, \dots, p-1 \}$ into ones which deal with the set $\{g, g^2, g^3, \dots, g^{p-1}\}$ -- a powerful exchange for many reasons.

They are also typically used when orders aren't powerful enough to solve a problem.

\begin{exam}[Primitive Root Problem]
Find all positive two digit integers $\overline{ab}$ with $a \neq b$ such that $\overline{ab} \mid k^a-k^b$ for all integers $k.$
\end{exam}

\begin{sol}
Let $p$ be any prime that divides $\overline{ab},$ and let $g$ be a primitive root modulo $p.$

Since we have $\overline{ab} \mid k^a-k^b$ for all integers $k,$ we must have $p \mid \overline{ab} \mid g^a - g^b,$ so $g^a \equiv g^b \pmod{p}.$ Multiplying by $g^{-b},$ we get $g^{a-b} \equiv 1 \pmod{p},$ so since $\operatorname{ord}_p(g) = p-1$ (as $x$ is a primitive root modulo $p$), we have $p - 1 \mid a-b.$ Thus, since the maximum value of $|a-b|$ is $9$ and $a - b \neq 0,$ we see that  $$(p - 1)+1 \le |a-b|+1 \le 10 \Longleftrightarrow p \in \{2, 3, 5, 7\}.$$ Thus, the only primes that can divide $\overline{ab}$ when $a \neq b$ are $\{2, 3, 5, 7\},$ and if a prime $p$ divides $\overline{ab},$ $p-1 \mid a-b.$ We proceed with casework

\textbf{Case 1:} $7 \mid \overline{ab}.$

If $7 \mid \overline{ab},$ then $6 \mid a-b,$ so either $a = b+6$ or $b = a+6.$ Thus, we must have $\overline{ab} \in \{17, 28, 39, 60, 71, 82, 93\},$ but since $7 \mid \overline{ab},$ we must have $\overline{ab} = 28.$ Checking (with CRT and Euler), we see $\overline{ab} = 28$ works.

\textbf{Case 2:} $5 \mid \overline{ab}$ and $7 \nmid \overline{ab}.$

Clearly, we must have either $b = 0$ or $b = 5.$ Since $5 \mid \overline{ab},$ we have $4 \mid a-b,$ so we must have either $a = b+4,$ $a = b+8,$ $a = b-4,$ or $a = b-8.$ Thus, $\overline{ab} \in \{40, 80, 45, 85, 15\}.$ We can't have $\overline{ab} = 45$ or $85,$ as if $\overline{ab} = 45,$ then $3 \mid \overline{ab}$ but $2 \nmid a-b,$ and if $\overline{ab} = 85,$ then $17 \mid \overline{ab}.$ Now, notice that if $\overline{ab} = 40$ or $80,$ then $k^a - 1 \neq 0 \pmod{8}$ whenever $k$ is even, so we must have $\overline{ab} = 15.$ Checking (with CRT and Euler), we see $\overline{ab} = 15$ works.

\textbf{Case 3:} $3 \mid \overline{ab}$ and $5 \nmid \overline{ab}$ and $7 \nmid \overline{ab}.$

Notice that we have $\overline{ab} = 3^p \cdot 2^q,$ where $p>1.$ Thus, we have $\overline{ab} \in \{27, 81, 12, 18, 24, 36, 48, 54, 72, 96\}.$ We can't have $\overline{ab} \in \{27, 81, 12, 18, 36, 54, 72, 96\},$ as then $3 \mid \overline{ab}$ but $2 \nmid a-b.$ Thus, $\overline{ab} \in \{24, 48\}.$ Now, notice that $\overline{ab} \neq 24,$ since whenever $k \equiv 2 \pmod{8},$ $k^2 - k^4 \not\equiv 0 \pmod{8}.$ Thus, we must have $\overline{ab} = 48.$
Checking (with CRT and Euler), we see $\overline{ab} = 48$ works.

\textbf{Case 4:} $2$ is the only prime that divides $\overline{ab}.$

Notice that we must have $\overline{ab} = 2^a,$ where $a>1.$ Thus, $\overline{ab} \in \{16, 32, 64\},$ but notice that when this is true, $\overline{ab }\nmid 2^a-2^b.$ Thus, this case gives no solutions.

Thus, the solution set is \ansbold{$\overline{ab} \in \{15, 28, 48\}.$}

\end{sol}

Understand why primitive roots were used and how they were used. If we have freedom to pick the values of our variables, it is often fruitful to use primitive roots.

\pagebreak

\section{Problems}

For some of the problems presented, it may be useful to know the \textbf{Lifting The Exponent Lemma}. We will not prove the lemma here. (If you want a thorough treatment of LTE, see Raymond Feng's \db{NRU-Prime}.)

\begin{theo}[Lifting The Exponent]
Let $v_p(n)$ where $p$ is prime be the number such that $p^{v_p(n)} \mid n$ and $p ^{v_p(n)+1} \nmid n.$

\begin{itemize}
    \item If an odd prime $p \mid a-b$ but $p \nmid a$ and $p \nmid b,$ we have $$v_p(a^n - b^n) = v_p(a-b)+v_p(n).$$

    \item If an odd prime $p \mid a+b$ but $p \nmid a$ and $p \nmid b,$ we have $$v_p(a^n + b^n) = v_p(a+b)+v_p(n)$$ if $n$ is odd, and $$v_p(a^n + b^n) = 0$$ if $n$ is even.

    \item if $2 \mid x - y$ but $2 \nmid x$ and $2 \nmid y,$ then whenever $2 \mid n,$ we have $$v_2(x^n-y^n) = v_2(x-y)+v_2(n)+v_2(x+y)-1.$$
\end{itemize}
\end{theo}

\noindent\minpt{37}

\psetquote{The Mafia is grievously wounded -- but not mortally.}{Five Families}

\begin{prob}[]{1}
Show $n \nmid 2^n-1$ for all $n>1.$ (This is actually a weaker form of Example 3.)
\end{prob}

\begin{prob}[PUMaC Div. A NT 2020/1, modified]{1}
Compute the last two digits of

$7^{2020}+7^{2020^2}+\ldots +7^{2020^{2020}}$.
\end{prob}

\begin{prob}[AIME I 2019/14]{2}
Find the least odd prime factor of $2019^8+1$.
\end{prob}


\begin{req}[China TST 1993/1]{3}
For all primes $p \geq 3$ such that $p-1 \nmid 120,$ define $$F(p) = \sum^{\frac{p-1}{2}}_{k=1}k^{120}$$ and $f(p) = \frac{1}{2} - \left\{ \frac{F(p)}{p} \right\}$, where $\{x\} = x - [x],$ find the value of $f(p).$
\end{req}

\begin{prob}[Euler]{3}
Prove that all factors of $2^{2^{n}} + 1$ are of the form $k \cdot 2^{n+1} + 1.$
\end{prob}

\begin{prob}[PUMaC Div. A NT 2016/7]{3}
Compute the number of positive integers $n$ between $2017$ and $2017^2$ such that $n^n \equiv 1\pmod{2017}$, given that $2017$ is prime.
\end{prob}

\begin{req}[]{4}
Suppose $p$ is a prime such that there exists an integer $q$ such $q^2 \equiv -3 \pmod{p}.$ Find all solutions to $x^3 \equiv 1 \pmod{p}$ in terms of $p$ and $q.$
\end{req}

\begin{prob}[Taiwan TST Quiz 2014/3J/2]{6}
Alice and Bob play the following game. They alternate selecting distinct nonzero digits (from $1$ to $9$) until they have chosen seven such digits, and then consider the resulting seven-digit number by concatenating the digits in the order selected, with the seventh digit appearing last (i.e. $\overline{A_1B_2A_3B_4A_6B_6A_7}$). Alice wins if and only if the resulting number is the last seven decimal digits of some perfect seventh power. Please determine which player has the winning strategy.
\end{prob}

\begin{prob}[ISL 2006/N5]{6}
Find all integral solutions to $\frac {x^{7} - 1}{x - 1} = y^{5} - 1.$
\end{prob}

\begin{prob}[DIME 2020/14]{6}
For a positive integer $n$ not divisible by $211$, let $f(n)$ denote the smallest positive integer $k$ such that $n^k - 1$ is divisible by $211$. Find the remainder when $$\sum_{n=1}^{210} nf(n)$$ is divided by $211$.
\end{prob}

\begin{prob}[IMO 1990/3, part of IMO 1999/4]{9}
Find all positive integers $n$ such that $n^2 \mid 2^n+1.$
\end{prob}

\begin{prob}[IMO 1999/4]{6}
Find all the pairs of positive integers $(x,p)$ such that p is a prime, $x \leq 2p$ and $x^{p-1}$ is a divisor of $ (p-1)^{x}+1$.
\end{prob}

\begin{prob}[Weak Dirichlet]{9}
Prove that there are infinitely many primes $p\equiv 1\pmod{k}$ for any positive integer $k$.
\end{prob}

\begin{prob}[ISL 2012/N6]{9}
Let $x$ and $y$ be positive integers. If ${x^{2^n}}-1$ is divisible by $2^ny+1$ for every positive integer $n$, prove that $x=1$.
\end{prob}

\begin{prob}[ISL 2003/N7]{13}
The sequence $a_0$, $a_1$, $a_2,$ $\ldots$ is defined as follows:
\[a_0=2, \qquad a_{k+1}=2a_k^2-1 \quad\text{for }k \geq 0.\]
Prove that if an odd prime $p$ divides $a_n$, then $2^{n+3}$ divides $p^2-1$.
\end{prob}

% b| phi(a^b-1), isl 2003 n7 isl 2012 n6, infinite 1 mod k primes
\end{document}
