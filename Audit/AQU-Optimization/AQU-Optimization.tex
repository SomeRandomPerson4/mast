\documentclass{article}
\usepackage[mast]{dennis}

\title{Optimization and Basic Inequalities}
\author{William Zhao}
\date{AQU}

\begin{document}

\maketitle

\section{Optimization}

In these types of problems, we usually have a list of numbers or actions and want to minimize of maximize some output. In many cases we try to perturb one of the properties or variables given in the problem, or set them to an extreme value and see what results we have. More often than not you will see such problems take the form of a word problem with a lengthy exposition. Sometimes the heart of the problem is simply solving some system of equations (usually that has multiple solutions) and providing the least value of some expression.  

\subsection{Examples}
\begin{exam}[AIME 1983/2]
Let $f(x)=|x-p|+|x-15|+|x-p-15|$, where $0<p<15$. Determine the minimum value taken by $f(x)$ for $x$ in the interval $p\le x\le 15$. 
\end{exam}
\begin{sol}
Notice that $|x-p|=x-p$, $|x-15|=15-x$, and $|x-p-15|=15+p-x$.

Adding these together, we find that the sum is equal to $30-x$, which attains its minimum value (on the given interval $p \leq x \leq 15$) when $x=15$, giving a minimum of $\boxed{015}$.
\end{sol}

\begin{exam}[AIME II 2009/4]
A group of children held a grape-eating contest. When the contest was over, the winner had eaten $n$ grapes, and the child in $k$-th place had eaten $n+2-2k$ grapes. The total number of grapes eaten in the contest was $2009$. Find the smallest possible value of $n$.
\end{exam}
\begin{sol}
Let there be $m$ children. The number of grapes eaten by the children collectively will be $(n)+(n-2)+(n-4)+\cdots+(n+2-2m)=\frac12m(n+n+2-2m)=m(n+1-m)$. Note that $2009=7^2\cdot 41$. Thus, there are only a limited number of values that $m$ can take; testing each value, we conclude that the minimum $n$ is achieved at $m=41$ and $n=89$. 
\end{sol}

\begin{exam}[AIME II 2003/12]
The members of a distinguished committee were choosing a president, and each member gave one vote to one of the 27 candidates. For each candidate, the exact percentage of votes the candidate got was smaller by at least 1 than the number of votes for that candidate. What was the smallest possible number of members of the committee?
\end{exam}
\begin{sol}
Let $v_i$ be the number of votes that candidate $i$ receives, and $v$ the total number of votes. The given condition implies $100\frac{v_i}v+1\le v_i\implies v\ge \frac{100v_i}{v_i-1}$ for all $1\le i\le 27$. Note that $v_i\ge 2$. If at least of $v_i$ is 2, then $v\ge 200$. If at least one of $v_i$ is 3, then $s\geq 150$. If at least one of $v_i$ is 4, then $s\geq \frac{400}3$, and hence $s\geq 134$. 
If $s<134$, then $v_i\ge 5$ for all $i$. However this implies $v=\sum_{i=1}^{27}v_i\ge 27\cdot 5=135$, contradiction. Hence $v=134$ is the minimum.
Suppose all but one candidate received $5$ votes, and the last received $4$. Then $v=134$, and the given criterion is satisfied. Thus $134$ is the minimum number of voters in the committee.  
\end{sol}

\section{Algebraic Inequalities}

Very rarely we can finish such problems directly or indirectly with well known inequalities. Usually these problems have a more algebraic taste. 

\begin{theo}[Trivial Inequality] 
For any real number $x$, $x^2\ge0$, with equality at $x=0$. 
\end{theo}
Take a moment and convince yourself that this is true. 

\begin{theo}[Arithmetic Mean-Geometric Mean Inequality] 
For any real numbers $x_1,x_2,\ldots,x_n\ge0$, \[\frac{\sum_{i=1}^nx_i}n\ge\sqrt[n]{\prod_{i=1}^nx_i}. \] Equality holds for $x_1=x_2=\cdots=x_n$. 
\end{theo}
This is a relatively elegant and complicated proof, featuring a very special form of induction called Cauchy-Induction. You do not need to remember, or even read, this proof (at this level), though convince yourself you could understand it if you wanted to. Also note that this proof is moderately difficult to motivate. This inequality is often abbreviated as "AM-GM".  
\begin{pro}
First, we prove that the inequality is true for the base case $n=2$. Then, we show that if the inequality is true for $n=k$, it is too for $n=2k$. Finally, if it is true for $n=k$, it is also true for $n=k-1$.
\\[1\baselineskip]
For $n=2$, the inequality rearranges to \[\frac{a+b}2\ge\sqrt{ab}\implies a^2+2ab+b^2\ge 4ab,\] or $(a-b)^2\ge0$.
\\[1\baselineskip]
Now assume that this inequality is true for $n=k$; ie. $\frac{x_1+x_2+\cdots+x_k}n\ge\sqrt[n]{x_1x_2\cdots x_k}$. Then, we have
\begin{align*}
\sqrt[2k]{x_1x_2\cdots x_{2k}}&=\sqrt{\sqrt[n]{x_1x_2\cdots x_{k}}\sqrt[k]{x_{k+1}x_{k+2}\cdots x_{2k}}}\\&\le \frac{\sqrt[k]{x_1x_2\cdots x_k}+ \sqrt[k]{x_1x_2\cdots x_{2k}}}2\\&\le \frac{\frac{x_1+x_2+\cdots+x_k}k+\frac{x_{k+1}+x_{k+2}+\cdots+x_{2k}}k}2\\&=\frac{x_1+x_2+\cdots+x_{2k}}{2k}.
\end{align*}
Thus, the inequality is true for $n=2k$. 
Again assume the inequality is true for $n=k$. Take $x_k=\frac{\sum\limits_{i=1}^{k-1}x_i}{k-1}$. Note that $\frac1k\sum\limits_{i=1}^kx_i=\frac1{k-1}\sum\limits_{i=1}^{k-1}x_i$. 
\begin{align*}
    \frac1{k-1}\sum_{i=1}^{k-1}x_i&=\frac1k\sum_{i=1}^kx_i\\&\ge \sqrt[k]{\prod_{i=1}^kx_i}\\&=\sqrt[k]{\frac{1}{k-1}\left(\sum_{i=1}^{k-1}x_i\right)\left(\prod_{i=1}^{k=1}x_i\right)}\\\implies \left(\frac1{k-1}\sum_{i=1}^{k-1}x_i\right)^{k}&\ge\frac{1}{k-1}\left(\sum_{i=1}^{k-1}x_i\right)\left(\prod_{i=1}^{k=1}x_i\right)\\\implies \left(\frac1{k-1}\sum_{i=1}^{k-1}x_i\right)^{k-1}&\ge \prod_{i=1}^{k-1}x_i\\\implies \frac1{k-1}\sum_{i=1}^{k-1}x_i&\ge \sqrt[n-1]{\prod_{i=1}^{k-1}x_i}.
\end{align*}
This implies the inequality is true for $n=k-1$. Our induction is complete. 
\end{pro}

\begin{theo}[Cauchy Schwarz Inequality] 
For any real numbers $x_1,x_2,\ldots,x_n$, $y_1,y_2,\ldots,x_n$,  \[\left(\sum_{i=1}^n x_i^2\right)\left(\sum_{i=1}^n y_i^2\right)\ge\left(\sum_{i=1}^n x_iy_i\right)^2.\] Equality holds for $\frac{x_1}{y_1}=\frac{x_2}{y_2}=\cdots=\frac{x_n}{y_n}$. 
\end{theo}
This inequality is often abbreviated "C-S". We present two of the more elegant proofs. The first proof is by considering the discriminant of a cleverly defined function. Again, skip if you would like.
\begin{pro}
Let $f(x)=\sum_{i=1}^n (x_ix-y_i)^2=\left(\sum_{i=1}^n x_i^2\right)x^2-2\left(\sum_{i=1}^n x_iy_i\right)x+\left(\sum_{i=1}^n y_i^2\right). $ By the trivial inequality, $f(x)\ge 0$. This means the discriminant of $f(x)$ is greater or equal to zero, which simplifies down to the desired inequality. Obviously equality holds if $f(x)=0\implies \frac{x_1}{y_1}=\frac{x_2}{y_2}=\cdots=\frac{x_n}{y_n}$. 
\end{pro}

The second proof is using an identity called the Cauchy-Schwarz expansion that solves the inequality immediately. 
\begin{pro}
Note that
\[\left(\sum_{i=1}^n x_i^2\right)\left(\sum_{i=1}^ny_i^2\right)-\left(\sum_{i=1}^n x_iy_i\right)^2=\sum_{1\leq i,j\leq n}(x_iy_i-x_jy_i)^2.\] Since the expression is obviously greater than $0$, we are done.
\end{pro}

\begin{theo}[Triangle Inequality] 
Given a triangle with side lengths $a,b,c$, it is non-degenerate if and only if $a+b>c$, $b+c>a$, and $c+a>b.$
\end{theo}
Take a moment and convince yourself that this is true. 
\subsection{Examples}

\begin{exam}[AIME 1991/3]
Expanding $(1+0.2)^{1000}$ by the binomial theorem and doing no further manipulation gives
\[\binom{1000}0(0.2)^0+\binom{1000}1(0.2)^1+\cdots+\binom{1000}{1000}(0.2)^{1000}= A_0 + A_1 + A_2 + \cdots + A_{1000},\]
where $A_k = \binom{1000}k(0.2)^k$ for $k = 0,1,2,\ldots,1000$. For which $k$ is $A_k$ the largest?
\end{exam}
\begin{sol}
We are looking for the smallest $k$ such that \[\frac1{5^k}\cdot\binom{1000}k>\frac1{5^{k+1}}\cdot \binom{1000}{k+1}\implies \binom{1000}k>\frac15\cdot \binom{1000}{k+1}.\] Expanding these binomial coefficients, we have \[\frac{1000!}{(1000-k)!\cdot k!}>\frac15\cdot\frac{1000!}{(1000-k-1)!\cdot (k+1)!}. \] Simplifying, we must have $\frac1{1000-k}>\frac1{5k+5}\implies k>165.8$, so our desired answer is $166$. 
\end{sol}

\begin{exam}[AIME 1983/9]
Find the minimum value of $\frac{9x^2\sin^2 x + 4}{x\sin x}$ for $0 < x < \pi$.
\end{exam}
\begin{sol}
By AM-GM, $9x\sin x+\frac{4}{x\sin x}\ge 2\cdot 6=12.$ Note that equality is achieved at $x\sin x=\frac23$, which is attainable in the range $0<x<\pi\implies 0<x\sin x<\frac{\pi}2$. 
\end{sol}

\begin{exam}
For which real value of $x$ is the function $f(x)=ax^2+bx+c$ minimized? Assume $a>0$ and the coefficients are real. 
\end{exam}
\begin{sol}
Completing the square, we have $f(x)=a\left(x+\frac{b}{2a}\right)^2-\frac{b^2}{4a}+c$. By the trivial inequality, this expression is minimized when $a\left(x+\frac{b}{2a}\right)^2=0\implies x=-\frac{b}{2a}$. 
\end{sol}

\begin{exam}[WOOT]
{Let $x$ be real. Find the maximum value of $f(x)=x^3(4-x)$. }
\end{exam}
\begin{sol}
By AM-GM, \[f(x)=\frac13[x\cdot x\cdot x\cdot (12-3x)]\le \frac13\cdot \left[\frac{x+x+x+(12-3x)}4\right]^4=27.\]Equality holds if $x=x=x=12-3x$. Indeed, $f(3)=27$, so the maximum value is 27. 
\end{sol}
\pagebreak

\section{Problems}
\minpt{40}

\begin{req}[WOOT]{2}
In Example 7 we showed that the maximum value of $f(x)=x^3(4-x)$ is 27. However, we could have manipulated the inequality in many different ways. For instance, we could have written \[f(x)=\frac 1{36}[x(2x)(3x)(24-6x)]\le\frac1{36}\left[\frac{x+2x+3x+24-6x}4\right]^4=36.\] So why is $\max(f(x))=27$ and not 36?
\end{req} 

\begin{prob}[]{2}
Show that for any $n\ge1$, the inequality $\sqrt{6+\cdots \sqrt{6+\sqrt{6}}}<3$ holds, where there are $n$ nested radicals. 
\end{prob} 

\begin{prob}[AIME II 2004/6]{3}
Three clever monkeys divide a pile of bananas. The first monkey takes some bananas from the pile, keeps three-fourths of them, and divides the rest equally between the other two. The second monkey takes some bananas from the pile, keeps one-fourth of them, and divides the rest equally between the other two. The third monkey takes the remaining bananas from the pile, keeps one-twelfth of them, and divides the rest equally between the other two. Given that each monkey receives a whole number of bananas whenever the bananas are divided, and the numbers of bananas the first, second, and third monkeys have at the end of the process are in the ratio $3: 2: 1,$what is the least possible total for the number of bananas?
\end{prob} 

\begin{prob}[AMC 10A 2019/19]{3}
What is the least possible value of\[(x+1)(x+2)(x+3)(x+4)+2019\]where $x$ is a real number?
\end{prob}  

\begin{prob}[AIME I 2010/5]{3}
Positive integers $a$, $b$, $c$, and $d$ satisfy $a > b > c > d$, $a + b + c + d = 2010$, and $a^2 - b^2 + c^2 - d^2 = 2010$. Find the number of possible values of $a$.
\end{prob} 

\begin{prob}[AIME I 2004/5]{4}
Alpha and Beta both took part in a two-day problem-solving competition. At the end of the second day, each had attempted questions worth a total of 500 points. Alpha scored 160 points out of 300 points attempted on the first day, and scored 140 points out of 200 points attempted on the second day. Beta who did not attempt 300 points on the first day, had a positive integer score on each of the two days, and Beta's daily success rate (points scored divided by points attempted) on each day was less than Alpha's on that day. Alpha's two-day success ratio was 300/500 = 3/5. The largest possible two-day success ratio that Beta could achieve is $m/n,$ where $m$ and $n$ are relatively prime positive integers. What is $m+n$?
\end{prob} 

\begin{prob}[]{4}
Let $\mathcal{S}$ be the set of all real numbers $x$ such that $\lfloor x^2\rfloor=\lfloor x\rfloor\lfloor x+1\rfloor$. Let $m$ be the minimum positive integer for which $m+0.51$ is not in $\mathcal{S}$. Find $m$. 
\end{prob} 

\begin{prob}[Andreescu]{4}
Let $P$ be a polynomial with positive coefficients. Show that if \[P\left(\frac1x\right)\ge \frac1{P(x)}\] holds for $x=1$, it holds for all $x>0$. 
\end{prob} 

\begin{req}[AIME 1990/11]{6}
Someone observed that $6! = 8 \cdot 9 \cdot 10$. Find the largest positive integer $n^{}_{}$ for which $n^{}_{}!$ can be expressed as the product of $n - 3_{}^{}$ consecutive positive integers.
\end{req} 

\begin{prob}[AIME II 2011/9]{6}
Let $x_1, x_2, ... , x_6$ be non-negative real numbers such that $x_1 +x_2 +x_3 +x_4 +x_5 +x_6 =1$, and $x_1 x_3 x_5 +x_2 x_4 x_6 \ge {\scriptstyle\frac{1}{540}}$. Let $p$ and $q$ be positive relatively prime integers such that $\frac{p}{q}$ is the maximum possible value of $x_1 x_2 x_3 + x_2 x_3 x_4 +x_3 x_4 x_5 +x_4 x_5 x_6 +x_5 x_6 x_1 +x_6 x_1 x_2$. Find $p+q$.
\end{prob} 

\begin{req}[AIME I 2008/12]{6}
On a long straight stretch of one-way single-lane highway, cars all travel at the same speed and all obey the safety rule: the distance from the back of the car ahead to the front of the car behind is exactly one car length for each 15 kilometers per hour of speed or fraction thereof (Thus the front of a car traveling 52 kilometers per hour will be four car lengths behind the back of the car in front of it.) A photoelectric eye by the side of the road counts the number of cars that pass in one hour. Assuming that each car is 4 meters long and that the cars can travel at any speed, let $M$ be the maximum whole number of cars that can pass the photoelectric eye in one hour. Find the quotient when $M$ is divided by $10$.
\end{req} 

\begin{prob}[AIME II 2009/11]{9}
For certain pairs $(m,n)$ of positive integers with $m\geq n$ there are exactly $50$ distinct positive integers $k$ such that $|\log m - \log k| < \log n$. Find the sum of all possible values of the product $mn$.
\end{prob} 

\begin{prob}[]{9}
A cubic polynomial has the property that all its roots are rational and its coefficients are positive prime integers. If $p(1)=144$, compute the value of $p(2)$. 
\end{prob} 

\begin{prob}[AIME 1998/14]{9}
An $m\times n\times p$ rectangular box has half the volume of an $(m + 2)\times(n + 2)\times(p + 2)$ rectangular box, where $m, n,$ and $p$ are integers, and $m\le n\le p.$ What is the largest possible value of $p$?
\end{prob} 

\begin{prob}[]{13}
Triangle $ABC$ has sides $a$, $b$, and $c$, and circumradius $R$. Prove that
\[b^2 + c^2 \ge a^2 - R^2.\]
When does equality occur?
\end{prob}

%\begin{prob}[ISL 2005/A1]{13}
%Find all pairs of integers $a,b$ for which there exists a polynomial $P(x) \in \mathbb{Z}[X]$ such that product $(x^2+ax+b)\cdot P(x)$ is a polynomial of a form\[ x^n+c_{n-1}x^{n-1}+\cdots+c_1x+c_0  \]where each of $c_0,c_1,\ldots,c_{n-1}$ is equal to $1$ or $-1$.
%\end{prob}
\end{document}