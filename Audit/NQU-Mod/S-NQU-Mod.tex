
\documentclass[mast]{lucky}

\title{Solutions to Modular Arithmetic}
\author{MAST}
\date{NQU}

\begin{document}

\maketitle

\toc

\pagebreak\section{Unsourced}

Find the inverse of $2\pmod {p}$ for odd prime $p$ in terms of $p.$

\subsection{Solution}

\pagebreak\section{Unsourced}

Find the remainder of $97!$ when divided by $101.$

\subsection{Solution}

\pagebreak\section{Unsourced}

Find the remainder of $(p-2)!$ when divided by $p,$ provided that $p$ is prime.

\subsection{Solution}

\pagebreak\section{AMC 12A 2003/18}

Let $n$ be a $5$-digit number, and let $q$ and $r$ be the quotient and the remainder, respectively, when $n$ is divided by $100$. For how many values of $n$ is $q+r$ divisible by $11$?

\subsection{Solution}

\pagebreak\section{MAST Diagnostic 2020}

How many integer values of $1\leq x\leq 100$ makes $x^2+8x+5$ divisible by $10?$

\subsection{Solution}

\pagebreak\section{1001 Problems in Number Theory}

For which positive integers $n$ is it true that $1+2+\cdots+n\mid 1\cdot 2\cdot \cdots \cdot n$?

\subsection{Solution}

\pagebreak\section{Unsourced}

What is the residue of $\frac{1}{1\cdot 2}\cdot \frac{1}{2\cdot 3}\cdot \dots \cdot \frac{1}{11\cdot 12}\pmod {13}?$

\subsection{Solution}

\pagebreak\section{AMC 10A 2020/18}

Let $(a,b,c,d)$ be an ordered quadruple of not necessarily distinct integers, each one of them in the set ${0,1,2,3}.$ For how many such quadruples is it true that $a\cdot d-b\cdot c$ is odd? (For example, $(0,3,1,1)$ is one such quadruple, because $0\cdot 1-3\cdot 1 = -3$ is odd.)

\subsection{Solution}

\pagebreak\section{AMC 10B 2018/16}

Let $a_1,a_2,\dots,a_{2018}$ be a strictly increasing sequence of positive integers such that\[a_1+a_2+\cdots+a_{2018}=2018^{2018}.\]What is the remainder when $a_1^3+a_2^3+\cdots+a_{2018}^3$ is divided by $6$?

\subsection{Solution}

\pagebreak\section{PUMaC 2018}

Find the number of positive integers $n<2018$ such that $25^n+9^n$ is divisible by $13.$

\subsection{Solution}

\pagebreak\section{Unsourced}

Prove $\phi(n)$ is composite for $n\geq 7.$

\subsection{Solution}

\pagebreak\section{AMC 10B 2019/14}

The base-ten representation for $19!$ is $121,6T5,100,40M,832,H00$, where $T$, $M$, and $H$ denote digits that are not given. What is $T+M+H$?

\subsection{Solution}

\pagebreak\section{Unsourced}

Find the remainder of $5^{31}+5^{17}+1$ when divided by $31.$

\subsection{Solution}

\pagebreak\section{OMO 15-16 Spring/9}

Let $f(n)=1 \times 3 \times 5 \times \cdots \times (2n-1)$. Compute the remainder when $f(1)+f(2)+f(3)+\cdots +f(2016)$ is divided by $100.$

\subsection{Solution}

\pagebreak\section{Unsourced}

Prove that the equation $x^2+y^2+z^2=x+y+z+1$ has no solutions over the rationals.

\subsection{Solution}

\pagebreak\section{MAST Diagnostic 2021}

Find the remainder of $(1^3)(1^3+2^3)(1^3+2^3+3^3)\dots(1^3+2^3+3^3\dots+99^3)$ when divided by $101.$

\subsection{Solution}

\pagebreak\section{Wolstenholme's Theorem}

Prove that for all prime $p\ge5$, we have $p^2\mid (p-1)!\left(\sum\limits_{i=1}^{p-1}\frac1{i}\right)$.

\subsection{Solution}

\pagebreak\section{AIME 1989/9}

One of Euler's conjectures was disproved in the 1960s by three American mathematicians when they showed there was a positive integer such that $133^5+110^5+84^5+27^5=n^{5}$. Find the value of $n$.

\subsection{Solution}

\pagebreak\section{USAMO 1979/1}

Determine all non-negative integral solutions $(n_1, n_2, \dots , n_k)$, if any, apart from permutations, of the Diophantine equation
$$n_1^4 + n_2^4 + \cdots + n_{14}^{4} = 1599.$$

\subsection{Solution}

\pagebreak\section{AIME II 2017/8}

Find the number of positive integers $n$ less than $2017$ such that
\[ 1+n+\frac{n^2}{2!}+\frac{n^3}{3!}+\frac{n^4}{4!}+\frac{n^5}{5!}+\frac{n^6}{6!} \]is an integer.

\subsection{Solution}

\pagebreak\section{IMO 1970/4}

Find all positive integers $n$ such that the set $\{n,n+1,n+2,n+3,n+4,n+5\}$ can be partitioned into two subsets so that the product of the numbers in each subset is equal.

\subsection{Solution}

\pagebreak\section{IMO 2005/4}

Determine all positive integers relatively prime to all the terms of the infinite sequence
\[a_n=2^n+3^n+6^n -1,\ n\geq 1.\]

\subsection{Solution}

\pagebreak\section{AIME I 2013/15}

Let $N$ be the number of ordered triples $(A,B,C)$ of integers satisfying the conditions
\begin{itemize}
\item $0\le A<B<C\le99$,
\item there exist integers $a$, $b$, and $c$, and prime $p$ where $0\le b<a<c<p$,
\item $p$ divides $A-a$, $B-b$, and $C-c$, and
\item each ordered triple $(A,B,C)$ and each ordered triple $(b,a,c)$ form arithmetic sequences.
\end{itemize}
Find $N$.

\subsection{Solution}

\pagebreak\section{USEMO 2019/4}

Prove that for any prime $p,$ there exists a positive integer $n$ such that
\[1^n+2^{n-1}+3^{n-2}+\cdots+n^1\equiv 2020\pmod{p}.\]

\subsection{Solution}

\pagebreak\section{Unsourced}

The expansion of $\frac{1}{7}$ is $0.\overline{142857},$ which is a repeating decimal with a $6$ digit long sequence. How many digits long is the expansion of $\frac{1}{13}?$

\subsection{Solution}

\pagebreak\section{Unsourced}

We define the cycle of a repeating fraction $\tfrac{m}{n}$ as the minimum number $i$ such that $\tfrac{m}{n} = 0.\overline{a_1a_2a_3\dots a_i}$. Find the cycle of $\tfrac{1}{23}$.

\subsection{Solution}

\pagebreak\section{AMC 10A 2019/18}

For some positive integer $k$, the repeating base-$k$ representation of the (base-ten) fraction $\frac{7}{51}$ is $0.\overline{23}_k = 0.232323\ldots_k$. What is $k$?

\subsection{Solution}

\pagebreak\section{e-dchen Mock MATHCOUNTS}

What is the sum of all odd $n$ such that $\frac{1}{n}$ expressed in base $8$ is a repeating decimal with period $4?$

\subsection{Solution}

\pagebreak\section{AMC 12A 2014/23}

The fraction\[\dfrac1{99^2}=0.\overline{b_{n-1}b_{n-2}\ldots b_2b_1b_0},\]where $n$ is the length of the period of the repeating decimal expansion. What is the sum $b_0+b_1+\cdots+b_{n-1}$?

\subsection{Solution}

\pagebreak\section{AMC 12B 2016/22}

For a certain positive integer $n$ less than $1000$, the decimal equivalent of $\frac{1}{n}$ is $0.\overline{abcdef}$, a repeating decimal of period $6$, and the decimal equivalent of $\frac{1}{n+6}$ is $0.\overline{wxyz}$, a repeating decimal of period $4$. Find $n.$

\subsection{Solution}

\end{document}