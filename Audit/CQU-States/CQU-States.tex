\documentclass[mast]{lucky}



\title{States}
\author{Dennis Chen}
\date{CQU}

\begin{document}

\maketitle

Independent probability problems are very easy. Dependent probability problems are not so easy. We introduce some techniques to solve dependent probability problems by considering the possible states of the problem.

\section{Probability}
States are hard to explain without examples, so we will start with them first before explaining the theory.

\begin{exam}
You are flipping a fair coin. What is the probability you get three heads in a row before you get two tails?
\end{exam}

This one isn't hard (and probably doesn't even need states). Thus we present an example that doesn't need states to make states easier to grasp.

\begin{sol}[1]
We start out with two attempts to flip three heads in a row, which succeeds with probability $\frac{1}{8}.$ If we fail, that means we've flipped tails and lose one of our two attempts.

The amount of ways to succeed are with one or two attempts. With one attempt, the probability is $\frac{1}{8}$ and with two, the probability is $\frac{7}{8}\cdot\frac{1}{8}.$ Thus the total probability is $\frac{15}{64}.$
\end{sol}

Now let's do this with states!

\begin{sol}[2]
Let $P_n$  be the probability of success with $n$ tails flipped. We want to find $P_0.$ Notice that $P_0=\frac{1}{8}\cdot 1+\frac{7}{8}P_1,$ as we have a $\frac{1}{8}$ chance of going to the success case. By states, $P_1=\frac{1}{8}\cdot 1+\frac{7}{8}\cdot 0.$

The rest of the solution is just arithmetic.
\end{sol}
Just because we don't succeed if we flip tails at $P_0$ doesn't mean we failed. You don't \textit{fail} until there's literally no way to succeed.\footnote{Which occurs when we flip tails at $P_1.$} Flipping tails at $P_0$ only sends us to $P_1.$

States are hard and can get algebraically involved even if done right. The unofficial mantra is \textit{keep it simple, stupid}.\footnote{I've seen people try to define states of $P_{(H,T)},$ where $H$ is the amount of heads flipped in a row and $T$ is the amount of tails flipped.}
\begin{theo}[Probability in States]
Consider the probability of some event occurring at an initial state $S.$ Let the probability that you are sent from $S$ to $S_i$ be $x_i,$ and let the probability of success at $S_i$ be $P_i.$ Then the probability of success at $S$ is $\sum\limits_{i=1}^{\infty}P_ix_i.$
\end{theo}

A couple of reminders:

\begin{enumerate}
    \item Remember that $\sum\limits_{i=1}^{\infty}x_i=1.$
    
    \item If $S_i$ doesn't exist or can't happen, then set $x_i$ as $0.$
    
    \item For success and failure cases, set $P_i$ as $1$ and $0,$ respectively.
    
    \item Most importantly, \textbf{this theorem applies to all $S_i.$} Since the success and failure states will eventually appear, you can use this to set a system of equations and solve for the probability of success at $S.$
    
    \item It's okay to complementary count. The success and failure states are arbitrary. One man's success is another's failure.
\end{enumerate}
\section{Expected Value}

Let's talk about expected value with a simple question.

\begin{exam}
On average, how many times do you have to flip a coin until it lands heads?
\end{exam}

The first (and most naive) method one could think of is creating a summation.\footnote{In fact, this can be useful. At the very least, it's an interesting way to relate certain summations with expected value.} This summation would look like $\sum_{n=1}^{\infty}\frac{n}{2^n}.$ You could evaluate the summation to get $2.$

We should notice that $2$ is the reciprocal of $\frac{1}{2}.$ Does the same hold for other probabilities? How many times must we roll a die before we expect to land on $1?$\footnote{The answer is unsurprisingly $6.$}

To do this, let's use the idea of states. If we flip heads on our first go, then great! We're done! But if we flip tails, \textbf{we're back where we started.} Let's try to encode this into states.

\begin{sol}
Let $E$ be the expected amount of flips to land heads. Then notice $E=(\frac{1}{2}\cdot 0+\frac{1}{2}\cdot E)+1.$ Solving, $E=2.$
\end{sol}
Let's decode what the stuff in the parentheses means. If we flip heads, which happens with a $\frac{1}{2}$ chance, the expected amount of flips to get heads is $0$ (since we already flipped heads). And if we flip tails, which happens with $\frac{1}{2}$ chance, we're back where we started, so the expected amount of flips to get heads is still $E.$ And of course, no matter what happens, we will always have used up a step, so we must add $1$ to the expected value.

Can we generalize? Yes. The theorem below shall do so.

\begin{theo}[Expected Tries for Single Event]
If you have an event that succeeds with probability $p$ every step, the expected amount of steps it takes to succeed once is $\frac{1}{p}.$
\end{theo}

\begin{pro}
Let $E$ be the expected amount of steps it takes. Then $E=(p\cdot 0+(1-p)\cdot E)+1.$ Solving, $E=\frac{1}{p}.$
\end{pro}

And how many steps does it take to, say, succeed twice?

\begin{theo}[Linearity of Expectation]
If you have an event that succeeds with probability $p$ every step, the expected amount of steps it takes to succeed $n$ times is $\frac{n}{p}.$
\end{theo}

\begin{pro}
We induct. This is proven to be true for the base case. Now assume it is true for $n.$ Let $E[n]$ denote the expected amount of steps to succeed $n$ times. Then notice $E[n+1]=(p\cdot E[n]+(1-p)\cdot E[n+1])+1.$ Substitute in $\frac{n}{p}$ for $E[n]$ and you get $E[n+1]=(pE[n]+(1-p)\cdot E[n+1])+1.$ Solving, $E[n+1]=\frac{n+1}{p},$ as desired.
\end{pro}

We've answered the obvious questions. Now let's show the power of states with something a bit harder: \textit{consecutive} successes.

\begin{exam}[Two Heads Up]
How many times must you flip a coin before you expect it to land heads twice in a row?
\end{exam}

The naive answer is $4$ (and it is the wrong answer!) and the reasoning for it being wrong is simple. The expected amount of flips should not be the same as having it land heads twice (without having to be consecutive). We use states to do this instead.

\begin{sol}
Let $E_n$ be the amount of times you expect to flip a coin for it to land heads $n$ times in a row. Then $E_2=(\frac{1}{2}\cdot E_1+\frac{1}{2}\cdot E_2)+1$ and $E_1=(\frac{1}{2}\cdot 0+\frac{1}{2}\cdot E_2)+1.$ Simplifying both of these equations, $E_2=E_1+2$ and $2E_1=E_2+2.$ Substituting $E_2$ into the second equation yields $2E_1=E_1+2+2=E_1+4,$ so $E_1=4.$ Substituting that into the first equation yields $E_2=4+2=6,$ so the expected amount of flips to get two heads in a row is $6.$
\end{sol}

The generalization for expected value is very similar to probability (just add a $1$ to the end).

\begin{theo}[Expected Value in States]
Consider the probability of some event occurring at an initial state $S.$ Let the probability that you are sent from $S$ to $S_i$ be $x_i,$ and let the expected amount of steps at $S_i$ be $E_i.$ Then the expected amount of steps to success at $S$ is $\sum\limits_{i=1}^{\infty}E_ix_i+1.$
\end{theo}

A couple of reminders:

\begin{enumerate}

    \item Remember that $\sum\limits_{i=1}^{\infty}x_i=1.$
    
    \item If $S_i$ doesn't exist or can't happen, then set $x_i$ as $0.$
    
    \item For the success case, set $E_i$ as $0.$
    
    \item Most importantly, \textbf{this theorem applies to all $S_i.$} Since the success state will eventually appear, you can use this to set a system of equations and solve for the expected value until success at $S.$
    
    \item There is only one success condition. You cannot complementary count.
\end{enumerate}

\section{Various Examples}
We present more examples to solidify the reader's understanding of states.

\begin{exam}[BMT Discrete 2019/1]
A fair coin is repeatedly flipped until $2019$ consecutive coin flips are the same. Compute the probability that the first and last flips of the coin come up differently.
\end{exam}

\begin{sol}
We complementary count because that makes the initial state less annoying.

Say without loss of generality the first flip comes up heads. Notice that if a run of heads breaks, we have $1$ tail and thus need to flip $2018$ more tails in a row, and vice versa. Therefore, let $p_1$ represent the probability of success when we have a run of one head at the end and let $p_2$ represent the probability of success when we have a run of one tail at the end, and note that
\[p_1=\frac{1}{2^{2018}}+(1-\frac{1}{2^{2018}})p_2\]
\[p_2=(1-\frac{1}{2^{2018}})p_1,\]
implying that
\[p_1=\frac{1}{2^{2018}}+(1-\frac{1}{2^{2018}})^2p_1\]
\[p_1(2-\frac{1}{2^{2018}})(\frac{1}{2^{2018}})=\frac{1}{2^{2018}}\]
\[p_1=\frac{2^{2018}}{2^{2019}-1}.\]

So the answer is $\frac{2^{2018}-1}{2^{2019}-1}.$
\end{sol}

\begin{exam}
Consider a number line with integers $0,1,2\dots n.$ Every second, a particle initially at the origin randomly moves to an adjacent integer. (If it is at $0,$ it goes to $1$ all the time.) In terms of $n,$ find the expected amount of time for the particle to reach $n.$
\end{exam}

\begin{sol}
We claim the answer is $n^2.$
    
    Let $E(k)$ denote the expected amount of seconds required to move from $k$ to $n.$ Note that $E(0)=E(1)+1,$ $E(n)=0,$ and for all other $k,$ $E(k)=\frac{1}{2}(E(k-1)+E(k+1))+1.$
    
    Then we have the system of equations
    $$E(0)=E(1)+1$$
    $$E(1)=\frac{1}{2}(E(0)+E(2))+1$$
    $$E(2)=\frac{1}{2}(E(1)+E(3))+1$$
    $$\cdots$$
    $$E(n-1)=\frac{1}{2}(E(n-2)+E(n))+1$$
    $$E(n)=0.$$
    
    We claim that $E(k)=E(k+1)+(2k+1)$ for all suitable $k.$ We prove this by induction with the base case of $k=0.$ Then $E(k+1)=\frac{1}{2}(E(k)+E(k+2))+1=\frac{E(k+1)}{2}+\frac{2k+1}{2}+\frac{E(k+2)}{2}+1,$ implying $E(k+2)=E(k+1)+2(k+1)+1,$ as desired.
\end{sol}

\section{Freedom}

Similar to the Freedom section of \db{CQV-Perspectives}, here you are trying to find out \textit{what the states are}. We begin with a well-known example.

\begin{exam}[Airplane Probability Problem]
$100$ passengers board an airplane with exactly $100$ seats. Everyone has a ticket with an assigned seat number. However, the first passenger has lost their ticket and takes a random seat. Every subsequent passenger attempts to choose their own seat, but takes a random seat if theirs is taken. What is the probability the last passenger sits at his seat?
\end{exam}

\begin{sol}
Starting with the first passenger, we see that the uncertainty ends either when the $1$st or $100$th seat is taken. If the 1st seat is taken, everyone else files in to their own seats, and if the $100$th seat is taken, then the $100$th passenger cannot sit there. The answer is then $\frac{1}{2}.$
\end{sol}

However, freedom is best characterized by the following MAST Diagnostic problem.

\begin{exam}[MAST Diagnostic S1/C6]
Andy the unicorn is on a number line from $1$ to $2019.$ He starts on $1.$ Each step, he randomly and uniformly picks a number greater than the number he is currently on, and goes to it. He stops when he reaches $2019.$ What is the probability he is ever on $1984?$
\end{exam}

\begin{sol}
Say Andy is in state $A$ if he is between $1$ and $1983$ inclusive and in state $B$ if he is between $1984$ and $2019$ inclusive. The main claim is that \db{we only care about the move from $A$ to $B$}, since it is the only move you can land on $1984$ with. Since any of the numbers in $B$ are equally likely to be chosen, the answer is $\frac{1}{2019-1984+1}=\frac{1}{36}.$
\end{sol}

\pagebreak

\section{Problems}

\minpt{50}

\psetquote{Isn't everybody sick to death of all this stuff? Can't we all stand up and say enough!}{Death Note Musical}



    \begin{prob}[]{1}
 There are $n$ people, each with a test. The teacher, who is lazy, randomly passes the tests back. What is the expected amount of people who will receive their own test back?
\end{prob}

    \begin{prob}[MATHCOUNTS 2017]{1}
There are $100$ chickens in a circle. Each chicken randomly and simultaneously pecks the chicken to its left or right. How many chickens are expected to not be pecked?
\end{prob}

\begin{prob}[]{2}
Consider a number line with a drunkard at $0,$ and two cops at $-2019$ and $1000.$ Each second, the drunkard will randomly move to an adjacent integer with equal probability. The cops must move to an adjacent integer of their choice every second as well, and the movements of the cops and drunkard happen simultaneously. If the goal of the cops is to occupy the same number as the drunkard, what is the expected amount of seconds it will take the cops to occupy the same space as the drunkard? Assume optimal movement from the cops.
\end{prob}

\begin{prob}[SMT 2020]{2}
 A rook is on a chess board with $8$ rows and $8$ columns. The rows are numbered $1,\, 2,\, \ldots,\, 8$ and the columns are lettered $a,\, b,\, \ldots,\, h.$ The rook begins at $a1$ (the square in both row $1$ and column $a$). Every minute, the rook randomly moves to a different square either in the same row or the same column. The rook continues to move until it arrives a square in either row $8$ or column $h.$ After infinite time, what is the probability the rook ends at $a8?$
\end{prob}

\begin{prob}[AIME II 2019/2]{2}
 Lily pads $1,2,3,\ldots$ lie in a row on a pond. A frog makes a sequence of jumps starting on pad $1$. From any pad $k$ the frog jumps to either pad $k+1$ or pad $k+2$ chosen randomly with probability $\tfrac{1}{2}$ and independently of other jumps. The probability that the frog visits pad $7$ is $\tfrac{p}{q}$, where $p$ and $q$ are relatively prime positive integers. Find $p+q$.
\end{prob}

\begin{req}[]{3}
Bob is flipping a fair coin and wants to get $n$ heads in a row. In terms of $n,$ how many times should he expect to flip his coin?
\end{req}

\begin{prob}[AMC 12B 2019/19]{3}
 Raashan, Sylvia, and Ted play the following game. Each starts with $\$ 1$. A bell rings every $15$ seconds, at which time each of the players who currently have money simultaneously chooses one of the other two players independently and at random and gives $\$1$ to that player. What is the probability that after the bell has rung $2019$ times, each player will have $\$1$? (For example, Raashan and Ted may each decide to give $\$1$ to Sylvia, and Sylvia may decide to give her her dollar to Ted, at which point Raashan will have $\$0$, Sylvia will have $\$2$, and Ted will have $\$1$, and that is the end of the first round of play. In the second round Rashaan has no money to give, but Sylvia and Ted might choose each other to give their $\$1$ to, and the holdings will be the same at the end of the second round.)
\end{prob}

\begin{prob}[AMC 10B 2019/21]{3}
Debra flips a fair coin repeatedly, keeping track of how many heads and how many tails she has seen in total, until she gets either two heads in a row or two tails in a row, at which point she stops flipping. What is the probability that she gets two heads in a row but she sees a second tail before she sees a second head?
\end{prob}

\begin{prob}[AIME I 2019/5]{3}
A moving particle starts at the point $(4,4)$ and moves until it hits one of the coordinate axes for the first time. When the particle is at the point $(a,b)$, it moves at random to one of the points $(a-1,b)$, $(a,b-1)$, or $(a-1,b-1)$, each with probability $\tfrac{1}{3}$, independently of its previous moves. The probability that it will hit the coordinate axes at $(0,0)$ is $\tfrac{m}{3^n}$, where $m$ and $n$ are positive integers. Find $m + n$.
\end{prob}

%\begin{prob}[AIME 1985/14]{}
%In a tournament each player played exactly one game against each of the other players. In each game the winner was awarded 1 point, the loser got 0 points, and each of the two players earned $\frac{1}{2}$ point if the game was a tie. After the completion of the tournament, it was found that exactly half of the points earned by each player were earned in games against the ten players with the least number of points. (In particular, each of the ten lowest scoring players earned half of her/his points against the other nine of the ten). What was the total number of players in the tournament?
%\end{prob}

\begin{req}[ART 2019/4]{4}
 Consider a number line with integers $-65,-64\dots 62,63.$ Every second, a particle at the origin randomly moves to an adjacent integer. Find the expected amount of seconds for the particle to reach either $-65$ or $63.$
\end{req}

\begin{prob}[AMC 10B 2019/18]{4}
Henry decides one morning to do a workout, and he walks $\tfrac{3}{4}$ of the way from his home to his gym. The gym is $2$ kilometers away from Henry's home. At that point, he changes his mind and walks $\tfrac{3}{4}$ of the way from where he is back toward home. When he reaches that point, he changes his mind again and walks $\tfrac{3}{4}$ of the distance from there back toward the gym. If Henry keeps changing his mind when he has walked $\tfrac{3}{4}$ of the distance toward either the gym or home from the point where he last changed his mind, he will get very close to walking back and forth between a point $A$ kilometers from home and a point $B$ kilometers from home. What is $|A-B|$?
\end{prob}

\begin{prob}[TMC 2020 10B/18]{4}
 Edwin has two chess pieces that he places both on the center square of a $5\times 5$ chessboard. He sets a border one square wide on the edges of the chessboard, leaving a $3\times 3$ area in the middle. In one move, each piece moves as follows:
    \begin{itemize}
\Item The white piece moves one square either vertically or horizontally and then two squares in a perpendicular direction.
\Item The black piece moves one square either vertically or horizontally.
\end{itemize}
Each piece moves repeatedly until it first lands on a square in the border, at which point it stops moving. If both pieces move randomly but always abide by their rules, what is the probability that the white and black pieces will end up on the same square after they each stop moving?
\end{prob}   

\begin{prob}[AIME 1995/15]{4}
 Let $p$ be the probability that, in the process of repeatedly flipping a fair coin, one will encounter a run of $5$ heads before one encounters a run of $2$ tails. Given that $p$ can be written in the form $m/n$ where $m$ and $n$ are relatively prime positive integers, find $m+n$.
\end{prob}

\begin{prob}[AMC 10A 2020/23]{4}
Frieda the frog begins a sequence of hops on a $3 \times 3$ grid of squares, moving one square on each hop and choosing at random the direction of each hop-up, down, left, or right. She does not hop diagonally. When the direction of a hop would take Frieda off the grid, she "wraps around" and jumps to the opposite edge. For example if Frieda begins in the center square and makes two hops "up", the first hop would place her in the top row middle square, and the second hop would cause Frieda to jump to the opposite edge, landing in the bottom row middle square. Suppose Frieda starts from the center square, makes at most four hops at random, and stops hopping if she lands on a corner square. What is the probability that she reaches a corner square on one of the four hops?
\end{prob}

\begin{prob}[Mildorf]{4}
 A single atom of Uranium rests at the origin. Each second, the particle has a $\frac{1}{4}$ chance of moving one unit in the negative $x$ direction and a $\frac{1}{2}$ chance of moving in the positive $x$ direction. If the particle reaches $(-3,0),$ it ignites a fission that will consume the earth. If it reaches $(7,0),$ it is harmlessly diffused. The probability that, eventually, the particle is safely contained can be expressed as $\frac{m}{n}$ for some relatively prime positive integers $m$ and $n.$ Determine the remainder obtained when $m+n$ is divided by $1000.$
\end{prob}

\begin{prob}[AVHS 2017]{6}
 Drake the toy snake is moving in the coordinate plane and he starts at the origin. Every second, if he is at $(x,y),$ he either moves to $(x-1,y),$ $(x+1,y),$ $(x,y-1),$ or $(x,y+1).$ What is the expected amount of seconds it takes for his four most recent moves to draw out a unit square?
\end{prob}

\begin{prob}[]{6}
Arthur the arthropod sits at a vertex of a cube.  Every minute he teleports to one of the three adjacent vertices, each having equal probability of being selected.  After six minutes, what is the probability that he is back at the start?
\end{prob}

\begin{prob}[AIME I 2021/12]{6}
Let $A_1A_2A_3...A_{12}$ be a dodecagon (12-gon). Three frogs initially sit at $A_4,A_8,$ and $A_{12}$. At the end of each minute, simultaneously, each of the three frogs jumps to one of the two vertices adjacent to its current position, chosen randomly and independently with both choices being equally likely. All three frogs stop jumping as soon as two frogs arrive at the same vertex at the same time. The expected number of minutes until the frogs stop jumping is $\frac mn$, where $m$ and $n$ are relatively prime positive integers. Find $m+n$.
\end{prob}

\begin{req}[HPMS MATHCOUNTS Tryouts]{6}
$64$ balls, labeled with the integers from $1$ through $64,$ are placed in a bag. Balls are removed form the bag (without replacement) one by one until a ball with an odd number is removed. What is the probability that among the balls removed from the bag is the ball labeled $42?$ Express your answer as a common fraction.
\end{req}

\begin{prob}[AIME I 2016/13]{6}
Freddy the frog is jumping around the coordinate plane searching for a river, which lies on the horizontal line $y = 24$. A fence is located at the horizontal line $y = 0$. On each jump Freddy randomly chooses a direction parallel to one of the coordinate axes and moves one unit in that direction. When he is at a point where $y=0$, with equal likelihoods he chooses one of three directions where he either jumps parallel to the fence or jumps away from the fence, but he never chooses the direction that would have him cross over the fence to where $y < 0$. Freddy starts his search at the point $(0, 21)$ and will stop once he reaches a point on the river. Find the expected number of jumps it will take Freddy to reach the river.
\end{prob}

%\begin{prob}[MAST Diagnostic 2020]{9}
%Andy the unicorn is on a number line from $1$ to $2019.$ He starts on $1.$ Each step, he randomly and uniformly picks a number greater than the number he is currently on, and goes to it. He stops when he reaches $2019.$ What is the probability he is ever on $1984?$
%\end{prob}

\begin{prob}[Variation on MAST S1/C6]{9}
Andy the unicorn is on a number line from $1$ to $2019.$ He starts on $1.$ Each step, he randomly picks a number greater than the number he is currently on, and the probabilities are distributed such that the probability of him going to $n+1$ is half the probability of him going to $n,$ where $n,n+1$ are both integers greater than his current position and less than $2019.$\footnote{This means that $n\leq 2018.$} He stops when he reaches $2019.$ What is the probability he is ever on $1984?$
\end{prob}

\begin{prob}[CMIMC Team 2019/7]{13}
 Suppose you start at $0,$ a friend starts at $6,$ and another friend starts at $8$ on the number line. Every second, the leftmost person moves left with probability $\frac{1}{4},$ the middle person with probability $\frac{1}{3},$ and the rightmost person with probability $\frac{1}{2}.$ If a person does not move left, they move right, and if two people are on the same spot, they are randomly assigned which one of the positions they are. Determine the expected time until you all meet in one point.
\end{prob}

\begin{req}[NARML 2020/8]{13}
The mad scientist Kyouma is traveling on a number line from $1$ to $2020,$ subject to the following rules:
\begin{itemize}
\Item He starts at $1.$
\Item Each move, he randomly and uniformly picks a number greater than his current number to go to.
\Item If he reaches $2020$, he is instantly teleported back to $1.$
\Item There is a time machine on $199.$
\Item A foreign government is waiting to ambush him on $1729.$
\end{itemize}
What is the probability that he gets to the time machine before being ambushed?
\end{req}
\end{document}
