\documentclass{article}

\usepackage[mast]{lucky}

\title{Solutions to States}
\author{Abrar Fiaz}
\date{CQU}

\begin{document}

\maketitle

\toc


\pagebreak\section{Classical} 
There are $n$ people, each with a test. The teacher, who is lazy, randomly passes the tests back. What is the expected amount of people who will receive their own test back?
\subsection{Solution}
This is a classic, notice that a person can get his test back with probability $\frac1n$. Thus, the expected value will be $n\cdot\frac1n = 1.$ 
	


\pagebreak\section{MATHCOUNTS 2017} There are $100$ chickens in a circle. Each chicken randomly and simultaneously pecks the chicken to its left or right. How many chickens are expected to not be pecked?
\subsection{Solution}
The probability of chicken getting pecked is $\frac14,$ then it is just $100\cdot\frac14=25.$



\pagebreak\section{AIME 1985/14} In a tournament each player played exactly one game against each of the other players. In each game the winner was awarded 1 point, the loser got 0 points, and each of the two players earned $\frac{1}{2}$ point if the game was a tie. After the completion of the tournament, it was found that exactly half of the points earned by each player were earned in games against the ten players with the least number of points. (In particular, each of the ten lowest scoring players earned half of her/his points against the other nine of the ten). What was the total number of players in the tournament?
\subsection{Solution}

Let $n$ be the number of op people, then $\binom n2+ 45 = 10n$ which gives $n=6,15$ and $n$ can't be $6$ for 
obvious reason, giving answer $10+15=25.$





\pagebreak\section{Unsourced}
Consider a number line with a drunkard at $0,$ and two cops at $-2019$ and $1000.$ Each second, the drunkard will randomly move to an adjacent integer with equal probability. The cops must move to an adjacent integer of their choice every second as well, and the movements of the cops and drunkard happen simultaneously. If the goal of the cops is to occupy the same number as the drunkard, what is the expected amount of seconds it will take the cops to occupy the same space as the drunkard? Assume optimal movement from the cops.
\subsection{Solution}
$-2019$ doesn't work for parity reason, For the cop at $1000,$ he will always move to the left. Ideally we would want the drunkard to go $500$ numbers right, then the expected value is just $1000$ from linearity of expectation.      


\pagebreak\section{Unsourced}
Consider a number line with integers $0,1,2\dots n.$ Every second, a particle initially at the origin randomly moves to an adjacent integer. (If it is at $0,$ it goes to $1$ all the time.) In terms of $n,$ find the expected amount of time for the particle to reach $n.$
\subsection{Solution}
This is just standard states, basically you come up with $E_{n-k}=2kn-k^2$ then we get $E_0 = n^2.$



\pagebreak\section{ART 2019/4} Consider a number line with integers $-65,-64\dots 62,63.$ Every second, a particle at the origin randomly moves to an adjacent integer. Find the expected amount of seconds for the particle to reach either $-65$ or $63.$
\subsection{Solution}    





\pagebreak\section{AVHS 2017} Drake the toy snake is moving in the coordinate plane and he starts at the origin. Every second, if he is at $(x,y),$ he either moves to $(x-1,y),$ $(x+1,y),$ $(x,y-1),$ or $(x,y+1).$ What is the expected amount of seconds it takes for his four most recent moves to draw out a unit square?
\subsection{Solution}




\pagebreak\section{Unsourced}
Bob is flipping a fair coin and wants to get $n$ heads in a row. In terms of $n,$ how many times should he expect to flip his coin?
\subsection{Solution}





\pagebreak\section{AIME II 2019/2} Lily pads $1,2,3,\ldots$ lie in a row on a pond. A frog makes a sequence of jumps starting on pad $1$. From any pad $k$ the frog jumps to either pad $k+1$ or pad $k+2$ chosen randomly with probability $\tfrac{1}{2}$ and independently of other jumps. The probability that the frog visits pad $7$ is $\tfrac{p}{q}$, where $p$ and $q$ are relatively prime positive integers. Find $p+q$.
\subsection{Solution}




\pagebreak\section{TMC 10B/18} Edwin has two chess pieces that he places both on the center square of a $5\times 5$ chessboard. He sets a border one square wide on the edges of the chessboard, leaving a $3\times 3$ area in the middle. In one move, each piece moves as follows:
    \begin{itemize}
\Item The white piece moves one square either vertically or horizontally and then two squares in a perpendicular direction.
\Item The black piece moves one square either vertically or horizontally.
\end{itemize}
Each piece moves repeatedly until it first lands on a square in the border, at which point it stops moving. If both pieces move randomly but always abide by their rules, what is the probability that the white and black pieces will end up on the same square after they each stop moving?
\subsection{Solution}    





\pagebreak\section{HPMS MATHCOUNTS Tryouts} $64$ balls, labeled with the integers from $1$ through $64,$ are placed in a bag. Balls are removed form the bag (without replacement) one by one until a ball with an odd number is removed. What is the probability that among the balls removed from the bag is the ball labeled $42?$ Express your answer as a common fraction.
\subsection{Solution}




\pagebreak\section{AMC 12B 2019/19} Raashan, Sylvia, and Ted play the following game. Each starts with $\$ 1$. A bell rings every $15$ seconds, at which time each of the players who currently have money simultaneously chooses one of the other two players independently and at random and gives $\$1$ to that player. What is the probability that after the bell has rung $2019$ times, each player will have $\$1$? (For example, Raashan and Ted may each decide to give $\$1$ to Sylvia, and Sylvia may decide to give her her dollar to Ted, at which point Raashan will have $\$0$, Sylvia will have $\$2$, and Ted will have $\$1$, and that is the end of the first round of play. In the second round Rashaan has no money to give, but Sylvia and Ted might choose each other to give their $\$1$ to, and the holdings will be the same at the end of the second round.)
\subsection{Solution}    




\pagebreak\section{MAST Diagnostic 2020} Andy the unicorn is on a number line from $1$ to $2019.$ He starts on $1.$ Each step, he randomly and uniformly picks a number greater than the number he is currently on, and goes to it. He stops when he reaches $2019.$ What is the probability he is ever on $1984?$
\subsection{Solution}





\pagebreak\section{AMC 10B 2019/18} Henry decides one morning to do a workout, and he walks $\tfrac{3}{4}$ of the way from his home to his gym. The gym is $2$ kilometers away from Henry's home. At that point, he changes his mind and walks $\tfrac{3}{4}$ of the way from where he is back toward home. When he reaches that point, he changes his mind again and walks $\tfrac{3}{4}$ of the distance from there back toward the gym. If Henry keeps changing his mind when he has walked $\tfrac{3}{4}$ of the distance toward either the gym or home from the point where he last changed his mind, he will get very close to walking back and forth between a point $A$ kilometers from home and a point $B$ kilometers from home. What is $|A-B|$?
\subsection{Solution}    




\pagebreak\section{AIME I 2019/5} A moving particle starts at the point $(4,4)$ and moves until it hits one of the coordinate axes for the first time. When the particle is at the point $(a,b)$, it moves at random to one of the points $(a-1,b)$, $(a,b-1)$, or $(a-1,b-1)$, each with probability $\tfrac{1}{3}$, independently of its previous moves. The probability that it will hit the coordinate axes at $(0,0)$ is $\tfrac{m}{3^n}$, where $m$ and $n$ are positive integers. Find $m + n$.
\subsection{Solution}





\pagebreak\section{Mildorf} A single atom of Uranium rests at the origin. Each second, the particle has a $\frac{1}{4}$ chance of moving one unit in the negative $x$ direction and a $\frac{1}{2}$ chance of moving in the positive $x$ direction. If the particle reaches $(-3,0),$ it ignites a fission that will consume the earth. If it reaches $(7,0),$ it is harmlessly diffused. The probability that, eventually, the particle is safely contained can be expressed as $\frac{m}{n}$ for some relatively prime positive integers $m$ and $n.$ Determine the remainder obtained when $m+n$ is divided by $1000.$
\subsection{Solution}





\pagebreak\section{AIME I 2016/13} Freddy the frog is jumping around the coordinate plane searching for a river, which lies on the horizontal line $y = 24$. A fence is located at the horizontal line $y = 0$. On each jump Freddy randomly chooses a direction parallel to one of the coordinate axes and moves one unit in that direction. When he is at a point where $y=0$, with equal likelihoods he chooses one of three directions where he either jumps parallel to the fence or jumps away from the fence, but he never chooses the direction that would have him cross over the fence to where $y < 0$. Freddy starts his search at the point $(0, 21)$ and will stop once he reaches a point on the river. Find the expected number of jumps it will take Freddy to reach the river.
\subsection{Solution}





\pagebreak\section{CMIMC Team 2019/7} Suppose you start at $0,$ a friend starts at $6,$ and another friend starts at $8$ on the number line. Every second, the leftmost person moves left with probability $\frac{1}{4},$ the middle person with probability $\frac{1}{3},$ and the rightmost person with probability $\frac{1}{2}.$ If a person does not move left, they move right, and if two people are on the same spot, they are randomly assigned which one of the positions they are. Determine the expected time until you all meet in one point.
\subsection{Solution}    




\pagebreak\section{AIME 1995/15} Let $p$ be the probability that, in the process of repeatedly flipping a fair coin, one will encounter a run of $5$ heads before one encounters a run of $2$ tails. Given that $p$ can be written in the form $m/n$ where $m$ and $n$ are relatively prime positive integers, find $m+n$.
\subsection{Solution}

\pagebreak\section{AMC 10A 2020/23}

Frieda the frog begins a sequence of hops on a $3 \times 3$ grid of squares, moving one square on each hop and choosing at random the direction of each hop-up, down, left, or right. She does not hop diagonally. When the direction of a hop would take Frieda off the grid, she "wraps around" and jumps to the opposite edge. For example if Frieda begins in the center square and makes two hops "up", the first hop would place her in the top row middle square, and the second hop would cause Frieda to jump to the opposite edge, landing in the bottom row middle square. Suppose Frieda starts from the center square, makes at most four hops at random, and stops hopping if she lands on a corner square. What is the probability that she reaches a corner square on one of the four hops?

\subsection{Solution}

\pagebreak\section{AMC 10B 2019/21} Debra flips a fair coin repeatedly, keeping track of how many heads and how many tails she has seen in total, until she gets either two heads in a row or two tails in a row, at which point she stops flipping. What is the probability that she gets two heads in a row but she sees a second tail before she sees a second head?
\subsection{Solution}




\pagebreak\section{SMT 2020} A rook is on a chess board with $8$ rows and $8$ columns. The rows are numbered $1,\, 2,\, \ldots,\, 8$ and the columns are lettered $a,\, b,\, \ldots,\, h.$ The rook begins at $a1$ (the square in both row $1$ and column $a$). Every minute, the rook randomly moves to a different square either in the same row or the same column. The rook continues to move until it arrives a square in either row $8$ or column $h.$ After infinite time, what is the probability the rook ends at $a8?$
\subsection{Solution}




\pagebreak\section{Unsourced}{Arthur the arthropod sits at a vertex of a cube.  Every minute he teleports to one of the three adjacent vertices, each having equal probability of being selected.  After six minutes, what is the probability that he is back at the start?}
\subsection{Solution}




\end{document}
