\documentclass[mast]{lucky}

\title{Introduction to Divisibility and Modular Arithmetic}
\author{Dennis Chen, Kelin Zhu}
\date{NPU}

\begin{document}
\maketitle
\section{Divisibility}
The concept of divisibility is a cornerstone of Number Theory across the AMC's and beyond, and therefore it must be reserved as our first topic.
\begin{defi}[Divisibility]
We say that an integer $m$ is \textbf{divisible by} another integer $n$ and equivalently $n$ \textbf{divides} $m$ (neither are necessarily positive, though we will mostly work with positive integers in this unit), notated as $n|m$, if there exists another integer $k$ with $m=kn$.
\end{defi}
\begin{remark}
If we instead define divisibility by saying that $\frac{m}{n}=k$ is an integer, our picture falls apart when 0 is introduced. Indeed, we can get that $0$ is divisible by $0$, but $\frac{0}{0}$ is indeterminate.
\end{remark}

We can derive some useful results immediately:
\begin{fact}[Divisibility Results]
\hfill
\begin{enumerate}
    \item If $a\mid b$ and both $a,b$ are positive then $a\le b$; if $a\mid b$, $b\mid a$ and both $a,b$ are positive then $a=b$.
    \item If $a\mid b$ and $b\mid c$ then $a\mid c.$
    \item If $a\mid b$ then $a\mid bc$ for all integers $c.$
    \item If $a\mid b$ and $a\mid c,$ then $a\mid b+c$ and $a\mid b-c$.
\end{enumerate}
\end{fact}

WARNING: There are many false results in divisibility that might be cited or used mistakenly. The most common ones will be listed below.
\begin{enumerate}
\item If $a\mid c$ and $b\mid c$, it is NOT NECESSARILY true that $ab\mid c$; take $a=4,b=8,c=16$ for a counterexample. However, this claim is always true if $a,b$ are relatively prime. (can you see why?)
\item If $a\mid bc$, it is NOT NECESSARILY true that $a\mid b$ or $a\mid c$; again, $a=16, b=4, c=8$ is a counterexample.
\item Conversely, if $a$ divides neither $b$ nor $c$, it is NOT NECESSARILY true that $a$ does not divide $bc$.
\end{enumerate}

\hfill

The concept of GCD and LCM also appear frequently; they will be covered more thoroughly in the unit NQV-Prime.

\begin{defi}[LCM/GCD]
The \textbf{Least Common Multiple} (often abbreviated as \textbf{LCM}) of two integers $a,b$ is the smallest positive integer that is a multiple of both $a$ and $b$.

The \textbf{Greatest Common Divisor} (often abbreviated as \textbf{GCD}) of two integers $a,b$ is the greatest positive integer that divides both $a$ and $b$. In particular, the GCD of $0$ and $n$ for any integer $n$ is equal to $n$, and the GCD of $0$ and $0$ is undefined.

Two integers $a,b$ are \textbf{relatively prime} if and only if their GCD is equal to $1$. In particular, $1$ and $-1$ are relatively prime to all integers.
\end{defi}

These concepts can be extended to three or more integers, but at this stage, we only really work with two.

To top off the section, here are a few relatively well-known divisibility rules. For a challenge, try proving all of them on your own after reading throrugh the next section!
\begin{fact}[Divisibility rules]
\hfill
\begin{itemize}
    \item 2: If the last digit of $n$ is even, then $n$ is even.
    \item 4: If the last 2 digits of $n$ is a multiple of 4, then $n$ is a multiple of 4.
    \item 8: If the last 3 digits of $n$ is a multiple of 8, then $n$ is a multiple of 8. (Can you see a pattern?)
    \item 3/9: If the sum of digits of $n$ is a multiple of 3/9, then $n$ is a multiple of 3/9. (This DOES NOT generalize to 27 or greater powers of 3!)
    \item 5: If the last digit of $n$ is a multiple of 5, then $n$ is a multiple of 5.
    \item 25: If the last 2 digits of $n$ is a multiple of 25, then $n$ is a multiple of 25.
    \item 125: If the last 3 digits of $n$ is a multiple of 125, then $n$ is a multiple of 125. (The pattern for powers of 2 also apply here.)
    \item 11: Let $a$ be the sum of the 1st, 3rd, 5th... digits from the right of $n$, and let $b$ be the sum of the 2nd, 4th, 6th... digits from the right of $n$. If $a-b$ is a multiple of 11, then $n$ is a multiple of 11.
\end{itemize}
For pairwise relatively prime integers, we can construct the divisibility rule of their product by simply combining their divisibility rules. For example, the divisibility rule of $60$ is simply being divisible by $3,4,5$, since any two of $3,4,5$ have GCD 1. 
\end{fact}

All of the above are bidirectional; that is, all multiples of $x$ will satisfy the divisibility rules of $x$, and all numbers that satisfy the divisibility rules of $x$ will be multiples of $x$.
\section{Modular Arithmetic}
The following section describes operations in Modular Arithmetic, intuitively motivated by operations over the integers, rationals, and even real and complex numbers\footnote{Yes, you can find $i \pmod{p}$.}. You will find you will be able to do almost everything$\pmod{n}$ that you would be able to do normally. Make sure you understand the rigorous reasons why these things are true, but you should simultaneously feel free to do whatever you want given a few restrictions.

Note that the modulus does have to be positive and greater than 1, unlike the previous section.

\begin{defi}[Modular Congruence]
We say $a\equiv b\pmod {n}$ if and only if $n\mid a-b.$
\end{defi}

The intuitive way to think about this is that $a$ and $b$ have the same remainder when divided by $n.$ (Remember that negative numbers also have a remainder when divided.)

As a corollary, we can derive the following: if $a\equiv b\pmod {n}$, then $a\equiv b\pmod {d}$ for any divisor $d$ of $n$. (The converse is obviously false.)

\begin{defi}[Modular Residue]
We say the \textit{residue} of an integer $a\pmod{n}$ is the integer $b$ that satisfies
\begin{itemize}
    \item $0\leq b<n$

    \item $a\equiv b\pmod{n}.$
\end{itemize}
\end{defi}

It can be helpful to think of $b$ as the remainder of $a$ when divided by $n.$

\subsection{Modular Operations}
You can add, subtract, multiply, and exponentiate modulos. You can also divide, but care must be taken.

\begin{fact}[Adding]
If $a \equiv x\pmod{n}$ and $b\equiv y\pmod{n},$ $a+b\equiv x+y\pmod{n}.$
\end{fact}

\begin{pro}
Since $n\mid x-a$ and $n\mid y-b,$ clearly $n\mid(x+y)-(a+b).$
\end{pro}

Subtracting is identical, so we do not discuss it further.

\begin{fact}[Multiplying]
If $a \equiv x\pmod{n}$ and $b\equiv y\pmod{n},$ $ab\equiv xy\pmod{n}.$
\end{fact}

\begin{pro}
Say $a=a_pn+q$ and $x=x_pn+q$ where $q$ is the residue of $a$ and $x,$ and $b=b_pn+r$ and $y=y_pn+r$ where $r$ is the residue of $b$ and $y.$ Then
\begin{align*}
xy-ab&=(x_pn+q)(y_pn+r)-(a_pn+q)(b_pn+r)\\
&=n^2(x_py_p-a_pb_p)+n(x_pr+y_pq-a_pr-b_pq)+qr-qr\\
&=n^2(x_py_p-a_pb_p)+n(x_pr+y_pq-a_pr-b_pq),
\end{align*}
which is clearly divisible by $n.$
\end{pro}

\begin{fact}[Exponentiating]
For integer $a,b$ and positive integers $n,k,$ if $a \equiv b\pmod{n}$, then $a^k \equiv b^k \pmod{n}$.
\end{fact}

\begin{pro}
Note that $n\mid a-b\mid a^k-b^k.$
\end{pro}

As an exercise for the operations we've defined so far, pick your favorite ordered triple of positive integers $(a,b,n),$ and compute the remainder of $a+b,$ $a-b,$ $ab,$ and $a^b$\footnote{For large enough $b,$ you'll want to know Fermat's Little Theorem!} when divided by $n.$

\begin{fact}[Dividing]
Let $a,b,c$ be positive integers such that $c\mid a$ and $c\mid b.$ If $a\equiv b\pmod{n}$ \textbf{and} $\gcd(c,n)=1,$ then $\frac{a}{c}\equiv\frac{b}{c}\pmod{n}.$
\end{fact}

Be careful to remember that we \textbf{must have} $\gcd(c,n)=1$!

\begin{fact}[Strengthened 3/9 divisibility]
Let the sum of digits of a positive integer $n$ be equal to $m$. Then, $n\equiv m\pmod{9}$.
\end{fact}

\begin{pro}
$10^0\cdot d_0+10^1\cdot d_1+10^2\cdot d^2+\ldots\equiv 1^0\cdot d_0+1^1\cdot d_1+1^2\cdot d_2\ldots\pmod{9}$
\end{pro}

\begin{fact}[Strengthened powers of 2/5 divisibility]
For nonnegative integers $n,a,b$, the remainder when $n$ is divided by $2^a5^b$ is equal to the remainder when the last $\max(a,b)$ digits of $n$ is divided by $2^a5^b$.
\end{fact}

\begin{pro}
$10^0\cdot d_0+10^1\cdot d_1\ldots +10^{\max(a,b)}d_{\max(a,b)}+\ldots= 2^05^0\cdot d_0+2^15^1\cdot d_1\ldots +2^{\max(a,b)}5^{\max(a,b)}\cdot d_{\max(a,b)}+\ldots \equiv 10^{0}\cdot d_{0}+10^1\cdot d_1\ldots +10^{\max(a,b)-1}\cdot d_{\max(a,b)-1}+0\cdot d_{\max(a,b)}+0\ldots \pmod{2^a5^b}$
\end{pro}

The above rules allows us to kill an AMC last five in mere seconds:

\begin{exam}[AMC 10B 2017/23]
The positive integer $N=1234\cdots44$ is the concatenation of the numbers $1,2,3,\ldots,44.$ Find the remainder when $N$ is divided by $45.$

$\textbf{(A)}\ 1 \qquad\textbf{(B)}\ 4 \qquad\textbf{(C)}\ 9 \qquad\textbf{(D)}\ 18 \qquad\textbf{(E)}\ 44$
\end{exam}

\begin{sol}
$N$ is equivalent to 4 mod 5 from its last digit, which immediately rules out choices A,D. It seems somewhat tedious to count the number of occurrences of each digit, which motivates the observation that $1+0+1+1+\ldots +4+3+4+4\equiv 10+11+\ldots +43+44=\frac{44\cdot 45}{2}\equiv 0\pmod{9}$, which eliminates choices D,E. Our final answer is $\ansbold{(C) 9}$.
\end{sol}

\subsection{Modular Inverses}
In normal arithmetic, we define $a\cdot a^{-1}=1.$ We can do something similar in modular arithmetic.

\begin{defi}[Modular Inverse]
We define $a^{-1}$ to be the number mod $n$ such that $a\cdot a^{-1}\equiv 1 \pmod{n}.$ We say that $a^{-1}$ is the inverse of $a \pmod{n}.$

\textbf{The modular inverse is defined if and only if $\gcd(a,n)=1.$} \end{defi}


We can treat inverses as fractions - for instance, $\frac{1}{2}+\frac{1}{3}+\frac{1}{6}\equiv 1\pmod{p}$ for $p\neq 2,3.$ The proof is non-trivial and inverses should be treated with care, so we will prove that all of these operations are valid.

You should rewrite all of these operations into fractions to understand what they're really saying. The proofs follow directly from the associative, distributive, and commutative properties.

\begin{fact}[Adding Inverses]
For integers $a,b$ relatively prime to $n,$ $a^{-1}+b^{-1}\equiv (a+b)(ab)^{-1}\pmod{n}.$
\end{fact}

\begin{pro}
Note that $(a+b)(ab)^{-1}\equiv aa^{-1}b^{-1}+ba^{-1}b^{-1}\equiv b^{-1}+a^{-1}\pmod{n}.$
\end{pro}

\begin{fact}[Multiplying Inverses]
For integers $a,b$ relatively prime to $n,$ $a^{-1}b^{-1}\equiv(ab)^{-1}\pmod{n}.$
\end{fact}

\begin{pro}
Note that $(ab)^{-1}ab\equiv 1\pmod{n}$ and $a^{-1}b^{-1}(ab)\equiv aa^{-1}bb^{-1}\equiv 1\pmod{n}.$
\end{pro}

Here is an example that uses the fact that modular inverses exist.

\begin{exam}
How many ordered quadruplets of integers $(a,b,c,d)$ with $1\leq a,b,c,d\leq 4$ exist such that $5\mid ab-cd$?
\end{exam}

\begin{sol}
Note that this implies $ab\equiv cd\pmod{5},$ or $\frac{ab}{c}\equiv d\pmod{5}.$ Notice that a choice of $(a,b,c)$ will uniquely determine $d,$ so the answer is just the number of ways to choose $(a,b,c),$ or $4^3=64$ ways.
\end{sol}

Make sure you understand \textbf{why} $d$ is uniquely determined!

\begin{fact}[General negative exponents]
We can also define any negative exponents mod $n$; $m^{-a}$ is the inverse of $m^a$, or $m^{-1}\cdot m^a$; both definitions give us the same residue.
\end{fact}

One last example, that uses purely standard modular arithmetic techniques; it epitomizes the ideas of this section. This is also an exercise in reading the problem carefully, and many students did overzealous approaches, for example bashing through all the cases. Unfourtunately, the answer was $E$, so this took a lot of time.\footnote{I (Ethan) seem to recall simply getting it wrong. Oops.}

\begin{exam}[AMC 10B 2017/25]
Last year Isabella took 7 math tests and received 7 different scores, each an integer between 91 and 100, inclusive. After each test she noticed that the average of her test scores was an integer. Her score on the seventh test was 95. What was her score on the sixth test?

$\textbf{(A)} \text{ 92} \qquad \textbf{(B)} \text{ 94} \qquad \textbf{(C)} \text{ 96} \qquad \textbf{(D)} \text{ 98} \qquad \textbf{(E)} \text{ 100}$
\end{exam}

\begin{sol}
Let $A$ be the average of the first 6 tests. We know $6A + 95$ is a multiple of 7, as it is the sum of the first seven tests, or \[6A + 95 \equiv 0 \pmod{7}\text{.}\] This means we have $6A  \equiv 3 \pmod 7$, which means \[A \equiv \frac12 \equiv \frac82 = 4 \pmod{7}\text{.}\] However, $A$ must be one of ${91,92,\dots, 100}$. In fact we find $A = 95$. If the sixth score is $S$ and the average of the first 5 tests is $B$, then since $570 = S + 5B$, $S$ is multiple of $5$ and must be $\ansbold{(E) 100}$.
\end{sol}

\section{Wilson's Theorem}

Factorials rarely appear in number theory (at least for the AMCs and the AIME). But Wilson's Theorem is still one of the standard tools you need to have at your disposal.

\begin{theo}[Wilson's Theorem]
For prime $p,$ $$(p-1)!\equiv -1\pmod{p}.$$
\end{theo}

\begin{pro}[1]
Notice that the numbers $2,3,4\dots p-2$ all have modular inverses. In addition, modular inverses come in pairs. Since $p$ is odd (the case where $p=2$ is very easy to deal with), then the modular inverses all multiply to $1.$ This leaves us with $(p-1)!\equiv 1\cdot (p-1)\equiv -1\pmod{p},$ as desired.
\end{pro}

We do not include $1,p-1$ in the pairing because for prime $p,$ $1$ and $p-1$ are the only numbers whose modular inverses are themselves.

Here is how Wilson's theorem simplify a problem you might have solved in the Season 3 application:

\begin{exam}[NARML/4, MAST S3 Diagnostics/N1]
Compute the smallest positive integer $n$ such that $9(n+3)$ divides $4n!+n+5$.
\end{exam}

\begin{sol}
Brute force works here, but there is an easier route.

First, we can check that none of $1\le n\le 5$ work, and therefore $4n!+n+5\equiv n+5\pmod{9}$ as $n!$ is necessarily a multiple of $3\cdot 6=9\cdot 2$. From here, we can see that $n$ must be equivalent to 4 mod 9.

Next, suppose that $n+3$ is composite; as $n\ge 6$, we have $2n>n+3$ and therefore $n!$ will be a multiple of $n+3$; we would need $n+5$ to be a multiple of $n+3$ which is impossible. Therefore, $n+3$ must be a prime.

The smallest possible $n$ that satisfies the above conditions is $40$. Indeed, by Wilson's theorem, $40!\equiv \frac{-1}{41\cdot 42} \equiv \frac{-1}{2}\pmod{43}$ and $4\cdot \frac{-1}{2}+40+5=40+5-2=43$.
\end{sol}

\begin{remark}
Even though Wilson's is only used towards the end to verify our solution, it invisibly motivates the entire solution. The idea of checking if $n+3$ is prime should naturally arise after seeing factorials in the problem.
\end{remark}
\pagebreak

\section{Problems}
\minpt{34}

\psetquote{Take what fortune grants you, use it while you've got it!}{Death Note Musical}

\begin{prob}[]{1}
Find the inverse of $2\pmod {p}$ for odd prime $p$ in terms of $p.$
\end{prob}
    
\begin{prob}[]{1}
Find the remainder of $98!$ when divided by $101.$
\end{prob}

\begin{prob}[]{2}
What is the residue of $\frac{1}{1\cdot 2}\cdot \frac{1}{2\cdot 3}\cdot \dots \cdot \frac{1}{11\cdot 12}\pmod {13}?$
\end{prob}

\begin{prob}[AMC 10B 2021/3, buffed]{2}
An after-school program contains a nonzero amount of juniors and seniors, with no other grades present. Among the 28 students in the program, 25\% of the juniors and 10\% of the seniors are on the debate team. how many juniors are in the program?
\end{prob}

\begin{prob}[1001 Problems in Number Theory]{3}
For which positive integers $n$ is it true that $1+2+\cdots+n\mid 1\cdot 2\cdot \cdots \cdot n$?
\end{prob}

\begin{prob}[AMC 10A 2017/13]{3}
Define a sequence recursively by $F_0 = 0$, $F_1 = 1$, and $F_n = $ the remainder when $F_{n-1} + F_{n-2}$ is divided by $3$, for all $n \ge 2$. Thus the sequence starts $0,1,1,2,0,2 \ldots$. What is $F_{2017} + F_{2018} + F_{2019} + F_{2020} + F_{2021} + F_{2022} + F_{2023} + F_{2024}$?
\end{prob}

\begin{req}[AMC 10B 2019/14]{4}
The base-ten representation for $19!$ is $121,6T5,100,40M,832,H00$, where $T$, $M$, and $H$ denote digits that are not given. What is $T+M+H$?
\end{req}

\begin{prob}[AIME I 2020/4, modified]{4}
Let $S$ be the set of positive integers $N$ with the property that the last three digits of $N$ are $343$, and when the last three digits are removed, the result is a divisor of $N$. For example, $7343$ is in $S$ because $7$ is a divisor of $7343$. Find the sum of all the digits of all the numbers in $S$. For example, the number $7343$ contributes $7+3+4+3=17$ to this total.
\end{prob}

\begin{prob}[AIME II 2017/8]{6}
Find the number of positive integers $n$ less than $2017$ such that
\[ 1+n+\frac{n^2}{2!}+\frac{n^3}{3!}+\frac{n^4}{4!}+\frac{n^5}{5!}+\frac{n^6}{6!} \]is an integer.
\end{prob}

\begin{prob}[AOIME 2020/10]{6}
Find the sum of all positive integers $n$ such that when $1^3+2^3+3^3+\cdots+n^3$ is divided by $n+5$, the remainder is $17.$
\end{prob}

\begin{req}[AIME 1989/9]{9}
One of Euler's conjectures was disproved in the 1960s by three American mathematicians when they showed there was a positive integer such that $133^5+110^5+84^5+27^5=n^{5}$. Find the value of $n$.
\end{req}
    
\begin{prob}[USAMO 1979/1]{9}
Determine all non-negative integral solutions $(n_1, n_2, \dots , n_k)$, if any, apart from permutations, of the Diophantine equation
$$n_1^4 + n_2^4 + \cdots + n_{14}^{4} = 1599.$$
\end{prob}

\begin{prob}[IMO 2005/4]{13}
Determine all positive integers relatively prime to all the terms of the infinite sequence
\[a_n=2^n+3^n+6^n -1,\ n\geq 1.\]
\end{prob}

\end{document}