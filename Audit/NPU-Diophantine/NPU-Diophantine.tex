\documentclass{article}
\usepackage{dennis}

\usepackage{float}

\title{Diophantine Equations}
\author{Dylan Yu}
\date{NPU}

\begin{document}
\maketitle

\section{Common Types and Techniques for Diophantine Equations}
\begin{enumerate}
    \item There are known formulas for some types of Diophantine equations. For example,
    \begin{itemize}
        \item $ax+by=c$: linear equations with two unknowns
        \item $a^2+b^2=c^2$: quadratic equations with three unknowns
        \item $xy=zt$: quadratic equations with four unknowns
    \end{itemize}
    \item Uniqueness of Prime Factorization
    \item Completing the Square
    \item Completing the Rectangle/SFFT
    \item Limit the size of solutions or key quantities
    \item The Squeeze Principle
    \item Parity Analysis
    \item Choosing Special Modulus
    \item Change of Variables/Substitution
    \item Modular contradiction (e.g. a number cannot be $1\pmod{4}$ and $0\pmod{2}$.)
    \item Factorizations
    \item LTE
    % \item Zsigmondy's theorem (note: you cannot quote this on Olympiads)
    \item Pell equations
    \item Recurrences
    % \item Vieta jumping
    \item Infinite descent
    % \item Quadratic reciprocity
    % \item Cyclotomic polynomials
    \item Algebraic substitutions
    \item Inequalities (similar to squeezing and bounding)
    \item Legendre's formula for $p$-adic valuation of factorials
    \item Geometric interpretation
    \item Induction
    \item Extensions of $\mathbb{Z}$
\end{enumerate}
\section{Introduction}
\subsection{Definitions}
Here we introduce some important notation and ideas that we will use throughout the handout.
\begin{defi}[Diophantine Equation]
A \db{diophantine equation} is an equation that can be solved over the integers. 
\end{defi}
For example, $a+b=32$, where $a,b$ are integers, is a diophantine equation.  A \textit{linear example} would be $ax+by=c$, where $a,b,c,x,y$ are integers.
\begin{defi}[$\mathbb{Z}$]
If $a\in$ \db{$\mathbb{Z}$}, then $a$ is an integer.
\end{defi}
Furthermore, \db{$\mathbb{Z}^-$} is the set of negative integers, \db{$\mathbb{Z}^+$} is the set of positive integers, \db{$\mathbb{Z}^{0+}$} is the set of nonnegative integers, and \db{$\mathbb{Z}^{0-}$} is the set of nonpositive integers.
\subsection{Modular Arithmetic}
When we say ``$a\equiv b\pmod{m}$'' (this is read as ``$a$ is congruent to $b$ mod $m$''), we mean that when we add or subtract $a$ with some integer number of $m$'s, we will get $b$. For example, $27\equiv 2\pmod{5}$ because if we subtract 5 5's from 27, we get 2. We can also say that $a\equiv b\pmod{m}$ if $a\div m$ and $b\div m$ have the same remainder. Now let us turn to one of the most important theorems for solving Diophantine equations:
\begin{theo}[Law of Diophantines] 
Let $m>1$ be a positive integer. If an equation has no solution modulo $m$, then it has no integer solutions.
\end{theo}
A few important properties we will use in solving Diophantines:
\begin{enumerate}
\item \textbf{Parity}. Taking odd numbers in mod 2 are always 1, and even numbers are always 0.
\item \textbf{Checking Squares}. In mod 3, squares are either 0 or 1. In mod 4, squares are also either 0 or 1. 
\item \textbf{Checking Cubes}. In mod 4, cubes are either 0, 1, or 3. 
\end{enumerate}
There are more properties, but they are easily derived (just check all the possibilities). 
\begin{exam}[Folklore]
Prove that if $x\in\mathbb{Z}$, $x^2\equiv 3\pmod{4}$ has no solutions.
\end{exam}
\begin{sol}
Note that $x$ is either $0,1,2,$ or 3 in mod 4. Let's make a chart:
\begin{figure}[H]
\centering
\begin{tabular}{c|c}
    $x\pmod{4}$ & $x^2\pmod{4}$\\
    \hline
    0 & 0\\
    1 & 1\\
    2 & 0\\
    3 & 1\\
\end{tabular}
\end{figure}
\noindent
Thus, in mod 4, squares are either 0 or 1 mod 4. This means $x^2$ can never be 3 mod 4.
\end{sol}
\begin{remark}
Although taking mod 11 does seem like a weird thing to do, we are motivated to take mod 11 due to the fact that $x^{10}\equiv1\pmod{11}$ for all $x$, which gives us a relatively low number of possible values of $x^5$.
\end{remark}
\subsection{Factoring}
Sometimes we can just factor the equation. However, it is usually extremely disguised, so \textbf{if you see a strangely arranged equation with many terms, try factoring!}

\db{Simon's Favoring Factoring Trick}, abbreviated SFFT, is useful here.
\begin{theo}[SFFT]
For all $x,y,a,b$ (usually integers),
$$xy+ax+by+ab=(x+b)(y+a).$$
\end{theo}
This isn't particularly special, but sometimes it is disguised.
\begin{exam}
Find all integral solutions to $xy-x+y=0$.
\end{exam}
\begin{sol}
Note that this is equivalent to $x(y-1)+y=0$. If we subtract 1 from both sides, we get $x(y-1)+y-1=-1$, so
$$(x+1)(y-1)=-1,$$
implying we have $x+1=1$ and $y-1=-1$ or $x+1=-1$ or $y-1=1$. Thus, the solutions for $(x,y)$ are $\boxed{(0,0)}$ or $\boxed{(-2,2)}$.
\end{sol}
Here is an important theorem to keep in mind while solving:
\begin{theo}
Let $x,y$ be positive integers and let $n=p_1^{e_1}p_2^{e_2}\ldots p_k^{e_k}$ (in other words, its prime factorization). Then the equation
$$\frac{1}{x}+\frac{1}{y}=\frac{1}{n}$$
has
$$\tau(n^2) = (2e_1+1)(2e_2+1)\ldots(2e_k+1),$$
solutions, where $\tau(n)$ is the number of divisors of $n$.
\end{theo}
Knowing key factorizations is important. For example,
$$x^3+y^3+z^3-3xyz=(x+y+z)(x^2+y^2+z^2-xy-yz-zx)$$
can help you solve problems of this nature quickly.
\section{Examples}
\begin{theo}[Linear Equation with Two Unknowns] 
The equation $ax+by=c$ has integer solutions if and only if $\gcd(a,b) \mid c$. If $\gcd(a,b) = 1$, and $x_0,y_0$ is one integer solution for $ax+by=c$, then the general solution is $x=x_0+bt,y=y_0-at$, where $t$ is an integer.
\end{theo}
There is a lot of mapping solutions to solutions in Diophantine equations. In other words, if $(x_0,y_0)$ is a solution, then $(f(x_o),g(y_0))$ is a solution, for some functions $f,g$.
\begin{exam}
Find all positive integer solutions of
$$19x+7y=260.$$
\end{exam}
\begin{sol}
A simple solution is $x=y=10$. Thus, we can apply the method listed above and solve for all solutions. It turns out the only solutions are $(10,10)$ and $(3,29)$. 
\end{sol}
Let's move on to higher degree equations:
\begin{theo}[Quadratic Equation with Three Unknowns] 
For the equation
$$x^2+y^2=z^2,$$
all positive integer solutions $(x,y,z)$ that satisfy $\gcd(x,y,z)=1$ is 
$$x=a^2-b^2,y=2ab,z=a^2+b^2,$$
where integers $a>b>0$, one even and one odd, and $\gcd(a,b)=1$.
\end{theo}
As you have probably realized, this is just the \db{parameterization} of the \db{Pythagorean triples}. This underlies something important in solving Diophantine equations -- parameterization. 
\begin{theo}[Quadratic Equation with Four Unknowns] 
For the equation
$$xy=zt,$$
all positive integer solutions can be found by letting
$$\frac{x}{z} = \frac{t}{y} = \frac{m}{n},$$
where $\gcd(m,n)$, then $x=pm,z=pn,t=qm,y=qn$, where $p=\gcd(x,z),q=\frac{y}{n}$. 
\end{theo}
\begin{exam}
Let $a,b,c,d$ be positive integers, and $ab=cd$. Prove that $a^4+b^4+c^4+d^4$ is not prime.
\end{exam}
\begin{sol}
From the theorem above, we have
$$\frac{a}{c} = \frac{d}{b} = \frac{m}{n},$$
so $a=pm,c=pn,d=qm,b=qn$. Now, plugging these numbers in, we get
$$a^4+b^4+c^4+d^4 = p^4m^4 + q^4n^4 + p^4n^4 + q^4m^4 = (p^4 + q^4)(m^4+n^4).$$
Because $p,q,m,n$ are all positive integers, then 
$$p^4+q^4>1,$$
$$m^4+n^4>1,$$
so the product cannot be prime.
\end{sol}
\begin{exam}
Let $k$ be an even number. Is it possible to find $k$ odd integers whose reciprocals add up to 1?
\end{exam}
\begin{sol}
Let $n_1,n_2,\ldots,n_k$ be $k$ odd integers. Then we must see if
$$\sum_{i=1}^k \frac{1}{n_i} = 1$$
exists. Note that if we take the common denominator, we have
$$\frac{\sum_{\text{sym}} n_2n_3\ldots n_k}{n_1n_2n_3\ldots n_k} = 1,$$
and because there are $k$ symmetric sums, the numerator is even, and the denominator is odd. Thus, it is impossible for any group of $k$ odd integers to satisfy these conditions.
\end{sol}
\begin{exam}
Find all distinct positive integers such that their product equals their sum.
\end{exam}
\begin{sol}
Let
$$x_1 \cdot x_2 \cdot \ldots \cdot x_n = x_1 + x_2 + \ldots + x_n$$
for positive integer $1\le x_1 < x_2 < x_3 < \ldots < x_n$. Then
$$(n-1)! x_n \le x_1 \cdot x_2 \cdot \ldots \cdot x_n = x_1 + x_2 + \ldots + x_n \le nx_n.$$
Thus,
$$(n-1)! < n,$$
so $n=2,3$. If $n=2$, then
$$x_1x_2 = x_1 + x_2,$$
so
$$(x_1-1)(x_2-1) = 1,$$
which implies $x_1=x_2=2$. However, they must be distinct, so there are no solutions for $n=2$. For $n=3$, we have
$$x_1x_2x_3 = x_1+x_2+x_3<3x_3,$$
so
$$x_1x_2<3.$$
Because $x_1 \neq x_2$, the only possible solution is
$$x_1=1,x_2=2.$$
Thus, $x_3=3$. Therefore, the only solution is $(1,2,3)$ over all positive integers.
\end{sol}
\problems
\minpt{36}
\psetquote{Diophan\underline{tine} and valen\underline{tine} somehow don't rhyme.}{Dylan during quaran\underline{tine}.}


\prob{1}{AMC 12A 2004/3}{
For how many ordered pairs of positive integers $(x,y)$ is $x+2y=100$?
}

\prob{1}{AMC 12B 2008/5}{
A class collects $50$ dollars to buy flowers for a classmate who is in the hospital. Roses cost $3$ dollars each, and carnations cost $2$ dollars each. No other flowers are to be used. How many different bouquets could be purchased for exactly $50$ dollars?
}

\prob{2}{AMC 12A 2005/8}{
Let $A,M$, and $C$ be digits with
\[(100A+10M+C)(A+M+C) = 2005\]
What is $A$?
}

\prob{2}{AHSME 1989/16}{
A lattice point is a point in the plane with integer coordinates. How many lattice points are on the line segment whose endpoints are $(3,17)$ and $(48,281)$? (Include both endpoints of the segment in your count.)
}

\prob{2}{AMC 12A 2006/9}{
Oscar buys $13$ pencils and $3$ erasers for $\$1.00$. A pencil costs more than an eraser, and both items cost a whole number of cents. What is the total cost, in cents, of one pencil and one eraser?
}

\prob{3}{AMC 12A 2006/14}{
Two farmers agree that pigs are worth $300$ dollars and that goats are worth $210$ dollars. When one farmer owes the other money, he pays the debt in pigs or goats, with ``change'' received in the form of goats or pigs as necessary. (For example, a $390$ dollar debt could be paid with two pigs, with one goat received in change.) What is the amount of the smallest positive debt that can be resolved in this way?
}

\prob{3}{AMC 12B 2003/18}{
Let $x$ and $y$ be positive integers such that $7x^5=11y^{13}$. The minimum possible value of $x$ has a prime factorization $a^c b^d$. What is $a+b+c+d$?
}

\prob{3}{AHSME 1968/19}{
Let $n$ be the number of ways $10$ dollars can be changed into dimes and quarters, with at least one of each coin being used. Then what is $n$?
}

\prob{4}{AMC 10B 2015/15}{
The town of Hamlet has $3$ people for each horse, $4$ sheep for each cow, and $3$ ducks for each person. Which of the following could not possibly be the total number of people, horses, sheep, cows, and ducks in Hamlet?
\begin{center}
    $\textbf{(A) }41\qquad\textbf{(B) }47\qquad\textbf{(C) }59\qquad\textbf{(D) }61\qquad\textbf{(E) }66$
\end{center}
}

\prob{6}{AMC 12 2001/21}{
Four positive integers $a$, $b$, $c$, and $d$ have a product of $8!$ and satisfy:
\begin{align*} ab + a + b & = 524 \\  bc + b + c & = 146 \\  cd + c + d & = 104 \end{align*}
What is $a-d$?
}

\prob{6}{AMC 12A 2014/19}{
There are exactly $N$ distinct rational numbers $k$ such that $|k|<200$ and\[5x^2+kx+12=0\]has at least one integer solution for $x$. What is $N$?
}

\prob{6}{AMC 12A 2019/15}{
Positive real numbers $a$ and $b$ have the property that\[\sqrt{\log{a}} + \sqrt{\log{b}} + \log \sqrt{a} + \log \sqrt{b} = 100\]and all four terms on the left are positive integers, where log denotes the base 10 logarithm. What is $ab$?
}

\prob{6}{AIME 1997/1}{
How many of the integers between 1 and 1000, inclusive, can be expressed as the difference of the squares of two nonnegative integers?
}

\prob{9}{AIME II 2000/2}{
A point whose coordinates are both integers is called a lattice point. How many lattice points lie on the hyperbola $x^2 -y^2 = 2000^2$?
}

\prob{9}{AIME I 2015/3}{
There is a prime number $p$ such that $16p+1$ is the cube of a positive integer. Find $p$.
}

\prob{9}{AIME I 2008/4}{
There exist unique positive integers $x$ and $y$ that satisfy the equation $x^2 + 84x + 2008 = y^2$. Find $x + y$.
}

\end{document}