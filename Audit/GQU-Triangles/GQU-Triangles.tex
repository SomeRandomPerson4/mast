\documentclass{article}

\usepackage[mast]{dennis}

\title{Lengths and Areas in Triangles}
\author{Dennis Chen}
\date{GQU}

\begin{document}

\maketitle

\section{Lengths}
There are a couple of important lengths in a triangle. These are the lengths of cevians, the inradius/exradius, and the circumradius.

\subsection{Law of Cosines and Stewart's}
We discuss how to find the third side of a triangle given two sides and an included angle, and use this to find a general formula for the length of a cevian.

\begin{theo}[Law of Cosines]
Given $\triangle ABC,$ $a^2+b^2-2ab\cos C=c^2.$
\end{theo}

\begin{pro}
Let the foot of the altitude from $A$ to $BC$ be $H.$ Then note that $A=b\sin C,$ $CH=b\cos C,$ and $BH=|a-b\cos C|.$ (The absolute value is because $\angle B$ can either be acute or obtuse.) Then note by the Pythagorean Theorem, $(b\sin C)^2+(a-b\cos C)^2=a^2+b^2-2ab\cos C=c^2.$
\begin{center}
    \begin{asy}
    size(4cm);
    dot((-3,0));
    dot((1,0));
    dot((0,3));
    dot((0,0));
    draw((-3,0)--(1,0)--(0,3)--cycle);
    draw((0,3)--(0,0));
    
    label("$C$",(-3,0),SW);
    label("$B$",(1,0),SE);
    label("$A$",(0,3),N);
    label("$H$",(0,0),S);
    \end{asy}
\end{center}
\end{pro}

\begin{theo}[Stewart's Theorem]
Consider $\triangle ABC$ with cevian $AD,$ and denote $BD=m,$ $CD=n,$ and $AD=d.$ Then $man+dad=bmb+cnc.$
\end{theo}

\begin{pro}
We use the Law of Cosines. Note that
\[\cos \angle ADB=\frac{d^2+m^2-c^2}{2dm}=-\frac{d^2+n^2-b^2}{2dn}=-\cos \angle ADC.\]
Multiplying both sides by $2dmn$ yields
\[c^2n-d^2n-m^2n=-bm^2+d^2m+mn^2\]
\[b^2m+c^2n=mn(m+n)+d^2(m+n)\]
\[bmb+cnc=man+dad.\]
\begin{center}
    \begin{asy}
    size(4cm);
    dot((-3,0));
    dot((1,0));
    dot((0,3));
    draw((-3,0)--(1,0)--(0,3)--cycle);
    draw((0,3)--(-1,0));
    
    label("$C$",(-3,0),SW);
    label("$B$",(1,0),SE);
    label("$A$",(0,3),N);
    label("$D$",(-1,0),S);
    \end{asy}
\end{center}
\end{pro}

Here are two corollaries that will save you a lot of time in computational contests.

\begin{fact}[Length of Angle Bisector]
In $\triangle ABC$ with angle bisector $AD,$ denote $BD=x$ and $CD=y.$ Then
\[AD=\sqrt{bc-xy}.\]
\end{fact}

\begin{fact}[Length of Median]
In $\triangle ABC$ with median $AD,$
\[AD=\frac{\sqrt{2b^2+2c^2-a^2}}{2}.\]
\end{fact}

\subsection{Law of Sines and the Circumradius}
The Law of Sines is a good way to length chase with a lot of angles.

\begin{theo}[Law of Sines]
In $\triangle ABC$ with circumradius $R,$
\[\frac{a}{\sin A}=\frac{b}{\sin B}=\frac{c}{\sin C}=2R.\]
\end{theo}

\begin{pro}
We only need to prove that $\frac{a}{\sin A}=2R,$ and the rest will follow.

Let the line through $B$ perpendicular to $BC$ intersect $(ABC)$ again at $A'.$ Then note that $A'C=2R$ by Thale's. By the Inscribed Angle Theorem, $\sin \angle CA'B=\sin A,$ so $\frac{a}{\sin A}=\frac{a}{\sin \angle CA'B}=\frac{a}{\frac{a}{2R}}=2R.$
\begin{center}
    \begin{asy}
    import olympiad;
size(4cm);
pair A=(-1,5), B=(-4,-1), C=(4,-1), X, O;
O = circumcenter(A,B,C);
X = (2O-C);

draw(C--A--B--C);
draw(C--X--B);
draw(circumcircle(A,B,C));

label("$A$",A,(-1,1));label("$B$",B,(-1,-1));label("$C$",C,(1,-1));label("$A'$",X,(-1,1));label("$O$",O,(1,1));

dot(A^^B^^C^^X^^O);
    \end{asy}
\end{center}
\end{pro}

Other texts will call this the Extended Law of Sines. But the Extended Law of Sines has a better proof than the "normal" Law of Sines, and redundancy is bad.

The Law of Sines gives us the Angle Bisector Theorem.

\begin{theo}[Angle Bisector Theorem]
Let $D$ be the point on $BC$ such that $\angle BAD=\angle DAC.$ Then $\frac{AB}{BD}=\frac{AC}{CD}.$
\end{theo}

\begin{pro}
By the Law of Sines, $\frac{\sin\angle ADB}{\sin\angle BAD}=\frac{AB}{BD}$ and $\frac{\sin\angle ADC}{\sin\angle CAD}=\frac{AC}{CD}.$ But note that $\angle BAD=\angle ADC$ and $\angle BAD+\angle CAD=180^{\circ},$ so $\frac{AB}{BD}=\frac{AC}{CD}.$
\begin{center}
    \begin{asy}
    import markers;
import olympiad;
size(4cm);
real a,b,c,d;
pair A=(1,9), B=(-11,0), C=(4,0), D; b = abs(C-A); c = abs(B-A); D = (b*B+c*C)/(b+c);
draw(A--B--C--A--D);
label("$A$",A,(1,1));label("$B$",B,(-1,-1));label("$C$",C,(1,-1));label("$D$",D,(0,-1)); dot(A^^B^^C^^D);
markangle(radius=15,n=1,B,A,D,marker(markinterval(stickframe(n=1,2mm),true)));
markangle(radius=15,n=1,D,A,C,marker(markinterval(stickframe(n=1,2mm),true)));
\end{asy}
\end{center}
\end{pro}

In fact, the Angle Bisector Theorem can be generalized in what is known as the ratio lemma.

\begin{theo}[Ratio Lemma]
Consider $\triangle ABC$ with point $P$ on $BC.$ Then $\frac{BP}{CP}=\frac{c\sin \angle BAP}{b\sin \angle CAP}.$
\end{theo}

The proof is pretty much identical to the proof for Angle Bisector Theorem.

\begin{pro}
By the Law of Sines, $BP=\frac{c\sin\angle BAP}{\sin\angle APB}$ and $CP=\frac{b\sin\angle CAP}{\sin\angle APC}.$ Since $\sin\angle APB=\sin\angle APC,$
\[\frac{BP}{CP}=\frac{c\sin \angle BAP}{b\sin \angle CAP}.\]
\end{pro}

Note that this remains true even if $P$ is on the \textit{extension} of $BC.$

Here's a classic example that cleverly utilizes the Law of Sines.

\begin{exam}
Show that $\triangle ABC$ is similar to the triangle with side lengths $\sin A,\sin B,\sin C.$
\end{exam}

\begin{sol}
Note that $\sin A=\frac{a}{2R},$ so the similarity factor is $2R.$
\end{sol}

We'll utilize this concept further in the next example.

\begin{exam}
Consider $\triangle ABC$ with side lengths $AB=13,$ $BC=5,$ and $CA=12.$ Find the area of the triangle with side lengths $\sin A,$ $\sin B,$ and $\sin C.$
\end{exam}

\begin{sol}
Note that $[ABC]=60$ and the triangle with lengths $\sin A,$ $\sin B,$ and $\sin C$ is similar to $\triangle ABC$ with a scale factor of $13.$ Thus the desired area is $\frac{60}{13^2}=\frac{60}{169}.$
\end{sol}

It's possible to just directly use the values of $\sin A,$ $\sin B,$ and $\sin C,$ but this will not work for general triangles.

\subsection{The Incircle, Excircle, and Tangent Chasing}
We provide formulas for the inradius, exradii, and take a look at some uses of the Two Tangent Theorem. Recall that the Two Tangent Theorem states that if the tangents from $P$ to $\omega$ intersect $\omega$ at $A,B,$ then $PA=PB.$

\begin{theo}[$rs$]
In $\triangle ABC$ with inradius $r,$
\[[ABC]=rs.\]
\end{theo}

\begin{pro}
Note that $[ABC]=r\cdot\frac{a+b+c}{2}=rs.$

\begin{center}
\begin{asy}
import olympiad;
size(4cm);
pair A = dir(110);
	pair B = dir(210);
	pair C = dir(330);
	pair I = incenter(A, B, C);
	draw(A--B--C--cycle);
	draw(A--I--B);
	draw(I--C);
	draw(circle(I,length(I-foot(I,B,C))));
	dot("$A$", A, dir(A));
	dot("$B$", B, dir(190));
	dot("$C$", C, dir(-10));
	dot("$I$", I, dir(270));
\end{asy}
\end{center}
\end{pro}

A useful fact of the incircle is that the length of the tangents from $A$ is $s-a.$ Similar results hold for the $B,C$ tangents to the incircle.

\begin{fact}[Tangents to Incircle]
Let the incircle of $\triangle ABC$ be tangent to $BC,CA,AB$ at $D,E,F.$ Then
\[AE=AF=s-a\]
\[BF=BD=s-b\]
\[CD=CE=s-c.\]
\end{fact}

\begin{pro}
Note that by the Two Tangent Theorem, $AE=AF=x,$ $BF=BD=y,$ and $CD=CE=z.$ Also note that
\[BD+CD=y+z=a\]
\[CE+EA=z+x=b\]
\[AF+FB=x+y=c.\]
Adding these equations gives $2x+2y+2z=a+b+c=2s,$ implying $x+y+z=s.$ Thus
\[x=AE=AF=s-a\]
\[y=BF=BD=s-b\]
\[z=CD=CE=s-c,\]
as desired.
\begin{center}
\begin{asy}
import olympiad;
size(4cm);
pair A = dir(110);
	pair B = dir(210);
	pair C = dir(330);
	pair I = incenter(A, B, C);
    pair D,E,F;
    D=foot(I,B,C);
    E=foot(I,C,A);
    F=foot(I,A,B);
	draw(A--B--C--cycle);
	draw(circle(I,length(I-foot(I,B,C))));
	dot("$A$", A, dir(A));
	dot("$B$", B, dir(190));
	dot("$C$", C, dir(-10));
    dot("$D$", D, dir(-90));
    dot("$E$", E, dir(40));
    dot("$F$", F, dir(155));
\end{asy}
\end{center}
\end{pro}

\begin{theo}[$r_a(s-a)$]
In $\triangle ABC$ with $A$ exradius $r_a,$
\[[ABC]=r_a(s-a).\]
\end{theo}

\begin{pro}
Let $AB,AC$ be tangent to the $A$ excircle at $P,Q,$ respectively, and let $BC$ be tangent to the $A$ excircle at $D.$ Then note that by the Two Tangent Theorem, $PB=BD$ and $DC=CQ.$ Thus $[ABC]=[API_A]+[AQI_A]-2[BI_AC]=r_a\cdot \frac{s+s-2a}{2}=r_a(s-a).$
\end{pro}
\begin{center}
    \begin{asy}
    import olympiad;
    pair excenter(pair X, pair Y, pair Z){
pair A, C;
A=X+expi((angle(X-Y)+angle(Z-X))/2);
C=Z+expi((angle(Z-Y)+angle(X-Z))/2);
return extension(A,X,C,Z);
}
    size(4cm);
    pair A = dir(110);
	pair B = dir(210);
	pair C = dir(330);
	real a,b,c;
    a=abs(B-C); b = abs(C-A); c = abs(B-A);
	pair I = incenter(A, B, C);
	pair exa=excenter(C,A,B);
	draw(A--B--C--cycle);
	pair L = dir(270);
	pair I_A = 2*L-I;
	pair D=foot(I_A,B,C);
    pair P=foot(I_A,A,B);
    pair Q=foot(I_A,A,C);
    draw(B--P);
	draw(C--Q);
    dot("$A$", A, dir(A));
	dot("$B$", B, dir(190));
	dot("$C$", C, dir(0));
	dot("$I_A$", I_A, dir(I_A));
	dot("$D$",D,dir(270));
	draw(circle(exa,length(exa-foot(exa,B,C))));
    dot("$P$",P,dir(190));
    dot("$Q$",Q,dir(350));
    draw(P--I_A--Q,dotted);
    draw(A--I_A,dotted);
    \end{asy}
\end{center}

The proof also implies the following corollary.

\begin{fact}[Tangents to Excircle]
Let the $A$ excircle of $\triangle ABC$ be tangent to $BC$ at $D$. Then $BD=s-c$ and $CD=s-b.$

Analogous equations hold for the $B$ and $C$ excircles.
\end{fact}

\begin{pro}
Let the $A$ excircle be tangent to line $AB$ at $P$ and line $AC$ at $Q.$ Note that $AP=AB+BP=c+BD$ and $AQ=AC+CQ=b+CD$ by the Two Tangent Theorem. Applying the Two Tangent Theorem again gives $AP=AQ,$ or $c+BD=b+CD.$ Also note that $AP+AQ=b+c+BD+DC=2s,$ so $AP=AQ=s$ and $s=c+BD=b+CD.$ Thus $BD=s-c$ and $CD=s-b.$
\begin{center}
\begin{asy}
import olympiad;
    pair excenter(pair X, pair Y, pair Z){
pair A, C;
A=X+expi((angle(X-Y)+angle(Z-X))/2);
C=Z+expi((angle(Z-Y)+angle(X-Z))/2);
return extension(A,X,C,Z);
}
    size(4cm);
    pair A = dir(110);
	pair B = dir(210);
	pair C = dir(330);
	real a,b,c;
    a=abs(B-C); b = abs(C-A); c = abs(B-A);
	pair I = incenter(A, B, C);
	pair exa=excenter(C,A,B);
	draw(A--B--C--cycle);
	pair L = dir(270);
	pair I_A = 2*L-I;
	pair D=foot(I_A,B,C);
    pair P=foot(I_A,A,B);
    pair Q=foot(I_A,A,C);
    draw(B--P);
	draw(C--Q);
    dot("$A$", A, dir(A));
	dot("$B$", B, dir(190));
	dot("$C$", C, dir(0));
	dot("$D$",D,dir(270));
	draw(circle(exa,length(exa-foot(exa,B,C))));
    dot("$P$",P,dir(190));
    dot("$Q$",Q,dir(350));
\end{asy}
\end{center}
\end{pro}

Keep these area and length conditions in mind when you see incircles and excircles.

\section{Areas}

There are a variety of methods to find area. For harder problems, computing the area in two different ways can give useful information about the configuration.

\begin{theo}[$\frac{bh}{2}$]
The area of a triangle is $\frac{bh}{2}.$
\end{theo}

\begin{pro}
The area of each right triangle is half of the area of the rectangle it is in.
\begin{center}
    \begin{asy}
    size(3cm);
    draw((0,0)--(4,0)--(4,3)--(0,3)--cycle);
    draw((0,0)--(1,3)--(4,0));
    draw((1,3)--(1,0));
    \end{asy}
\end{center}
\end{pro}

\begin{theo}[$rs$]
The area of a triangle is $rs,$ where $r$ is the inradius and $s$ is the semiperimeter.
\end{theo}

We have already proved this in Length Chasing - but we mention this theorem again because it is useful for area too.

\begin{theo}[$\frac{1}{2}ab\sin C$]
The area of a triangle is $\frac{1}{2}ab\sin C,$ where $a,b$ are side lengths and $C$ is the included angle.
\end{theo}

\begin{pro}
Drop an altitude from $B$ to $AC$ and let it have length $h.$ Then note $\frac{1}{2}\cdot a\sin C\cdot b=\frac{1}{2}\cdot hb=\frac{bh}{2}.$
\begin{center}
    \begin{asy}
    size(4cm);
    draw((0,0)--(1,3)--(4,0)--cycle);
    draw((1,3)--(1,0));
    
    dot((0,0));
    label("$C$",(0,0),SW);
    
    dot((1,0));
    label("$H$",(1,0),S);
    
    dot((4,0));
    label("$A$",(4,0),SE);
    
    dot((1,3));
    label("$B$",(1,3),N);
    \end{asy}
\end{center}
\end{pro}

We present a useful corollary of this theorem.

\begin{fact}[$\frac{[PAB]}{[PXY]}=\frac{PA\cdot PB}{PX\cdot PY}$]
Let $P,A,X$ be on $\ell_1$ and $P,B,Y$ be on $\ell_2.$ Then $\frac{[PAB]}{[PXY]}=\frac{PA\cdot PB}{PX\cdot PY}.$
\end{fact}

\begin{pro}
Note $\frac{[PAB]}{[PXY]}=\frac{\frac{1}{2}\cdot PA\cdot PB\cdot \sin\theta}{\frac{1}{2}\cdot PX\cdot PY\cdot \sin\theta}=\frac{PA\cdot PB}{PX\cdot PY},$ where $\theta=\angle APB.$

This works for all configurations since $\sin\theta=\sin(180-\theta).$
\end{pro}

Here is a very difficult example done with the help of some tricky angle chasing, trig addition formulas, and the sine area formula.

\begin{exam}[AMC 12A 2021/24]
Semicircle $\Gamma$ has diameter $\overline{AB}$ of length $14$. Circle $\Omega$ lies tangent to $\overline{AB}$ at a point $P$ and intersects $\Gamma$ at points $Q$ and $R$. If $QR=3\sqrt3$ and $\angle QPR=60^\circ$, then the area of $\triangle PQR$ is $\frac{a\sqrt{b}}{c}$, where $a$ and $c$ are relatively prime positive integers, and $b$ is a positive integer not divisible by the square of any prime. What is $a+b+c$?
\end{exam}

\begin{sol}
Let $S$ be the center of $\Omega,$ and note that by the Law of Sines, the circumradius of $\triangle PQR$ is $\frac{QR}{2\sin RPQ}=\frac{3\sqrt{3}}{\sqrt{3}}=3.$ Also note that by the Pythagorean Theorem, the distance from $O$ to $RQ$ is $\sqrt{7^2-(\frac{3\sqrt{3}}{2})^2}=\frac{13}{2}$, as $\triangle ORQ$ is isosceles. So $SO=\frac{13}{2}-\frac{3}{2}=5$ and $SP=4$ by the Pythagorean Theorem.

Let $\angle OSP=\theta$ and note that $\arccos\theta=\frac{3}{5}.$ Now, by the Sine Area Formula,
\begin{align*}
[PQR]=[PSR]+[PSQ]+[RSQ]&=\frac{9}{2}(\sin 60^{\circ}+\sin(150^{\circ}-\theta)+\sin(150^{\circ}+\theta)) \\
&=\frac{9}{2}(\frac{\sqrt{3}}{2}+2\sin 150^{\circ}\cos \theta) \\
&=\frac{9}{2}(\frac{\sqrt{3}}{2}+\frac{3\sqrt{3}}{5}) \\
&=\frac{99\sqrt{3}}{20}.
\end{align*}
Thus the answer is $99+3+20=122.$

\begin{center}
\begin{asy}
pair A = (-7,0), O = origin, B = (7,0), pS = (-4,3), P = (-4,0);
path semi = arc(O,7,0,180);
path c = circle(pS,3);

draw(semi^^A--B);
draw(O--pS--P,blue);
pair[] qr = intersectionpoints(semi,c);
draw(qr[0]--qr[1],red);

draw(c, purple);

dot("$A$", A, S);
dot("$S$", pS, NE);
dot("$O$", O, S);
dot("$P$", P, S);
dot("$Q$", qr[0], N);
dot("$R$", qr[1], W);
draw(qr[0]--O--qr[1], red+dotted);
\end{asy}

\end{center}
\end{sol}

\begin{theo}[$\frac{abc}{4R}$]
In $\triangle ABC$ with side lengths $a,b,c$ and circumradius $R,$
\[[ABC]=\frac{abc}{4R}.\]
\end{theo}

\begin{pro}
Note that $[ABC]=\frac{1}{2}ab\sin C=\frac{1}{2}ab\cdot\frac{c}{2R}=\frac{abc}{4R}.$
\end{pro}

Heron's Formula can find the area of a triangle with \textit{only} the side lengths.

\begin{theo}[Heron's Formula]
In $\triangle ABC$ with sidelengths $a,b,c$ such that $s=\frac{a+b+c}{2},$
\[[ABC]=\sqrt{s(s-a)(s-b)(s-c)}.\]
\end{theo}

\begin{pro}
Since $\cos C=\frac{a^2+b^2-c^2}{2ab},$ the Pythagorean Identity gives us \[\sin C=\sqrt{\frac{4a^2b^2-(a^2+b^2-c^2)^2}{4a^2b^2}}=\sqrt{\frac{(a+b+c)(-a+b+c)(a-b+c)(a+b-c)}{4a^2b^2}}.\] So \[\frac{1}{2}ab\sin C=\sqrt{\left(\frac{a+b+c}{2}\right)\left(\frac{-a+b+c}{2}\right)\left(\frac{a-b+c}{2}\right)\left(\frac{a+b-c}{2}\right)}=\sqrt{s(s-a)(s-b)(s-c)}.\]
\end{pro}

Heron's Formula has a reputation for being notoriously tricky to prove, but the proof isn't too bad if you consider what we're actually doing.
\begin{enumerate}
	\item Use the Law of Cosines to find $\cos C.$
	
	\item Use the Pythagorean Identity to find $\sin C.$
	
	\item Use $\frac{1}{2}ab\sin C$ to find $[ABC].$
	
	\item Clean the expression up.
\end{enumerate}

\begin{fact}[Heron's with Altitudes]
If $x,y,z$ are the lengths of the altitudes of $\triangle ABC,$
\[\frac{1}{[ABC]}=\sqrt{\left(\frac{1}{x}+\frac{1}{y}+\frac{1}{z}\right)\left(-\frac{1}{x}+\frac{1}{y}+\frac{1}{z}\right)\left(\frac{1}{x}-\frac{1}{y}+\frac{1}{z}\right)\left(\frac{1}{x}+\frac{1}{y}-\frac{1}{z}\right)}.\]
\end{fact}

To prove this, substitute $x=\frac{[ABC]}{2a}.$

\pagebreak

\section{Problems}

\minpt{55}

\psetquote{Do you know, Poole, that you and I are about to place ourselves in a situation of some peril?}{Strange Case of Dr. Jekyll and Mr. Hyde}

\prob{1}{}{Find the inradius of the triangles with the following lengths:
    
    \begin{itemize}
        \Item $3,4,5$
        
        \Item $5,12,13$
        
        \Item $13,14,15$
        
        \Item $5,7,8$
    \end{itemize}

    (These are arranged by difficulty. All of these are good to know.)}
    
\prob{2}{}{Prove that in a right triangle with legs of length $a,b$ and hypotenuse with length $c,$ $r=\frac{a+b-c}{2}.$}

\prob{2}{}{In $\triangle ABC,$ $AB=5,$ $BC=12,$ and $CA=13.$ Points $D,E$ are on $BC$ such that $BD=DC$ and $\angle BAE=\angle CAE.$ Find $[ADE].$}

\req{2}{}{Find the maximum area of a triangle with two of its sides having lengths $10,11.$}

\prob{2}{e-dchen Mock MATHCOUNTS}{Consider rectangle $ABCD$ such that $AB=2$ and $BC=1.$ Let $X,Y$ trisect $AB.$ Then let $DX$ and $DY$ intersect $AC$ at $P$ and $Q,$ respectively. What is the area of quadrilateral $XYQP?$}

\req{2}{Autumn Mock AMC 10}{Equilateral triangle $ABC$ has side length $6$. Points $D, E, F$ lie within the lines $AB, BC$ and $AC$ such that $BD=2AD$, $BE=2CE$, and $AF=2CF$. Let $N$ be the numerical value of the area of triangle $DEF$. Find $N^2$.}

\prob{2}{AIME I 2019/3}{In $\triangle PQR$, $PR=15$, $QR=20$, and $PQ=25$. Points $A$ and $B$ lie on $\overline{PQ}$, points $C$ and $D$ lie on $\overline{QR}$, and points $E$ and $F$ lie on $\overline{PR}$, with $PA=QB=QC=RD=RE=PF=5$. Find the area of hexagon $ABCDEF$.}

\prob{3}{AMC 8 2019/24}{In triangle $ABC$, point $D$ divides side $\overline{AC}$ so that $AD:DC=1:2$. Let $E$ be the midpoint of $\overline{BD}$ and let $F$ be the point of intersection of line $BC$ and line $AE$. Given that the area of $\triangle ABC$ is $360$, what is the area of $\triangle EBF$?}
\begin{center}
\begin{asy}
import olympiad;
size(5cm);
pair A,B,C,DD,EE,FF;
B = (0,0); C = (3,0); 
A = (1.2,1.7);
DD = (2/3)*A+(1/3)*C;
EE = (B+DD)/2;
FF = intersectionpoint(B--C,A--A+2*(EE-A));
draw(A--B--C--cycle);
draw(A--FF); 
draw(B--DD);dot(A); 
label("$A$",A,N);
dot(B); 
label("$B$",
B,SW);dot(C); 
label("$C$",C,SE);
dot(DD); 
label("$D$",DD,NE);
dot(EE); 
label("$E$",EE,NW);
dot(FF); 
label("$F$",FF,S);
\end{asy}
\end{center}

\prob{3}{}{Consider $\triangle ABC$ such that $AB=8,$ $BC=5,$ and $CA=7.$ Let $AB$ and $CA$ be tangent to the incircle at $T_C,$ $T_B,$ respectively. Find $[AT_BT_C].$}

\prob{3}{}{Consider trapezoid $ABCD$ with bases $AB$ and $CD.$ If $AC$ and $BD$ intersect at $P,$ prove the sum of the areas of $\triangle ABP$ and $\triangle CDP$ is at least half the area of trapezoid $ABCD.$}

\prob{4}{AIME I 2001/4}{In triangle $ABC$, angles $A$ and $B$ measure $60$ degrees and $45$ degrees, respectively. The bisector of angle $A$ intersects $\overline{BC}$ at $T$, and $AT=24$. The area of triangle $ABC$ can be written in the form $a+b\sqrt{c}$, where $a$, $b$, and $c$ are positive integers, and $c$ is not divisible by the square of any prime. Find $a+b+c$.}

\prob{6}{PUMaC 2016}{Let $ABCD$ be a cyclic quadrilateral with circumcircle $\omega$ and let $AC$ and $BD$ intersect at $X$. Let the line through $A$ parallel to $BD$ intersect line $CD$ at $E$ and $\omega$ at $Y \ne A$. If $AB = 10, AD = 24, XA = 17$, and $XB = 21$, then the area of $\triangle DEY$ can be written in simplest form as $\frac{m}{n}$. Find $m + n$.}
    
\req{6}{CIME 2020}{An excircle of a triangle is a circle tangent to one of the sides of the triangle and the extensions of the other two sides. Let $ABC$ be a triangle with $\angle ACB=90^\circ$ and let $r_A$, $r_B$, $r_C$ denote the radii of the excircles opposite to $A$, $B$, $C$, respectively. If $r_A=9$ and $r_B=11$, then $r_C$ can be expressed in the form $m+\sqrt{n}$, where $m$ and $n$ are positive integers and $n$ is not divisible by the square of any prime. Find $m+n$.}

\prob{6}{AIME II 2019/11}{Triangle $ABC$ has side lengths $AB=7, BC=8,$ and $CA=9.$ Circle $\omega_1$ passes through $B$ and is tangent to line $AC$ at $A.$ Circle $\omega_2$ passes through $C$ and is tangent to line $AB$ at $A.$ Let $K$ be the intersection of circles $\omega_1$ and $\omega_2$ not equal to $A.$ Then $AK=\tfrac mn,$ where $m$ and $n$ are relatively prime positive integers. Find $m+n.$}

\prob{6}{AIME II 2005/14}{In triangle $ABC, AB=13, BC=15,$ and $CA = 14.$ Point $D$ is on $\overline{BC}$ with $CD=6.$ Point $E$ is on $\overline{BC}$ such that $\angle BAE = \angle CAD.$ Given that $BE=\frac pq$ where $p$ and $q$ are relatively prime positive integers, find $q.$}

\prob{6}{ART 2019/6}{Consider unit circle $O$ with diameter $AB.$ Let $T$ be on the circle such that $TA<TB.$ Let the tangent line through $T$ intersect $AB$ at $X$ and intersect the tangent line through $B$ at $Y.$ Let $M$ be the midpoint of $YB,$ and let $XM$ intersect circle $O$ at $P$ and $Q.$ If $XP=MQ,$ find $AT.$}

\prob{9}{AMC 12A 2017/24}{Quadrilateral $ABCD$ is inscribed in circle $O$ and has sides $AB = 3$, $BC = 2$, $CD = 6$, and $DA = 8$. Let $X$ and $Y$ be points on $\overline{BD}$ such that
\[\frac{DX}{BD} = \frac{1}{4} \quad \text{and} \quad \frac{BY}{BD} = \frac{11}{36}.\]Let $E$ be the intersection of intersection of line $AX$ and the line through $Y$ parallel to $\overline{AD}$. Let $F$ be the intersection of line $CX$ and the line through $E$ parallel to $\overline{AC}$. Let $G$ be the point on circle $O$ other than $C$ that lies on line $CX$. What is $XF \cdot XG$?}

\prob{9}{AIME II 2016/10}{Triangle $ABC$ is inscribed in circle $\omega$. Points $P$ and $Q$ are on side $\overline{AB}$ with $AP<AQ$. Rays $CP$ and $CQ$ meet $\omega$ again at $S$ and $T$ (other than $C$), respectively. If $AP=4,PQ=3,QB=6,BT=5,$ and $AS=7$, then $ST=\frac{m}{n}$, where $m$ and $n$ are relatively prime positive integers. Find $m+n$.}

\prob{9}{IMO 2003/4}{Let $ABCD$ be a cyclic quadrilateral. Let $P$, $Q$, $R$ be the feet of the perpendiculars from $D$ to the lines $BC$, $CA$, $AB$, respectively. Show that $PQ=QR$ if and only if the bisectors of $\angle ABC$ and $\angle ADC$ are concurrent with $AC$.}

\prob{9}{AIME I 2019/11}{In $\triangle ABC$, the sides have integer lengths and $AB=AC$. Circle $\omega$ has its center at the incenter of $\triangle ABC$. An excircle of $\triangle ABC$ is a circle in the exterior of $\triangle ABC$ that is tangent to one side of the triangle and tangent to the extensions of the other two sides. Suppose that the excircle tangent to $\overline{BC}$ is internally tangent to $\omega$, and the other two excircles are both externally tangent to $\omega$. Find the minimum possible value of the perimeter of $\triangle ABC$.}

\prob{9}{AIME I 2020/13}{Point $D$ lies on side $BC$ of $\triangle ABC$ so that $\overline{AD}$ bisects $\angle BAC$. The perpendicular bisector of $\overline{AD}$ intersects the bisectors of $\angle ABC$ and $\angle ACB$ in points $E$ and $F$, respectively. Given that $AB=4$, $BC=5$, $CA=6$, the area of $\triangle AEF$ can be written as $\tfrac{m\sqrt n}p$, where $m$ and $p$ are relatively prime positive integers, and $n$ is a positive integer not divisible by the square of any prime. Find $m+n+p$.}

\prob{13}{CIME 2019}{Let $\triangle ABC$ be a triangle with circumcenter $O$ and incenter $I$ such that the lengths of the three segments $AB,$ $BC$ and $CA$ form an increasing arithmetic progression in this order$.$ If $AO=60$ and $AI=58,$ then the distance from $A$ to $BC$ can be expressed as $\tfrac mn,$ where $m$ and $n$ are relatively prime positive integers$.$ Find $m+n.$}
    
\prob{13}{AIME I 2020/15}{Let $ABC$ be an acute triangle with circumcircle $\omega$ and orthocenter $H$. Suppose the tangent to the circumcircle of $\triangle HBC$ at $H$ intersects $\omega$ at points $X$ and $Y$ with $HA=3$, $HX=2$, $HY=6$. The area of $\triangle ABC$ can be written as $m\sqrt n$, where $m$ and $n$ are positive integers, and $n$ is not divisible by the square of any prime. Find $m+n$.}

\end{document}