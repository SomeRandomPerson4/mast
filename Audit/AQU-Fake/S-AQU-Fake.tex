\documentclass{article}

\usepackage[mast]{dennis}

\title{Solutions to Fake Algebra}
\author{Allen Wang}
\date{AQU}

\begin{document}

\maketitle

\toc

\pagebreak\section{Unsourced}

If $a<b<c<a+b,$ order $\frac{b^2+c^2-a^2}{bc},\frac{c^2+a^2-b^2}{ca},\frac{a^2+b^2-c^2}{ab}$ in ascending order.

\subsection{Solution}

Consider a triangle with side lengths $a<b<c$, and let $\angle A,\angle B,\angle C$ denote the angles opposite the sides $a,b,c$ as usual. This is motivated by our somewhat-symmetric expressions, as well as the requirement that $c<a+b$. Observe that by Law of Cosines:
\begin{align*}
    b^2+c^2-2bc\cos \angle A&=a^2\\
    b^2+c^2-a^2&=2bc\cos\angle A\\
    2\cos \angle A=\frac{b^2+c^2-a^2}{bc}.
\end{align*}
Doing this to the other two expressions, we want to sort $2\cos \angle A,2\cos\angle B, 2\cos\angle C$ in ascending order. Since $\cos$ is decreasing on $[0,\pi]$—in degrees, this is $[0^\circ,180^\circ]$—we want to sort $\angle A,\angle B,\angle C$ in descending order. Noting that, in a triangle, the shortest and longest sides are opposite the smallest and largest angles respectively, we have $\angle C>\angle B>\angle A$. Hence $2\cos\angle C<2\cos\angle B<2\cos\angle A$, so we conclude that $\ansbold{\frac{a^2+b^2-c^2}{ab},\frac{c^2+a^2-b^2}{ca},\frac{b^2+c^2-a^2}{bc}}$ is the correct ordering.

\pagebreak\section{Unsourced}
Prove that the $A$ and $B$ angle bisectors of a triangle are equal in length if and only if $BC=CA.$

\subsection{Solution}

\pagebreak\section{AIME 1986/2}
Evaluate the product $(\sqrt 5+\sqrt6+\sqrt7)(-\sqrt 5+\sqrt6+\sqrt7)(\sqrt 5-\sqrt6+\sqrt7)(\sqrt 5+\sqrt6-\sqrt7).$

\subsection{Solution}
Consider a triangle with side lengths $2\sqrt{5},2\sqrt{6},2\sqrt{7}$. By Heron's formula, the area of this triangle is:
$$\sqrt{(\sqrt 5+\sqrt6+\sqrt7)(-\sqrt 5+\sqrt6+\sqrt7)(\sqrt 5-\sqrt6+\sqrt7)(\sqrt 5+\sqrt6-\sqrt7)}.$$
To be continued.

\pagebreak\section{Unsourced}
Let $x$ and $y$ be real numbers such that $(x - 5)^2 + (y - 5)^2 = 18.$ Determine the maximum value of $\frac{y}{x}.$

\subsection{Solution 1}
It is possible to solve this purely algebraically.
Suppose $\frac{y}{x}=k \implies y=kx$. We wish to maximize $k$. Substituting $y=kx$ into our equation and treating it as a quadratic in $x$ gives:
\begin{align*}
    (x-5)^2+(kx-5)^2&=18\\
    x^2-10x+25+k^2x^2-10kx+25&=18\\
    (k^2+1)x^2-10(k+1)x+32&=0.
\end{align*}
For this to have real roots, the discriminant must be nonnegative, so:
\begin{align*}
    (10(k+1))^2-4(k^2+1)(32)&\geq 0\\
    100k^2+200k+100-128k^2-128&\geq 0\\
    -28k^2+200k-28&\geq 0\\
    7k^2-50k+7&\leq 0\\
    (7k-1)(k-7)&\leq 0.
\end{align*}
Hence the maximum value of $k$ is $\ansbold{7}$, since for $k>7$ we have $(7k-1)(k-7)>0$.

\subsection{Solution 2}

\pagebreak\section{Unsourced}
Let $a,b,c$ be positive reals. Prove that $\sqrt{a^2 - ab + b^2} + \sqrt{b^2 - bc + c^2} \geq \sqrt{a^2 + ac + c^2}.$

\subsection{Solution}

\pagebreak\section{Unsourced}
Minimze $\sqrt{x^2-3x+3}+\sqrt{y^2-3y+3}+\sqrt{x^2-\sqrt{3}xy+y^2}$ over the reals.

\subsection{Solution}

\pagebreak\section{Unsourced}
Prove that for reals $a,b\geq 1,$
\[\sqrt{a^2-1}+\sqrt{b^2-1}\leq ab.\]

\subsection{Solution}
Consider a triangle with two sides of length $a,b \geq 1$ such that the altitude from the vertex formed from the sides of length $a,b$ has length $1$. Label the triangle and its altitudes as shown:\\
(There should be a diagram here, but things are broken)
\begin{asy}
    size(5cm);
    pair A,B,C,D;
    A=(0,0);
    B=(0,8);
    C=(3,5);
    D=(0,5);
    draw(A--B--C--cycle);
    draw(C--D);
\end{asy}
% Diagram is not finished but doesn't even render for me wtf how does Asymptote

We note that $AD=\sqrt{b^2-1}$ and $BD=\sqrt{a^2-1}$ by the pythagorean theorem. Hence by the $\frac{1}{2}bh$ area formula, we have $2[ABC]=\sqrt{a^2-1}+\sqrt{b^2-1}$. Now let $\angle ACB=\theta$. By the sine area formula, we have $2[ABC]=ab\sin\theta$. As $\sin\theta\leq 1$, it follows that $ab\sin\theta\leq ab$, so we have $\sqrt{a^2-1}+\sqrt{b^2-1}\leq ab$ as desired.

\pagebreak\section{Unsourced}
What value of $x$ maximizes $(21+x)(1+x)(x-1)(21-x),$ if $x$ must be positive?

\subsection{Solution 1}
Note that this is the square of the area of a triangle with sides $20,22,2x$, by Heron's. From the sine area formula, we get that the area of the triangle is $220\sin\theta$, where $\theta$ is the measure of the angle between the sides of lengths $20$ and $22$. $\sin\theta$ attains its maximum value when $\theta=90^\circ$, where it is equal to $1$. In this case, we get from the Pythagorean Theorem that $2x=\sqrt{20^2=22^2} \implies x=\ansbold{\sqrt{221}}.$

\subsection{Solution 2}
Also possible to just use difference of squares and just do algebra.

\pagebreak\section{TrinMaC 2020/19}
Compute
\[\sum_{n=0}^{\infty}\cos^{-1}\left(\frac{\sqrt{n(n+1)(n+2)(n+3)}+1}{(n+1)(n+2)}\right).\]

\subsection{Solution}

\pagebreak\section{Unsourced}
Let $a,b,c,d$ be real numbers such that $a^2 - b^2 - c^2 + d^2 = ad + bc$ and $a^2 + b^2 - c^2 - d^2 = 0.$ Determine the value of $\frac{ab + cd}{ad + bc}.$

\subsection{Solution}
We note that the first condition rewrites as $a^2+d^2-2ad\cos 60^\circ=b^2+c^2+2bc\cos 120^\circ$, while the second rearranges as $a^2+b^2=c^2+d^2$. So $a,b,c,d$ are the side lengths of a cyclic quadrilateral with angles $60^\circ,120^\circ$ inscribed in a circle. WLOG $AB=a,BC=b,CD=c,DA=d$. Now the Pythagorean inequality combined with $a^2+b^2=c^2+d^2$ gives us $\angle ABC=\angle ADC=90^\circ$. So $\triangle ABC,\triangle ADC$ are $30-60-90$. WLOG setting $b=c=1$ then gives us $a=d=\sqrt{3}$, after which we can easily get the answer as $\ansbold{\frac{\sqrt{3}}{2}}$.

\pagebreak\section{AIME II 2006/15}
Given that $x, y,$ and $z$ are real numbers that satisfy:
\[x = \sqrt{y^2-\frac{1}{16}}+\sqrt{z^2-\frac{1}{16}}\]
\[y = \sqrt{z^2-\frac{1}{25}}+\sqrt{x^2-\frac{1}{25}}\]
\[z = \sqrt{x^2-\frac{1}{36}}+\sqrt{y^2-\frac{1}{36}}\]
and that $x+y+z = \frac{m}{\sqrt{n}},$ where $m$ and $n$ are positive integers and $n$ is not divisible by the square of any prime, find $m+n.$

\subsection{Solution}
The RHS looks suspiciously like the Pythagorean Theorem. After a bit of trial and error based on this observation, we realize that $x,y,z$ are the side lengths of a triangle with altitudes $\frac{1}{4},\frac{1}{5},\frac{1}{6}$ (the altitudes and the sides are ordered in the same way, so the altitude of length $\frac{1}{4}$ is perpendicular to the side of length $x$). Since the area is the same we have $\frac{x}{4}=\frac{y}{5}=\frac{z}{6}$. Let this quantity equal $k$, so $x=4k,y=5k,z=6k$. Then the area is $\frac{k}{2}$. On the other hand, Heron's gives us the area as $\frac{15k^2\sqrt{7}}{4}$. Setting these equal gives us $k=\frac{2}{15\sqrt{7}}$. Since $x+y+z=15k$ it follows that the desired quantity is $\frac{2}{\sqrt{7}} \implies \ansbold{9}$vv.

\pagebreak\section{Unsourced}
Consider sequence $a_n$ with $a_1=\sqrt{2}+1$ and $a_na_{n-1}^2+2a_{n-1}-a_n=0$ for $n\geq 2.$ Find $a_{1000}.$

\subsection{Solution}

\pagebreak\section{AIME 1991/15}
For positive integer $n$, define $S_n$ to be the minimum value of the sum\[ \sum_{k=1}^n \sqrt{(2k-1)^2+a_k^2}, \]where $a_1,a_2,\ldots,a_n$ are positive real numbers whose sum is 17. There is a unique positive integer $n$ for which $S_n$ is also an integer. Find this $n$.

\subsection{Solution}

\pagebreak\section{Unsourced}
If $x,y,z$ are positive numbers such that
    \[x^2+xy+\frac{1}{3}y^2=25\]
    \[\frac{1}{3}y^2+z^2=9\]
    \[z^2+zx+x^2=16,\]
    find $xy+2yz+3zx.$

\subsection{Solution}
We substitute $(a,b,c)=(x,\frac{y}{\sqrt{3}},z)$. The equations rewrite as:
$$a^2+ab\sqrt{3}+b^2=25$$
$$b^2+c^2=9$$
$$a^2+ac+c^2=16$$
We then use the implicit LoC trick to get that $\frac{1}{2}bc+\frac{1}{4}ab+\frac{\sqrt{3}}{4}ca=[ABC]$ where $\triangle ABC$ is a triangle with side lengths $3,4,5$. In this case, $[ABC]$ is simply $6$, so $$\frac{1}{2}bc+\frac{1}{4}ab+\frac{\sqrt{3}}{4}ca=6.$$
Substituting into $(x,y,z)$ gives us $$\frac{1}{4\sqrt{3}}xy+\frac{1}{2\sqrt{3}}yz+\frac{\sqrt{3}}{4}zx=6.$$ Multiplying by $4\sqrt{3}$ gives the desired quantity equal to $\ansbold{24\sqrt{3}}.$\\
Not completely sure this is right pls check!

\pagebreak\section{HMMT Feb. Algebra 2014/9}
Given $a$, $b$, and $c$ are complex numbers satisfying

\[ a^2+ab+b^2=1+i \]
\[ b^2+bc+c^2=-2 \]
\[ c^2+ca+a^2=1, \]

compute $(ab+bc+ca)^2$. (Here, $i=\sqrt{-1}$.)

\subsection{Solution}
The idea is to use LoC to show a more general statement for reals, which can be phrased as a polynomial identity and thus must hold in complex numbers as well! Will add more later.

\pagebreak\section{Unsourced}
Find all triples $(x,y,z)$ such that $xy+yz+zx=1$ and $5(x+\frac{1}{x})=12(y+\frac{1}{y})=13(z+\frac{1}{z}).$

\subsection{Solution}

\pagebreak\section{rd123/tworigami Mock AIME 2020/13}
If $a,b,c,d$ are positive real numbers such that
\begin{align*} ab + cd &= 90, \\ ad + bc &= 108, \\ ac + bd &= 120, \\ a^2+b^2 &= c^2+d^2, \end{align*}and $a+b+c+d=\sqrt{n}$ for some integer $n$, find $n$.

\subsection{Solution}
Consider a quadrilateral $ABCD$ with $AB=a,BC=b,CD=c,DA=a$, and $\angle B=\angle D=90^\circ$. Then from Pythagoras we have $a^2+b^2=c^2+d^2=AC$. Further since $\angle B+\angle D=180^\circ$ this quadrilateral is cyclic, so inscribe it in a circle. This also means that $\angle C=180^\circ-\angle A$. We know that
$$[ABCD]=[ABC]+[ADC]=\frac{ab+cd}{2}.$$
Since $ab+cd=90$ is given, we get $[ABCD]=45$. We can also write $[ABCD]=[ABD]+[CBD]$. Then by the sine area formula and using the fact that $\sin(180^\circ - \theta) = \sin \theta$, this is equal to
$$\frac{1}{2}\sin \angle A (ad+bc)=54 \sin \angle A.$$
But $[ABCD]=45$ as well, so $\sin \angle A=\frac{5}{6}$. Finally, we note that by Ptolemy's we have:
$$ac+bd=AC\cdot BD \implies AC \cdot BD=120.$$
Now, since the inscribed angle with measure $\theta$ of chord $\overline{BD}$ satisfies $\sin \theta = \frac{5}{6}$, it follows from LoS on either $\triangle BDA$ or $\triangle BDC$ that $BD=\frac{5}{6}AC$, since $\overline{AC}$ is a diameter and therefore $AC=2R$. This gives us:
$$\frac{5}{6}AC^2=120 \implies AC^2=144=a^2+b^2=c^2+d^2.$$
To finish, we consider the identity:
$$(a+b+c+d)^2=a^2+b^2+c^2+d^2+2(ab+ac+ad+bc+bd+cd)$$
Substituting $a^2+b^2=c^2+d^2=144$ as well as the values given at the start of the problem, we get $(a+b+c+d)^2=n=\ansbold{924}$.

\pagebreak\section{PUMaC Div. A Algebra 2018/6}
Let $a,b,c$ be nonzero reals such that $\frac{1}{abc}+\frac{1}{a}+\frac{1}{c}=\frac{1}{b}.$ The maximum possible value of $$\frac{4}{a^2+1}+\frac{4}{b^2+1}+\frac{7}{c^2+1}$$ is $\frac{m}{n}$ for relatively prime positive integers $m$ and $n.$ Find $m+n.$

\subsection{Solution}

\pagebreak\section{2018 Mock AIME, by TheUltimate123}
Let $a$,$b$,$c$,$d$ be positive real numbers such that \[195=a^2+b^2=c^2+d^2=\frac{13(ac+bd)^2}{13b^2-10bc+13c^2}=\frac{5(ad+bc)^2}{5a^2-8ac+5c^2}\] Then $a+b+c+d$ can be expressed in the form $m\sqrt{n}$, where $m$ and $n$ are positive integers and $n$ is not divisible by the square of any prime. Find $m+n$.

\subsection{Solution}

\pagebreak\section{Mildort AIME 3/15}
Let $\Omega$ denote the value of the sum

\[\sum\limits_{k=1}^{40} \cos^{-1}\left(\frac{k^2 + k + 1}{\sqrt{k^4 + 2k^3 + 3k^2 + 2k + 2}}\right).\]

The value of $\tan\left(\Omega\right)$ can be expressed as $\frac{m}{n}$, where $m$ and $n$ are relatively prime positive integers. Compute $m + n$.

\subsection{Solution}
We note that $\frac{k^2+k+1}{\sqrt{k^4+2k^3+3k^2+2k+2}}=\frac{k^2+k+1}{\sqrt{(k^2+k+1)^2+1}}$. Drawing out a right triangle quickly, it becomes clear that the summation is equivalent to:
$$\sum_{k=1}^{40} \arctan \left(\frac{1}{k^2+k+1} \right).$$
We would ideally like to make this sum telescope. Define a function $f$ such that:
$$\arctan \left(\frac{1}{k^2+k+1}\right)=\arctan \left(\frac{1}{f(k)}\right)-\arctan \left(\frac{1}{f(k+1)}\right).$$
Then the summation telescopes to $\arctan \left(\frac{1}{f(1)}\right)-\arctan \left(\frac{1}{f(41)}\right)$ which is hopefully easier to evaluate. Using arctangent addition, we have $\arctan \left(\frac{1}{x}\right)-\arctan \left(\frac{1}{y}\right)=\frac{y-x}{1+xy}$, so we need:
$$\frac{1}{k^2+k+1}=\frac{f(k+1)-f(k)}{f(k)f(k+1)+1}.$$
After looking at this for a while it becomes clear that $f(k)=k$ works (verifiable with substitution). So we just have to evaluate $\arctan(1)-\arctan \left(\frac{1}{41}\right)$. Using the arctangent addition formula again, we get that this is equal to $\arctan \left(\frac{20}{21}\right)$, so $\tan (\Omega)=\frac{20}{21}$ which yields an answer of $\ansbold{41}$.

\pagebreak\section{IMO 2001/6}
Let $a > b > c > d$ be positive integers and suppose that \[ ac + bd = (b+d+a-c)(b+d-a+c). \] Prove that $ab + cd$ is not prime.

\subsection{Solution}
Look at the problem for a few minutes and cry until you decide to give up and do another unit because Dennis made an IMO P6 required.

\end{document}
