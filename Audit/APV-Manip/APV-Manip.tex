\documentclass[mast]{lucky}


\title{Algebraic Manipulation}
\author{Dennis Chen, Jerry Xiao, William Dai}
\date{APV}

\begin{document}
\maketitle

Here we discuss some clever algebra manipulations. These basic tricks are fairly common in competition questions, and knowledge of these tricks generally reduces the time needed to solve algebra questions.

\section{Symmetry}
\subsection{Symmetric Systems}

The most basic of algebraic manipulations involves systems of equations. The gist is that when you are given $x\circ y,y\circ z,$ and $x\circ z$ (for some operation $\circ$), you can find $x \circ y \circ z$ to determine $x,y,$ and $z.$
\begin{exam}[2018 AMC 10B/4]
A three-dimensional rectangular box with dimensions $X$, $Y$, and $Z$ has faces whose surface areas are $24$, $24$, $48$, $48$, $72$, and $72$ square units. What is $X$ + $Y$ + $Z$?
\end{exam}
\begin{sol}
Manipulation does the trick better than just bashing out possible values. Note that the surface areas can be expressed as $XY,XZ,$ and $YZ.$ Therefore, we can let $XY = 24, XZ = 48, YZ = 72.$

A good way to solve for $X,Y,$ and $Z$ individually is by finding $XYZ$ and then dividing this quantity by $XY, XZ,$ and $YZ.$ Multiplying all equations together yields
\[XY \cdot XZ \cdot YZ = (XYZ)^2 = 24 \cdot 48 \cdot 72,\]
and taking the square root gives $XYZ = 288,$ so it follows that $X = \frac{XYZ}{YZ} = \frac{288}{72} = 4.$ Similarly, $Y = 6$ and $Z=12,$ so our answer is $4+6+12=\ansbold{22}.$
\end{sol}
\begin{exam}[Classic]
Find positive integers $a,b,$ and $c$ that satisfy
\begin{align*}
(a+b)(a+c)&=96(b+c) \\
(a+b)(b+c)&=54(a+c) \\
(a+c)(b+c)&=24(a+b).
\end{align*}
\end{exam}

\begin{sol}
The idea of this problem is similar to the previous example, in that there is a way to "manipulate" the product. Multiplying the equations gives
\[(a+b)^2(a+c)^2(b+c)^2 = 96 \cdot 54 \cdot 24(a+b)(a+c)(b+c)\]
\[(a+b)(a+c)(b+c)= 96 \cdot 54 \cdot 24\]
We can divide by $(a+b)(a+c)$ to obtain $b+c = \frac{96 \cdot 54 \cdot 24}{96} \cdot (b+c),$ or $b+c = 36,$ and then apply the same method with $a+b$ and $a+c.$ But we're not done, we still need to resolve:
\begin{align*}
b+c &= 36 \\
a+c &= 72\\
a+b &= 48
\end{align*}
Applying the manipulation trick again, we have: $2a +2b+2c = 156$, implying that $a+b+c = 78.$ Subtracting $a+b+c$ from $b+c,a+c,$ and $a+b$ leads to our desired answer: $(a,b,c)=\ansbold{(42,6,30)}.$
\end{sol}
\subsection{Symmetric and Cyclic Expressions}
We can sometimes use the fact that a sum is symmetric along with combinatorical thinking to quickly determine it. For example, in $(a+b+c)^3$, what is the coefficient of $a^2b$? Without expanding the whole expression out, we can see that we must choose $a$ twice and $b$ once. Hence, there are $\binom{3}{2}=3$ of that term. Additionally, it's clear terms such  as$b^2c$ and $a^2c$ must have the same coefficient by symmetry.

\begin{exam}[AMC 10A 2021/14]
All the roots of the polynomial $z^6-10z^5+Az^4+Bz^3+Cz^2+Dz+16$ are positive integers, possibly repeated. What is the value of $B$?
\end{exam}

\begin{sol}
Let the roots be $r_{1},r_{2}\ldots r_{6}$. By Vieta's, we have that $\sum r_{i}=10$ and $r_{1}r_{2}\ldots r_{6}=16$. Using guess and check, we get that $\{r_{1},r_{2}\ldots r_{6}\} = \{2,2,2,2,1,1\}$. Then, by Vieta's, $B=-\sum r_{i}r_{j}r_{k}$. We do casework on the number of $1$'s. If there are $0$ $1's$, then the term is $2^3=8$ and there are $\binom{4}{3}=4$ such terms. If there are $1$ $1's$, then the term is $2^2=4$ and there are $\binom{4}{2}\cdot \binom{2}{1}=12$ such terms. If there are $2$ $1's$, then the term is $2^1=2$ and there are $\binom{4}{1}\cdot \binom{2}{2}=4$ such terms.

Hence, the sum is $8\cdot 4+4\cdot 12+2\cdot 4=32+48+8=\ansbold{88}$.
\end{sol}
\section{Identities}
An identity is an equation that is always true. We often use identities in our algebraic manipulations.
\subsection{Special Functions}
We will give a very brief overview of the absolute value, floor, ceiling and fractional part functions. This handout will not go in depth but we'll give some brief identities with these functions.
\subsubsection{Absolute Value}
\begin{defi}[Absolute Value]
The absolute value of a number is defined as
\[|x|=\begin{cases}
x\text{ if }x\geq 0 \\
-x\text{ if }x<0.
\end{cases}\]
\end{defi}
Another definition for the absolute value of a number is its distance from $0.$

\begin{theo}[Square Root Representation]
For all $a$, $|a|=\sqrt{a^2}$
\end{theo}

\begin{theo}[Symmetric]
For all $a$, $|a|=|-a|$
\end{theo}

We often use this property while graphing to only focus on a certain part, say above the $x$-axis, since the other parts are just reflections of that part.
\subsubsection{Floor, Ceiling, Fractional Part}
In short, the floor and ceiling function round down and up to the nearest integer, respectively, and the fractional part does not have a unique and universal mathematical definition.

\begin{defi}[Floor]
The floor of $x,$ denoted as $\lfloor x\rfloor,$ is the unique integer that satisfies
\[\lfloor x\rfloor\leq x<\lfloor x\rfloor+1.\]
\end{defi}
You can also think of it as the greatest integer that does not exceed $x.$

\begin{defi}[Ceiling]
The floor of $x,$ denoted as $\lceil x\rceil,$ is the unique integer that satisfies
\[\lceil x\rceil-1< x\leq \lceil x\rceil.\]
\end{defi}
You can also think of it as the smallest integer that is not less than $x.$

\begin{defi}[Fractional Part]
The fractional part of a real number is defined as $x - \lfloor x \rfloor$ and denoted as $\{x\}.$
\end{defi}

You can think of it as the part after the decimal \textbf{for positive numbers} if it helps.

\begin{theo}[Summation Representation]
If $x\ge 0$, then $\lfloor x \rfloor = \sum_{1\leq i < x} 1$.
\end{theo}

\begin{theo}[Counting Multiples]
$\lfloor \frac{k}{n} \rfloor$ is the number of positive multiples of $n$ less than $k$.
\end{theo}
These often appear in counting problems.

\begin{theo}[Adding Integers]
$\lfloor k+n\rfloor = \lfloor k \rfloor + n$ for real $k$ and integer $n$
\end{theo}
\subsection{Polynomial Identities}
\subsubsection{Factoring}
\begin{theo}[Difference Of Squares]
For any $x,y$, we have $x^2-y^2=(x-y)(x+y)$.
\end{theo}
The identity may look pretty simple but it's often used in many problems as an intermediate step. It also commonly appears in number theory problems as a way to find the prime factorization of some number.

It's also a very handy tool for doing computations.

\begin{exam}[ARML Local Individual 2020/1]
There are four distinct prime factors of $999039$. Compute the sum of these four prime factors.
\end{exam}

\begin{sol}The first step here is to notice that $999039$ is a very large number. If a contest problem asks you to explicitly factorize some large number, then it likely has a simple trick. Now, note that $999039$ is very close to $1000000=1000^2$. Because of this, we might think of difference of squares. Checking, we see that the difference between $1000^2$ and $999039$ is $961=31^2$. 

Applying Difference of Squares, $999039=1000^2-31^2=1031\cdot 969$. Now, $969$ is obviously a multiple of $3$ because its digits sum to $24$. Then, $969=323\cdot 3$. If you memorized your squares, note that $323$ is very close to $324=18^2$. We can apply Difference of Squares again! So, $323=18^2-1=17\cdot 19$. We're given that there are $4$ distinct prime factors and we already found $3$: $3,17,19$. This implies $1031$ is the last prime factor.

We sum all of these up to get $1031+3+17+19={1060}$.
\end{sol}

\begin{theo}[Difference of Squares Corollary]
For any $a,b$, we have $a\cdot b = (\frac{a+b}{2})^2-(\frac{a-b}{2})^2$.
\end{theo}
This is somewhat the inverse of Difference of Squares where you go from a product to the difference of two squares. This is very useful in number theory as you can show that some product of integers $ab$ is the difference of two perfect squares very easily using this by just showing that $a+b$ is even (because $a+b$ is even implies $a-b$ is even).

\begin{exam}[MATHCOUNTS Chapter Sprint 2017/26]
Find the value of $x$ that makes
$$55\times 59-53\times 57=x^2-1$$
true.
\end{exam}
\begin{walk}
\begin{enumerate}
    \item How can you rewrite $55\times 59$ and $53\times 57$ using Difference Of Squares Corollary?
    \item Note that the $2^2$ cancels. What can you do to $57^2-55^2$ to easily compute it without multiplying out $57^2$ and $55^2$?
    \item Now, $x^2-1=\text{something}$. We're done!
\end{enumerate}
\end{walk}

\begin{theo}[Simon's Favorite Factoring Trick]
For any $x,y,a,b$, we have $xy+ax+by=(x+b)(y+a)-ab$.
\end{theo}

Also known as completing the rectangle.

While this identity is true for all $x,y,a,b$, it's mainly useful in number theory problems where you want to count the number of integer $(x,y)$ satisfy some equation or find what kind of $(x,y)$ that satisfy some equation. That might seem vague, but it'll become clear in the following examples. 

\begin{exam}[AMC 10A 2015/20, fixed]
A rectangle with positive integer side lengths in $\text{cm}$ has area $A$ $\text{cm}^2$ and perimeter $P$ $\text{cm}$. What is the least integer greater than or equal to $100$ that can't be written as $A+P$?
\end{exam}
\begin{walk}
\begin{enumerate}
    \item If you let $l,h$ be the length and height of the rectangle, what are $A$ and $P$?
    \item How can you rewrite $A+P$ using SFFT?
    \item If $A+P=k$ for some positive integer $k$, how would you find $l,h$? 
    \item Then, for what $k$ is it impossible for $A+P=k$ for any $l$ or $h$? That is, what $k$ will make the process for finding $l,h$ you discovered before fail?
    \item Now, we just find the smallest $k$ larger than or equal to $100$ with that property to get the answer. We're done!
\end{enumerate}
\end{walk}

\begin{theo}[Difference of Powers]
For all $a,b$ and positive integer power $n$, $a^{n}-b^{n}=(a-b)(a^{n-1}+a^{n-2}b\ldots + b^{n-1})$.
\end{theo}
Note that this can applied subtly; a sixth power is also a second power and a third power. So, you can both factor $a^6-b^6$ as not only $(a-b)(a^5+a^4b\ldots + b^5)$ but also as $(a^3-b^3)(a^3+b^3)$ and $(a^2-b^2)(a^4+a^2b^2+b^4)$.

\begin{theo}[Sum of Powers]
For all $a,b$ and $\textbf{odd}$ positive integer power $n$, $a^n+b^n=(a+b)(a^{n-1}-a^{n-2}b+a^{n-3}b^2\ldots +b^{n-1})$.
\end{theo}

This can similarily be applied subtly as with difference of powers.

\subsubsection{Symmetric Polynomial Identities}
These are some helpful polynomial identities when dealing with symmetric expressions.
\begin{theo}[Square of Sum]
$(\sum_{i=1}^{n} a_{i})^2 = \sum_{i=1}^{n} a_{i}^2 + (\sum_{1\leq i < j\leq n } a_{i}a_{j})$
\end{theo}
This is a common way to get the sum of the product of each unordered pair of elements. 
\begin{theo}[Titu's Identity]
$a^3+b^3+c^3-3abc=(a+b+c)(a^2+b^2+c^2-ab-ac-bc)$
\end{theo}

\subsection{Fractions}
There's no ``official'' name for this manipulation, but it is often very useful in manipulating fractions with ugly denominators and numerators. 
\begin{theo}[Fraction Trick]
If $\frac{a}{b}=\frac{c}{d}$, then
\[\frac{a}{b}=\frac{c}{d}=\frac{a+c}{b+d}.\]
\end{theo}

\begin{pro}
Let $\frac{a}{b}=\frac{c}{d}=k$. Then $\frac{a+c}{b+d}=\frac{bk+dk}{b+d}=k$.
\end{pro}

This may seem obvious but a large amount of people often forget that this exists when it comes up.

A similarly ``obvious'' trick is flipping the fraction when the denominator is complicated and the numerator is simple.

\begin{exam}[Classic]
If $\frac{x}{x^2+x+1}=\frac{1}{3}$, find $x$.
\end{exam}

\begin{sol}
We reciprocate both sides to get $\frac{x^2+x+1}{x}=3$. Then, $x+1+\frac{1}{x}=3$ which implies $x-2+\frac{1}{x}=0$. We multiply by $x$ to get $x^2-2x+1=(x-1)^2=0$, so $x=\ansbold{1}$.
\end{sol}
\section{Substitution}

\subsection{Numbers As Variables}
Like the title implies, we treat certain numbers in the problems as variables. Then, we can apply more general algebraic identities or manipulate them easier.
\begin{exam}[SMT 2021/N5]
There are exactly four distinct positive integers $n$ for which $15380 - n^2$ is a perfect square. Noting that $13^2 + 37^2 = 1538$, compute the sum of the four possible values of $n$.
\end{exam}

This problem is quite difficult compared to the rest of the handout. Even then, I believe the problem is worth presenting because it's such a good example of how numbers can obscure the intent of an algebraic problem, and how knowing that these numbers are used solely to obscure can give you the courage to push through and find the hidden identity.

\begin{sol}
The solution hinges on the fact that
\[(a^2+b^2)(c^2+d^2)=(ac+bd)^2+(ad-bc)^2.\]
Note that
\[15380=(1^2+3^2)(13^2+37^2)=(13+111)^2+(36-27)^2=124^2+9^2\]
and
\[15380=(1^2+3^2)(37^2+13^2)=(37+39)^2+(13-74)^2=76^2+61^2.\]
Since we know there are only four values of $n,$ the answer is
\[124+9+76+61=\ansbold{270}.\]
\end{sol}

This $(a^2+b^2)(c^2+d^2)=(ac+bd)^2+(ad-bc)^2$ identity is known as the Brahmagupta-Fibonacci Identity. It is a quite popular identity used in non-AMC style contests, such as ARML, college contests, etc.

\begin{exam}[2017-2018 Mandelbrot]
Let us say that a decade is \emph{primeval} if it contains four prime numbers. For instance, the decade from $1480$ to $1490$ was primeval, since $1481, 1483, 1487$ and $1489$ are all primes. Let $p_1,p _2, p_3$ and $p_4$ be the primes (in order) in the  next primeval decade after $2020.$ Compute the value of $p_2 p_3-p_1 p_4$.
\end{exam}

\begin{sol}
Finding the next primeval decade is too time consuming! We instead use two observations:
\begin{itemize}
\item Prime numbers greater than $5$ must end in $1,3,7,$ or $9.$
\item If the oldest year of a primeval decade is $x,$ then the four prime years are $x+1,x+3,x+7,$ and $x+9.$
\end{itemize}
Now the problem becomes standard algebra fare; our answer is $(x+3)(x+7)-(x+1)(x+9) = 12.$ We avoided unnecessary computation here by treating large numbers as variables instead.
\end{sol}

\begin{exam}[Math Prizes For Girls 2015]{6}
Let $S$ be the sum of all distinct real solutions of the equation 
\[\sqrt{x + 2015} = x^2 - 2015.\]
Compute $\lfloor 1/S \rfloor$.  Recall that if $r$ is a real number, then $\lfloor r \rfloor$ (the floor of $r$) is the greatest integer that is less than or equal to $r$.
\end{exam}

\begin{sol}
Let $2015=y$. Then, we have $\sqrt{x+y}=x^2-y$, which implies that $x+y=x^4-2x^2y+y^2\implies y^2+(-2x^2-1)y+x^4-x=0$. Thus $y=\frac{2x^2+1 \pm (2x+1)}{2}$. Now, we have $2015=x^2+x+1$ or $2015=x^2-x$. These give $x=\frac{-1 \pm \sqrt{8057}}{2}$ and $x=\frac{1 \pm \sqrt{8061}}{2}$. Now, note that we have
\[x+y\ge 0 \implies x \ge -2015~~\text{and}~~x^2-y\ge 0\implies |x|\ge \sqrt{2015}.\]
We can see that $\frac{-1 + \sqrt{8057}}{2}>\sqrt{2015}$ and that $\frac{-1-\sqrt{8057}}{2} < -\sqrt{2015}$. Also, $\frac{1-\sqrt{8061}}{2} > - \sqrt{2015}$ and $\frac{1+\sqrt{8061}}{2} > \sqrt{2015}$. So our only two solutions are $\frac{-1-\sqrt{8057}}{2}$ and $\frac{1+\sqrt{8061}}{2}$.

Then $\frac{1}{S}=\frac{2}{\sqrt{8061}-\sqrt{8057}}=\frac{\sqrt{8061}+\sqrt{8057}}{2}$. Since $89 < \frac{1}{S} < 90$ so our answer is $\ansbold{89}$.
\end{sol}


\subsection{Expressions As Variables}
Basically, we let expressions be variables and represent the given equations in terms of these variables.

All of the following involve the very common substitution of letting $x+y=a$ and $xy=b$ and rewriting the terms you know in terms of these expressions. The upshot of this is that we can represent $x^{n}+y^{n}$ in terms of $a$ and $b$. The most common identity you'll use here is $x^2+y^2=a^2-2b$.

\begin{exam}[HMMT February 2013]
Let $x$ and $y$ be real numbers with $x > y$ such that $x^2 y^2 + x^2 + y^2 + 2xy = 40$ and $xy + x + y = 8$. Find
the value of $x$.
\end{exam}
\begin{sol}
Let $x+y=a$ and $xy=b$. Then, $b^2+a^2=40$ and $b+a=8$. Then, $(8-a)^2+a^2=40$ implies $2a^2-16a+64=40$. We solve this to get $a=2,6$. Then $\{x+y,xy\}=\{2,6\}$. Note $x+y=2$ is impossible by AM-GM. So, $x+y=6$ and $xy=2$. Then, solving $x(6-x)=2$, we have $x=\ansbold{\frac{1+2\sqrt{7}}{2}}$.
\end{sol}


\begin{exam}[HMMT February 2013]
Let $a$ and $b$ be real numbers such that $\frac{ab}{a^2+b^2}=\frac{1}{4}$. Find all possible values of $\frac{|a^2-b^2|}{a^2+b^2}$.
\end{exam}

\begin{sol}
Let $a+b=x$ and $ab=y$. Then, $\frac{y}{x^2-2y}=\frac{1}{4}$. Multiplying, we get $x=\sqrt{6y}$. Then, $\frac{|a^2-b^2|}{a^2+b^2}=\frac{|a-b||a+b|}{a^2+b^2}=\frac{|a-b||x|}{x^2-2y}=\frac{\sqrt{x^2-4y}|x|}{x^2-2y}$. Substituting in,$\frac{\sqrt{2y}\cdot \sqrt{6y}}{4y}=\ansbold{\frac{\sqrt{3}}{2}}$
\end{sol}


\pagebreak\section{Problems}

%about 2/3 of pts
\minpt{60}

\psetquote{It’s hard to do a really good job on anything you don’t think about in the shower.}{Paul Graham}

\begin{prob}[AMC 10A 2018/14]{2}
What is the greatest integer less than or equal to\[\frac{3^{100}+2^{100}}{3^{96}+2^{96}}?\]
\end{prob}

\begin{prob}[HMMT November 2012/1]{2}
What is the sum of all of the distinct prime factors of $25^3 - 27^2$?
\end{prob}

\begin{prob}[Classical]{2}
If $x+\frac{1}{x} = 3$, find $x^2+\frac{1}{x^2}$.
\end{prob}

\begin{prob}[AHMSE 1993/19]{2}
How many ordered pairs $(m,n)$ of positive integers are solutions to
\[\frac{4}{m}+\frac{2}{n}=1?\]
\end{prob}

\begin{prob}[1989 AIME/1]{2}
Compute $\sqrt{(31)(30)(29)(28)+1}$.
\end{prob}

\begin{prob}[2008-2009 Mandelbrot]{3}
Austin currently owns some shirts, pants, and pairs of shoes; he chooses one of each to create an outfit. If he were to obtain one more shirt, his total number of outfits would increase by $48.$ Similarly, if he bought another pair of pants he would have $90$ more outfits, while an extra pair of shoes would result in $120$ more outfits. How many outfits can he currently create?
\end{prob}
  
\begin{prob}[2018 AMC 10A/10]{3}
Suppose that real number $x$ satisfies \[\sqrt{49-x^2}-\sqrt{25-x^2}=3.\] What is the value of $\sqrt{49-x^2}+\sqrt{25-x^2}$?
\end{prob}

\begin{prob}[AMC 10A 2015/15]{3}
Consider the set of all fractions $\frac{x}{y}$, where $x$ and $y$ are relatively prime positive integers. How many of these fractions have the property that if both numerator and denominator are increased by $1$, the value of the fraction is increased by $10\%$?
\end{prob}

\begin{prob}[2018 BmMT]{4}
Let $x$ be a positive real number so that $x - \frac{1}{x} = 1$.  Compute  $x^8 - \frac{1}{x^8}$.
\end{prob}

\begin{prob}[1983 AIME/3]{4}
What is the product of the real roots of the equation \[x^2 + 18x + 30 = 2 \sqrt{x^2 + 18x + 45}?\]
\end{prob}

\begin{prob}[OMO Fall 2014/12]{4}
Let $a$, $b$, $c$ be positive real numbers for which
\[
  \frac{5}{a} = b+c, \quad
  \frac{10}{b} = c+a, \quad \text{and} \quad
  \frac{13}{c} = a+b.
\]
If $a+b+c = \frac mn$ for relatively prime positive integers $m$ and $n$, compute $m+n$.
\end{prob}
  
\begin{prob}[AMC 10B 2011/19]{4}
What is the product of all the roots of the equation\[\sqrt{5 | x | + 8} = \sqrt{x^2 - 16}?\]
\end{prob}

\begin{prob}[AMC 10A 2018/12]{5}
How many ordered pairs of real numbers $(x,y)$ satisfy the following system of equations?\[x+3y=3\]\[\big||x|-|y|\big|=1\]
\end{prob}

\begin{prob}[1985 AIME/7]{5}
Assume that $a$, $b$, $c$ and $d$ are positive integers such that $a^5 = b^4$, $c^3 = d^2$ and $c - a = 19$. Determine $d - b$.
\end{prob}  
%757

\begin{prob}[JMC 10 2021/22]{5}
Let $r_1,r_2,r_3,r_4$ be the roots of $P(x)= x^4+4x^3-3x^2+2x-1.$ Suppose $Q(x)$ is the monic polynomial with all six roots in the form $r_{i}+r_{j}$ for integers $1\le i < j \le 4.$ What is the coefficient of the $x^4$ term in the polynomial $Q(x)?$
\end{prob}


\begin{prob}[2019 hARMLess Mock ARML/4]{5}
Fran has two congruent rectangular prisms, each with volume $V$. There are three distinct ways for Fran to glue the two prisms together along congruent faces to form a larger rectangular prism. The three prisms that can be created in this way have surface areas $20,18,$ and $17$. Compute $V$.
\end{prob}

\begin{prob}[NEMO 2017]{5}
The value of the expression
\[\sqrt{1+\sqrt{\sqrt[3]{32}-\sqrt[3]{16}}} + \sqrt{1-\sqrt{\sqrt[3]{32}-\sqrt[3]{16}}}\]
can be written as $\sqrt[m]{n}$, where $m$ and $n$ are positive integers. Compute the smallest possible value of
$m + n$.
\end{prob}

\begin{prob}[vvluo]{6}
Determine all solutions, if any exist, to the system


\[\frac{1}{xy}=\frac{x}{z}+1\]
\[\frac{1}{yz}=\frac{y}{x}+1 \]
\[\frac{1}{zx}=\frac{z}{y}+1.\]
\end{prob}


\begin{prob}[BMT 2020]{6}
Let $a,b$ and $c$ be real numbers such that $a+b+c=\frac{1}{a}+\frac{1}{b}+\frac{1}{c}$ and $abc=5$. The value of
\[(a-\frac{1}{b})^3+(b-\frac{1}{c})^3+(c-\frac{1}{a})^3\]
can be written in the form $\frac{m}{n}$, where $m$ and $n$ are relatively prime positive integeres. Compute $m+n$.
\end{prob}

\begin{prob}[NICE Spring 2021/13]{6}
Suppose $x$ and $y$ are nonzero real numbers satisfying the system of equation.
   \[3x^2 + y^2 = 13x,\]
   \[ x^2 + 3y^2 = 14y.\]
Find $x+y$.
\end{prob}


\begin{prob}[HMMT February 2013]{9}
Let $x,y$ be complex numbers such that $\frac{x^2+y^2}{x+y} = 4$ and $\frac{x^4+y^4}{x^3+y^3} = 2$. Find all possible values of 
$\frac{x^6+y^6}{x^5+y^5}$.
\end{prob}

\begin{req}[JMC 10 2021/18]{9}
If $x,y,$ and $z$ are positive real numbers that satisfy the equation
\[xy+yz+zx=96(y+z)-x^2 = 24(x+z) -y^2 = 54(x+y) -z^2,\]
what is the value of $xyz$?
\end{req}

\begin{prob}[CARML 2019/8]{13}
There exist unique positive integers $1<a<b$ satisfying
\[a^2+b^2=2^{20}+1.\]
Compute $a+b$.
\end{prob}
\end{document}