\documentclass{article}
\usepackage[mast]{dennis}
\newcommand{\skipline}{\vspace{2mm}}

\title{Induction}
\author{Brian Zhang}
\date{CRU}


\begin{document}

\maketitle

\section{Theory}
\subsection{Motivation}
Let's say you have an infinitely long line of dominoes, and you want to topple them all.  If you want to knock them over, then you just need to tap the first one, and they will all fall down, right? Wrong. You also need to make sure that if the fist domino is knocked down, the second one will also start to fall. You also need to make sure the second domino falling down implies the third one falling down, the third one falling implies the fourth one falling, and so on. Thus, in order to rigorously prove you can knock all of these dominoes down, you need to prove that you can knock down the first domino, and if the $n$th domino is knocked down, the $n+1$th domino is also knocked down. This is the main idea behind induction.
\subsection{Basics}
Let's say we have a statement $P(n)$ which we want to prove it true for all nonnegative integers $n$. A good way to prove this to be true is by induction. For induction, you need to do the following things:
\begin{enumerate}
    \item Show that $P(0)$ is true. This is called the base case. You can think of this step as knocking the first domino down.
    \item Prove that if $P(n)$ is true, then $P(n+1)$ is also true. This step is called the inductive step, and this is like showing that if you knock one domino down, the next will also fall.
\end{enumerate}
This allows us to prove a statement to be true for all positive integers.
\begin{exam}[Principle of Weak Induction]
Prove the principle of induction. That is, if you know some proposition $P(0)$ is true and $P(n)$ being true implies $P(n+1)$ being true, then $P(n)$ is true for all nonnegative integers $n$.
\end{exam}
\skipline
\noindent
There are also several variations on the principle of induction, as follows:
\begin{itemize}
    \item \textbf{Starting at a number other than 0:} Instead of proving $P(0)$ to be true, you prove $P(i)$ to be true for some integer $i$.
    \item \textbf{Proving for $P(n+k)$ instead of $P(n+1)$:} If $P(n)$ does not necessarily imply that $P(n+1)$ is true, you can instead prove that $P(n)$ implies $P(n+k)$ to be true, and check that the base cases $P(1), P(2), \dots P(k)$. You can also use this to prove a statement to be true for all odd numbers, or all multiples of a certain number.
    \item \textbf{Strong induction:} Instead of assuming only $P(n)$ to be true, you can assume $P(1), P(2), \dots P(n)$ to be true. This is called \textit{strong induction}, as the induction hypothesis is stronger. Example 2.4 is a good example of this.
\end{itemize}

\subsection{Things to be careful about}
When doing induction, you need to make sure that you actually prove both the base case and the inductive step.
\begin{itemize}
    \item \textbf{Not proving the base case:} Consider the function $f(x)=2x+1$. If we know that $f(n)$ is an even number, then it is easy to show that $f(n+1)$ is also an even number. This clearly doesn't make sense, as $f(x)=2x+1$ can never be even! This is because you did not prove the base case. Even if knocking down one domino means that the next one will fall, if you never start knocking the dominoes down they will never fall! You need to make sure you check the base case. For this example, proving the base case is impossible, so the statement is false.
    \item \textbf{Not proving the inductive step:} Lets say that we wanted to show that $n^2-n+41$ is prime for all positive integers $n$. Upon inspection, we find that it is in fact prime for $n=1, 2, \dots 39$, so this is our base case, so we are done, right? Again, wrong. You need to make sure you do the inductive step, or otherwise it won't necessarily be true for all positive integers. There is this strategy that is useful in computational contests called \textit{engineers induction}, which is where you find a pattern that holds for some small numbers, and assume it is true for all numbers. However, this is not an accepted strategy on olympiads, and I would recommend against using it on computational contests.
\end{itemize}

\section{Examples}
Let's start with a basic example introducing the idea of induction. If you feel like you already have an understanding of induction, then feel free to skip this example.
\begin{exam}[Classic]
Show that $1+2+\dots+n$ is equal to $\frac{n(n+1)}{2}$.
\end{exam}
\begin{sol}
We have a statement, and we want to prove it for all positive integers $n$. This motivates us to think of induction. We want to choose the variable to induct on. In this problem, it's kind of obvious, as there is only 1 variable ($n$). In more involved problems, you may need to be more careful on what to induct on. Anyways, lets get started on the induction.

Lets do the base case first. The base case is just when $n=1$, but this is trivial as $1=\frac{1(1+1)}{2}$. Usually, the base case is a lot easier than the actual inductive step, but this isn't always the case (See exercise 5.4.).

Now, onto the inductive step. We assume that the given statement is true for $n$, or $1+2+\dots+n$ = $\frac{n(n+1)}{2}$. Since we want to show that the given statement is true for $n+1$, so we want to prove that $1+2+\dots+n+(n+1)=\frac{((n)+1)((n+1)+1)}{2}$, or $1+2+\dots+n+n+1=\frac{(n+1)(n+2)}{2}$.

Lets start with something that we already know is true, or $1+2+\dots+n$ = $\frac{n(n+1)}{2}$. From this, we want to somehow get $1+2+\dots+n+(n+1)=$something. But we can just add $n+1$ to both sides of $1+2+\dots+n$ = $\frac{n(n+1)}{2}$ to get
\begin{align*}
    1+2+\dots+n+(n+1)&=\frac{n(n+1)}{2}+(n+1)\\
    &=\frac{n(n+1)+2(n+1)}{2}\\
    &=\frac{(n+1)(n+2)}{2}
\end{align*}
as desired.

Thus, we have proved the inductive step, so we are done by the principle of induction.
\end{sol}

\begin{exam}[Fermat's Little theorem]
Prove that a prime $p$ divides $n^{p}-n$ for all positive integers $n$.
\end{exam}

\begin{sol}
We will need the following lemma.
\begin{lemma}
$p$ divides $\binom{p}{k}$ for all integers $k$ such that $0\leq k \leq p$.
\end{lemma}
\begin{proof}
Note that $\binom{p}{k}=\frac{p!}{k!(p-k)!}$. However, the denominator has no factors of $p$ since $p>k, p-k$. But the denominator has a single factor of $p$, so $p|\binom{p}{k}$, as desired.
\end{proof}

We induct on $n$.

The base case is trivial, as $1^{p}-1=0$, which is obviously divisible by $p$.

Assume that $p$ divides $n^{p}-n$ for some number $n$. We wish to show that $p$ divides $(n+1)^{p}-(n+1)$. Note that by the binomial theorem,
\[(n+1)^{p}=\binom{p}{0}n^p+\binom{p}{1}n^{p-1}+\dots+\binom{p}{p-1}n+\binom{p}{p}\equiv n^p+1\pmod{p}.\]
Thus, $(n+1)^{p}-(n+1)\equiv n^p+1-(n+1)\equiv n^p-n\equiv 0\pmod{p}$, as desired.
\end{sol}

Here is an example showcasing the idea of Strong Induction.
\begin{exam}[JMO 2011/4]
A word is defined as any finite string of letters. A word is a palindrome if it reads the same backwards and forwards. Let a sequence of words $W_0, W_1, W_2,...$ be defined as follows: $W_0 = a, W_1 = b$, and for $n \ge 2$, $W_n$ is the word formed by writing $W_{n-2}$ followed by $W_{n-1}$. Prove that for any $n \ge 1$, the word formed by writing $W_1, W_2, W_3,..., W_n$ in succession is a palindrome.
\end{exam}

\begin{sol}
We use strong induction. The base case is easy to see, since $W_1+W_2=bab$.

Now, assume that $W_1+\dots+W_{k}$ is a palidrome for all $k\leq n$, and we WTS $W_1+\dots+W_{n+1}$ is a palindrome.

Let $f(W)$ to be the string obtained by reversing $W$. Now, notice that
\begin{align*}
        f(W_1+W_2+\dots+W_{n+1}) &= f(W_{n+1})+f(W_n)+\dots+f(W_1) \\
        &= f(W_{n+1}) + W_1+\dots+W_n \\
        &= f(W_n)+f(W_{n-1}) + W_1 + \dots + W_{n-2} + W_{n-1} + W_{n}
\end{align*}
The last line is clearly a palindrome, so we are done.
\end{sol}

\begin{exam}[JMO 2018/1]
For each positive integer $n$, find the number of $n$-digit positive integers that satisfy both of the following conditions:\begin{itemize}\item no two consecutive digits are equal, and\item the last digit is a prime.\end{itemize}
\end{exam}

\begin{sol}
The answer is $\ansbold{\frac{2}{5}(9^n-(-1)^n)}$.

Let a number be \textit{good} if it has no two consecutive digits equal and the last digit is prime. Let $f(n)$ be the number ways to make a $n$-digit number that is \textit{good}. We define the $i$th digit to be the digit that is in the $10^{i-1}$ position.

Note that $f(1)=4$, since there is just 4 ways to choose the prime digit. For $f(2)=32$, there are 4 ways to choose the 1st digit, and for the 2nd digit there are 8 choices.

We wish to find a recursive sequence for the number of $n$-digit positive integers. There are two cases for a $n$-digit number, as follows:
\begin{itemize}
    \item \textbf{The $n-1$th digit is a 0} \\
     Then, the number of way to make this is number is just taking a $n-2$ digit \textit{good} number, and then adding a 0 as the $n-1$th digit, and any nonzero number as the $n$th digit. Then, there are $9f(n-2)$ ways for this case.
    \item \textbf{The $n-1$th digit is nonzero} \\
    We can take a $n-1$ digit good number, and then add a nonzero digit that is not equal to the $n-1$th digit. There are 8 ways to choose this digit, so there are $8f(n-1)$ ways for this case.
\end{itemize}
Then, the total amount of ways to make a $n$ digit \textit{good} number is $f(n)=8f(n-1)+9f(n-2)$. Solving this recurrence, we have
\[n^2=8n+9 \Rightarrow n^2-8n-9=0 \Rightarrow n=9, -1\]
Now, we let $f(n)=a(9)^n+b(-1)^n$. Using $f(1)=4=9a-b$ and $f(2)=32=81a+b$, we get $a=\frac{2}{5}$ and $b=-\frac{2}{5}$, so $f(n)=\frac{2}{5}(9^n-(-1)^n)$, as desired.
\end{sol}

\begin{exam}[Germany TST 2004]
In a finite simple graph $G$, all vertices can be black or red. In a move, you can pick a vertex $v$ and toggle the colors of $v$ and its neighbors. Initially, all vertices are black; prove that it's possible to make all vertices red.
\end{exam}

\begin{sol}
We induct on $n$. The cases $n=1$, $n=2$ are obvious.

Now, assume that its possible to toggle everything for all graphs of size $k\leq n$, and we want to show that its possible for a graph of size $n+1$. Now, we let function $f(i)$ be the moves for which we have a vertex $v_i$ and we ``exclude" it, and perform the moves that will toggle everything but $v_i$. If $f(i)$ also toggles $v_i$, then we're done. Thus, we can assume all $f(i)$ does not toggle $v_i$.

If $n+1$ is even, then by applying $f(1),f(2),\dots,f(n+1)$ then we can toggle everything. If $n+1$ is odd, then we can choose some vertex $v$ with some even degree (which must exist since the sum of the degrees is even and we have an odd number of vertices), so we can consider the set that does not include any of it or its neighbors. Now, this new smaller set must have an even number of vertices, so we can use the earlier trick to turn everything in there red also, and then toggle $v$.

\end{sol}

\newpage
\section{Computational Problems}
Engineer's induction is \textbf{not} allowed.

\skipline

\begin{prob}[Classic]{1}
Find the number of ways to tile a $2\times n$ rectangle with dominoes. A domino is a $1\times 2$ or a $2 \times 1$ rectangle.
\end{prob}

\begin{prob}[AIME 2008]{4}
Consider sequences that consist entirely of A's and B's and that have the property that every run of consecutive A's has even length, and every run of consecutive B's has odd length. Examples of such sequences are AA, B, and AABAA, while BBAB is not such a sequence. How many such sequences have length 14?
\end{prob}

\begin{prob}[AIME 2006/11]{2}
A collection of 8 cubes consists of one cube with edge-length $k$ for each integer $k,\thinspace 1 \le k \le 8.$  A tower is to be built using all 8 cubes according to the rules:\\[1\baselineskip]$\bullet$ Any cube may be the bottom cube in the tower.\\[0\baselineskip]$\bullet$ The cube immediately on top of a cube with edge-length $k$ must have edge-length at most $k+2.$\\[1\baselineskip]Let $T$ be the number of different towers than can be constructed.  What is the remainder when $T$ is divided by 1000?
\end{prob}

\begin{prob}[BMT 2018]{3}
Suppose there are 2017 spies, each with $\frac{1}{2017}$th of a secret code. They communicate by telephone; when two of them talk, they share all information they know with each other. What is the minimum number of telephone calls that are needed for all 2017 people to know all parts of the
code?
\end{prob}

\begin{prob}[NICE Spring Computational 2021/11]{3}
Fifty rooms of a castle are lined in a row.  The first room contains $100$ knights, while the remaining $49$ rooms contain one knight each.  These knights wish to escape the castle by breaking the barriers between consecutive rooms, ending with the barrier from room $50$ to the outside.

\noindent At the stroke of midnight, each knight in the $i^{\text{th}}$ room begins breaking the barrier between the $i^{\text{th}}$ and $(i+1)^{\text{st}}$ rooms, where we count the $51^{\text{st}}$ room as the exterior.  Each person works at a constant rate and is able to break down a barrier in 1 hour, and once a group of knights breaks down the $i^{\text{th}}$ barrier, they immediately join the knight breaking down the $(i+1)^{\text{st}}$ barrier.

\noindent The number of hours it takes for the knights to escape the castle is $\tfrac mn$, where $m$ and $n$ are positive relatively prime integers.  Compute the product $mn$.
\end{prob}

\begin{prob}[HMMT 2016 T6]{13}
A nonempty set $S$ is called $\emph{well-filled}$ if for every $m \in S$, there are fewer than $\frac 12 m$ elements of $S$ which are less than $m$. Determine the number of well-filled subsets of $\left\{ 1,2,\dots,42 \right\}$.
\end{prob}

\section{Olympiad Problems}
\begin{prob}[Shortlist 2015 C1]{6}
In Lineland there are $n\geq1$ towns, arranged along a road running from left to right. Each town has a \textit{left bulldozer} (put to the left of the town and facing left) and a \textit{right bulldozer} (put to the right of the town and facing right). The sizes of the $2n$ bulldozers are distinct. Every time when a left and right bulldozer confront each other, the larger bulldozer pushes the smaller one off the road. On the other hand, bulldozers are quite unprotected at their rears; so, if a bulldozer reaches the rear-end of another one, the first one pushes the second one off the road, regardless of their sizes.\\[1\baselineskip]Let $A$ and $B$ be two towns, with $B$ to the right of $A$. We say that town $A$ can \textit{sweep} town $B$ \textit{away} if the right bulldozer of $A$ can move over to $B$ pushing off all bulldozers it meets. Similarly town $B$ can sweep town $A$ away if the left bulldozer of $B$ can move over to $A$ pushing off all bulldozers of all towns on its way.\\[1\baselineskip]Prove that there is exactly one town that cannot be swept away by any other one.
\end{prob}

\begin{prob}[EGMO 2012/2]{6}
Let $n$ be a positive integer. Find the greatest possible integer $m$, in terms of $n$, with the following property: a table with $m$ rows and $n$ columns can be filled with real numbers in such a manner that for any two different rows $\left[ {{a_1},{a_2},\ldots,{a_n}}\right]$ and $\left[ {{b_1},{b_2},\ldots,{b_n}} \right]$ the following holds: \[\max\left( {\left| {{a_1} - {b_1}} \right|,\left| {{a_2} - {b_2}} \right|,...,\left| {{a_n} - {b_n}} \right|} \right) = 1\]
\end{prob}

\begin{prob}[IMO 2000/4]{9}
A magician has one hundred cards numbered 1 to 100. He puts them into three boxes, a red one, a white one and a blue one, so that each box contains at least one card. A member of the audience draws two cards from two different boxes and announces the sum of numbers on those cards. Given this information, the magician locates the box from which no card has been drawn.\\[1\baselineskip]How many ways are there to put the cards in the three boxes so that the trick works?
\end{prob}

\begin{prob}[USA TST 2015/5]{9}
A tournament is a directed graph for which every (unordered) pair of vertices has a single directed edge from one vertex to the other.  Let us define a proper directed-edge-coloring to be an assignment of a color to every (directed) edge, so that for every pair of directed edges $\overrightarrow{uv}$ and $\overrightarrow{vw}$, those two edges are in different colors.  Note that it is permissible for $\overrightarrow{uv}$ and $\overrightarrow{uw}$ to be the same color.  The directed-edge-chromatic-number of a tournament is defined to be the minimum total number of colors that can be used in order to create a proper directed-edge-coloring.  For each $n$, determine the minimum directed-edge-chromatic-number over all tournaments on $n$ vertices.
\end{prob}

\begin{prob}[USA TST 2009/1]{13}
Let $m$ and $n$ be positive integers.  Mr. Fat has a set $S$ containing every rectangular tile with integer side lengths and area of a power of $2$.  Mr. Fat also has a rectangle $R$ with dimensions $2^m \times 2^n$ and a $1 \times 1$ square removed from one of the corners.  Mr. Fat wants to choose $m + n$ rectangles from $S$, with respective areas $2^0, 2^1, \ldots, 2^{m + n - 1}$, and then tile $R$ with the chosen rectangles.  Prove that this can be done in at most $(m + n)!$ ways.
\end{prob}

\begin{prob}[China TST 2016/2/6]{13}
Let $m$, $n$ be integers satisfying $n \geq m \geq 2$ and let $S$ be a set consisting of n integers. Prove that $S$ has at least $2^{n-m+1}$ distinct subsets, each of whose sum is divisible by $m$. (This includes the empty set.)
\end{prob}

\end{document}
