\documentclass[mast]{lucky}



\title{Basics of Geometry}
\author{Dennis Chen}
\date{GPV}

\begin{document}

\maketitle

This is geometry lite - just similarity and angle chasing. Things that you should keep in mind include similarity/congruence criterion, collinearity/concurrency angle conditions, parallel line angle conditions, and the fact that a tangent is perpendicular to the radius.

\section{Theory}
Here are two theorems to keep in mind.

\begin{theo}[Inscribed Angle]
Let $A,B$ be points on a circle with center $O.$

If $C$ is a point on major arc $AB,$ then $\angle ACB=\frac{\angle AOB}{2}.$

If $C$ is a point on minor arc $AB,$ then $\angle ACB=180^{\circ}-\frac{\angle AOB}{2}.$
\end{theo}

\begin{pro}
Let $D$ be the antipode of $C.$ Then $\angle ACD=\frac{180^{\circ}-\angle AOC}{2}=\frac{\angle AOD}{2}.$ Thus addition or subtraction, depending on whether $O$ is inside acute angle $\angle ACB,$ of $\angle ACD$ and $\angle BCD$ will yield the result.
\end{pro}
\begin{center}
    \begin{asy}
    import olympiad;
    size(4cm);
    draw(circle((0,0), 1)); 
draw((-sqrt(2)/2,-sqrt(2)/2)--(0,1)); 
draw((-sqrt(2)/2,-sqrt(2)/2)--(0,0)); 
draw((0,1)--(0,-1)); 
draw((0,1)--((sqrt(6)+sqrt(2))/4,(-sqrt(6)+sqrt(2))/4)); 
draw((0,0)--((sqrt(6)+sqrt(2))/4,(-sqrt(6)+sqrt(2))/4));

draw(arc((0,0),0.2,225,270));
draw(arc((0,1),0.2,247.5,270));

dot((0,0)); 
label("$O$", (0,0), NE); 
dot((-sqrt(2)/2,-sqrt(2)/2)); 
label("$A$", (-sqrt(2)/2,-sqrt(2)/2), SW); 
dot(((sqrt(6)+sqrt(2))/4,(-sqrt(6)+sqrt(2))/4));
label("$B$", ((sqrt(6)+sqrt(2))/4,(-sqrt(6)+sqrt(2))/4), SE); 
dot((0,1)); 
label("$C$", (0,1), N); 
dot((0,-1)); 
label("$D$", (0,-1), S);
    \end{asy}
\end{center}


\begin{theo}[Tangent Angle]
Consider circle $\omega$ with center $O$ and points $A,B$ on $\omega.$ Let $\ell$ be the tangent to $\omega$ through $B$ and let $\theta$ be the acute angle between $AB$ and $\ell.$ Then $\theta=\frac{\angle AOB}{2}.$
\end{theo}

\begin{pro}
Let $B'$ be the antipode of $B.$ Then note that $\theta=90^{\circ}-\angle ABB'=\frac{180^{\circ}-\angle AOB'}{2}=\frac{\angle AOB}{2}.$
\end{pro}
\begin{center}
    \begin{asy}
    import olympiad;
    size(4cm);
    draw(circle((0,0), 1));
    dot((0,-1));
    dot((0.8,-0.6));
    dot((0,1));
    draw((0.8,-0.6)--(0,-1));
    draw((-2,-1)--(2,-1));
    draw((0,-1)--(0,1));
    
    label("$B$",(0,-1),S);
    label("$B'$",(0,1),N);
    label("$A$",(0.8,-0.6),NW);
    \end{asy}
\end{center}

A corollary of this theorem is that if $C$ is some point on $\overarc{AB},$ then $\theta=\angle ACB.$

With the Inscribed Angle Theorem in mind, try to prove these two theorems.

\begin{theo}[Angle of Secants/Tangents]
Let lines $AX$ and $BY$ intersect at $P$ such that $A,X,P$ and $B,Y,P$ are collinear in that order. Then $\angle APB=\frac{\angle AOB-\angle XOY}{2}.$

\begin{center}
    \begin{asy}
    import olympiad;
    size(4cm);
    draw(circle((0,0), 1)); 
draw((-1.5137699061468413,-0.5200439036267299)--(0.5006080051165431,0.8656740871790233)); 
draw((-1.5137699061468413,-0.5200439036267299)--(0.8698306996792652,-0.49335033586233623)); 

dot((-1.5137699061468413,-0.5200439036267299)); 
label("$P$", (-1.5137699061468413,-0.5200439036267299), NW); 
dot((0.5006080051165431,0.8656740871790233)); 
label("$A$", (0.5006080051165431,0.8656740871790233), NE); 
dot((0.8698306996792652,-0.49335033586233623)); 
label("$B$", (0.8698306996792652,-0.49335033586233623), NW); 
dot((-0.98744289059587,-0.15797638371501177)); 
label("$X$", (-0.98744289059587,-0.15797638371501177), NW); 
dot((-0.858564029240042,-0.5127063562070442)); 
label("$Y$", (-0.858564029240042,-0.5127063562070442), NE); 
    \end{asy}
\end{center}
\end{theo}

\begin{theo}[Angle of Chords]
Let chords $AC,BD$ intersect at $P.$ Then $\angle APB=\frac{\angle AOB+\angle COD}{2}.$
\begin{center}
    \begin{asy}
    import olympiad;
    size(3.5cm);
    draw(arc((-0.002944953385821347,0.1945321088804997),0.25,396.63059336924744,499.02864317828227)); 

draw(circle((0,0), 1)); 
draw((-0.6520239138456277,0.758198401326084)--(0.8421035046648182,-0.5393159439801781)); 
draw((0.6982426762645825,0.7158611353068928)--(-0.8866287744814202,-0.462481801005807)); 

dot((-0.6520239138456277,0.758198401326084)); 
label("$A$", (-0.6520239138456277,0.758198401326084), NW); 
dot((0.6982426762645825,0.7158611353068928)); 
label("$B$", (0.6982426762645825,0.7158611353068928), NE); 
dot((0.8421035046648182,-0.5393159439801781)); 
label("$C$", (0.8421035046648182,-0.5393159439801781), SE); 
dot((-0.8866287744814202,-0.462481801005807)); 
label("$D$", (-0.8866287744814202,-0.462481801005807), SW); 
dot((-0.002944953385821347,0.1945321088804997)); 
label("$P$", (-0.002944953385821347,0.1945321088804997), S); 
\end{asy}
\end{center}
\end{theo}

Here's a very important application of Inscribed Angle.

\begin{theo}[Cyclic Quadrilaterals]
Any one of the three implies the other two:
\begin{enumerate}
    \item Quadrilateral $ABCD$ is cyclic.
    
    \item $\angle ABC+\angle ADC=180^{\circ}.$
    
    \item $\angle BAC=\angle BDC.$
\end{enumerate}
\begin{center}
\begin{asy}
import olympiad;
size(3.5cm);

draw(circle((0,0), 1)); 

dot((-0.6520239138456277,0.758198401326084)); 
label("$A$", (-0.6520239138456277,0.758198401326084), NW); 
dot((0.6982426762645825,0.7158611353068928)); 
label("$B$", (0.6982426762645825,0.7158611353068928), NE); 
dot((0.8421035046648182,-0.5393159439801781)); 
label("$C$", (0.8421035046648182,-0.5393159439801781), SE); 
dot((-0.8866287744814202,-0.462481801005807)); 
label("$D$", (-0.8866287744814202,-0.462481801005807), SW); 

draw((-0.6520239138456277,0.758198401326084)--(0.6982426762645825,0.7158611353068928)--(0.8421035046648182,-0.5393159439801781)--(-0.8866287744814202,-0.462481801005807)--cycle);
\end{asy}
\begin{asy}
import olympiad;
size(3.5cm);
draw(arc((-0.6520239138456277,0.758198401326084),0.3352236296243588,319.02864317828227,358.2040937604978)); 
draw(arc((-0.8866287744814202,-0.462481801005807),0.3352236296243588,357.45514278703183,396.63059336924744)); 

draw(circle((0,0), 1)); 
draw((0.6982426762645825,0.7158611353068928)--(-0.6520239138456277,0.758198401326084)); 
draw((-0.6520239138456277,0.758198401326084)--(0.8421035046648182,-0.5393159439801781)); 
draw((0.6982426762645825,0.7158611353068928)--(-0.8866287744814202,-0.462481801005807)); 
draw((-0.8866287744814202,-0.462481801005807)--(0.8421035046648182,-0.5393159439801781)); 
draw(arc((-0.6520239138456277,0.758198401326084),0.3352236296243588,319.02864317828227,358.2040937604978)); 
draw((-0.3762942045336219,0.650231969055075)--(-0.3034599416964878,0.6217125341155632)); 
draw(arc((-0.8866287744814202,-0.462481801005807),0.3352236296243588,357.45514278703183,396.63059336924744)); 
draw((-0.6035182048501434,-0.37569454123994417)--(-0.5287342807965981,-0.35276960469801844));

dot((-0.6520239138456277,0.758198401326084)); 
label("$A$", (-0.6520239138456277,0.758198401326084), NW); 
dot((0.6982426762645825,0.7158611353068928)); 
label("$B$", (0.6982426762645825,0.7158611353068928), NE); 
dot((0.8421035046648182,-0.5393159439801781)); 
label("$C$", (0.8421035046648182,-0.5393159439801781), SE); 
dot((-0.8866287744814202,-0.462481801005807)); 
label("$D$", (-0.8866287744814202,-0.462481801005807), SW);
\end{asy}
\end{center}
\end{theo}

\section{Examples}
We present several examples of angle chasing problems, sorted by ``flavor.''

\subsection{Computational Problems}
This is a compilation of computational problems meant to serve as low-level examples for first-time readers. If this is your first time encountering the material, I strongly suggest you focus on this section.
\begin{exam}[AMC 10B 2011/18]
Rectangle $ABCD$ has $AB = 6$ and $BC = 3$. Point $M$ is chosen on side $AB$ so that $\angle AMD = \angle CMD$. What is the degree measure of $\angle AMD?$
\end{exam}

\begin{sol}
Note that $\angle CMD=\angle AMD=\angle AMD=\angle MDC,$ implying that $CM=CD=6.$ Thus $\angle BMC=30^{\circ},$ implying that $\angle AMD=75^{\circ}.$
\begin{center}
\begin{asy}
size(4cm);

pair A=(0,3), B=(6,3), C=(6,0), D=(0,0);
pair M=(0.80385,3);

draw(A--B--C--D--cycle);
draw(M--C);
draw(M--D);

dot("$A$",A,NW);
dot("$B$",B,NE);
dot("$C$",C,SE);
dot("$D$",D,SW);
dot("$M$",M,N);
\end{asy}
\end{center}
\end{sol}

\begin{exam}
Two circles $\omega_1,\omega_2$ intersect at $P,Q.$ If a line intersects $\omega_1$ at $A,B$ and $\omega_2$ at $C,D$ such that $A,B,C,D$ lie on the lie in that order, and $P$ and $Q$ lie on the same side of the line, compute the value of $\angle APC+\angle BQD.$
\end{exam}

\begin{sol}
Without loss of generality, let $P$ be closer to $\ell$ than $Q.$ Note
\[\angle APC=180-\angle PAB-\angle BCP=\angle DCP-\angle PAB\]\[\angle BQD=\angle BQP+\angle DQP.\]Since $\angle PAB=\angle BDP,$ the sum is $\angle DCP+\angle DQP=180.$
\end{sol}

\subsection{Construct the Diagram}
These problems are very simple; just construct the diagram and the problem will solve itself for you.

\begin{exam}[USA EGMO TST 2020/4]
Let $ABC$ be a triangle. Distinct points $D$, $E$, $F$ lie on sides $BC$, $AC$, and $AB$, respectively, such that quadrilaterals $ABDE$ and $ACDF$ are cyclic. Line $AD$ meets the circumcircle of $\triangle ABC$ again at $P$. Let $Q$ denote the reflection of $P$ across $BC$. Show that $Q$ lies on the circumcircle of $\triangle AEF$.
\end{exam}

\begin{sol}
Note that $Q$ is the intersection of $BE$ and $CF,$ since $\angle EBD=\angle CAP=\angle CBP$ and $\angle FCB=\angle BAP=\angle BCP.$ Now note that $\angle BQC=\angle BPC=180^{\circ}-\angle A.$

\begin{center}
\begin{asy}
size(6cm);
//import geometry;

point A=(0.5,3);
point B=(0,0);
point C=(4,0);
point D=(B+C)/2.5;

circle O1=circle(A,B,D);
circle O2=circle(A,C,D);
circle cABC=circle(A,B,C);

point E_=intersectionpoints(O1, line(A,C))[0];
point F=intersectionpoints(O2, line(A,B))[0];
point P=intersectionpoints(cABC, line(A,D))[0];
point Q=reflect(B,C)*P ;
point M=(P+Q)/2 ;

draw(A--B--C-- cycle );
draw(O1 ^^ O2 ^^ cABC);
draw(B--Q--C,blue);
draw(B--P--C);
draw(F--Q--E_, blue);
//draw(A--P ^^ P--Q , gray);

dot("$A$", A, N+W);
dot("$B$", B, S+W);
dot("$C$", C, E+S);
dot("$D$", D, 2*S+W);
dot("$E$", E_, E+N);
dot("$F$", F, W);
dot("$P$", P, S);
dot("$Q$", Q, N);
\end{asy}
\end{center}
\end{sol}

The motivation is just drawing the diagram -- as soon as you figure out that $Q$ lies on $BE$ and $CF,$ the problem solves itself from there.

Here's a slightly harder example.

\begin{exam}[KJMO 2015/1]
In an acute, scalene triangle $\triangle ABC$, let $O$ be the circumcenter. Let $M$ be the midpoint of $AC$. Let the perpendicular from $A$ to $BC$ be $D$. Let the circumcircle of $\triangle OAM$ hit $DM$ at $P\neq M$. Prove that $B, O, P$ are colinear.
\end{exam}

\begin{sol}
Instead we show that the intersection of $MD$ and $BO,$ which we will call $P',$ lies on $(MAO).$ The central claim is that $PABD$ is cyclic.

Note $\angle PDA=\angle MDA=90^{\circ}-\angle C,$ and also note that $\angle PAD=\angle PAB-\angle DAB.$ Note that $\angle PAB=\angle C$ since $\angle APB=90^{\circ}$ and $\angle ABP=\angle ABO=90^{\circ}-\angle C$ and $\angle BAD=90^{\circ}-\angle B.$ Thus $\angle PAD=\angle B+\angle C-90^{\circ}.$

Now consider $\triangle PAD.$ Note $\angle DPA=180^{\circ}-(\angle PDA+\angle PAD)=180^{\circ}-\angle B.$ Thus $PABD$ is cyclic.

This implies that $\angle APO=\angle APB=\angle ADB=90^{\circ}.$ Since $\angle AMO=90^{\circ}$ as well, we are done.
\begin{center}
\begin{asy}
size(6cm);
pair A = dir(110);
pair B = dir(-40);
pair C = dir(220);
pair O = circumcenter(A,B,C);
pair M;
M=(A+C)/2;
pair D;
D=foot(A,B,C);
pair P=extension(D,M,B,O);

draw(A--B--C--cycle);
draw(D--P);
draw(P--B,dotted);
draw(circumcircle(P,A,B),blue);
draw(circumcircle(A,B,C));
draw(circumcircle(O,A,M));

dot("$A$",A,N);
dot("$B$",B,SE);
dot("$C$",C,SW);
dot("$D$",D,S);
dot("$M$",M,W);
dot("$O$",O,S);
dot("$P$",P,dir(190));
\end{asy}
\end{center}
\end{sol}

This final example demonstrates the power of wishful thinking.

\begin{exam}[ISL 2010/G1]
Let $ABC$ be an acute triangle with $D, E, F$ the feet of the altitudes lying on $BC, CA, AB$ respectively. One of the intersection points of the line $EF$ and the circumcircle is $P.$ The lines $BP$ and $DF$ meet at point $Q.$ Prove that $AP = AQ.$
\end{exam}

\begin{sol}
We work in directed angles because there are plenty of configuration issues. (If you don't know what directed angles are, consult the chapter on them.)

Note that $AFPQ$ is cyclic, as
\[\measuredangle AFQ=\measuredangle BFD=\measuredangle ACB=\measuredangle APB=\measuredangle APQ.\]

Now note that
\[\measuredangle APQ=\measuredangle AFQ=\measuredangle BFD=\measuredangle ACB\]
\[\measuredangle PQA=\measuredangle PFA=\measuredangle EFA=\measuredangle ACB,\]
implying that $\angle APQ=\angle PQA,$ or that $AP=AQ.$

\begin{center}
\begin{asy}
size(6cm);
pair C=origin,
A=(5,12),
B=(14,0),
D=foot(A,B,C),
E=foot(B,A,C),
F=foot(C,A,B),
H=orthocenter(A,B,C),
om=extension(E,F,C,B),
nom=extension(E,F,D,(9,3));
path circ = circumcircle(A,B,C);
pair P1=intersectionpoint(F--om, circ),
P2=intersectionpoint(E--nom, circ),
Q1=intersectionpoint(D--F, B--P1),
Q2=extension(B,P2,D,F),
Ep=reflect(A,B)*E,
Cp=reflect(A,B)*C,
Hp=reflect(A,B)*H,
Dp=reflect(A,B)*D;
draw(A--B--C--cycle);
draw(D--F);
draw(P1--F);
draw(P1--B);
dot(A^^B^^C^^D^^E^^F^^Q1^^P1);
draw(circumcircle(A,P1,Q1),dotted);
draw(circ);
label("$A$", A, N);
label("$B$", B, SE);
label("$C$", C, SW);
label("$D$", D, S);
label("$E$", E, NW);
label("$F$", F, dir(60));
label("$P$", P1, W);
label("$Q$", Q1, dir(290));
\end{asy}
\end{center}
\end{sol}

I personally thought this problem was harder than the other two, especially since the cyclic quadrilateral had an asymmetric structure with respect to the whole diagram.\footnote{This is explain by the entire diagram being asymmetric.} We're inclined to look for cyclic quadrilaterals involving $A,P,Q$ in some way because the problem is essentially equivalent to showing that $\angle AQP=\angle APQ,$ and a little bit of experimentation shows that it's hard to show directly. The motivation for trying to prove $F$ is the point on $(APQ)$ is drawing in the circumcircles for both configurations, and noting that the second intersection point of them is $F.$

The rest of the motivation is quite straightforward -- all you have to do afterwards is try to solve the problem with the assumption that $AFPQ$ is cyclic, and that part is fairly easy if you have any knowledge about the \hyperref[itm:orthicBCEFcyclic]{orthic triangle}.

\subsection{Tangent Angle Criterion}
When tangent lines are given, you have to pay close attention the the tangent angle criterion.

\begin{exam}[British Math Olympiad Round 1 2000/1]
Two intersecting circles $C_1$ and $C_2$ have a common tangent which touches $C_1$ at $P$ and $C_2$ at $Q.$ The two circles intersect at $M$ and $N,$ where $N$ is nearer to $PQ$ than $M$ is. The line $PN$ meets the circle $C_2$ again at $R.$ Prove that $MQ$ bisects angle $PMR.$
\end{exam}

\begin{sol}
Note that $\angle RMQ=180^{\circ}-\angle RNQ=180^{\circ}-(\angle PNM+\angle QNM)=180^{\circ}-(\angle QPM+\angle PQM)=\angle PMQ.$

(This actually only takes care of the case where $R$ is in between $P$ and $N.$ Can you show this is true for the other configuration as well?)
\end{sol}
Let's expound on the motivation for this. We want to prove that $PQ$ bisects $\angle PMN,$ but it's quite hard to find the supplement of $\angle PMQ$ and $\angle RMQ.$ This then motivates showing that $\angle PMQ=\angle RMQ,$ because those angles seem more workable. We start by manipulating $\angle RMQ$ because it seems more unwieldy, and it feels like there are more ways to get to $\angle PMQ$ than $\angle RMQ.$ (This part is personal preference, but a good rule of thumb is to try to manipulate the least independently defined points into the most independently defined points.\footnote{A heuristic for the independence of a point is how much it would affect the diagram on GeoGebra if it was deleted.})

The cyclic quadrilateral $RMNQ$ is the source of the only useful manipulation we can do with $\angle RMQ,$ so we're pretty much forced into using it. Now looking at $\triangle PNQ$ as a whole motivates $\angle RNQ=180^{\circ}-(\angle PNM+\angle QNM),$ and at this point we want to start manipulating $\angle PMQ.$ We're forced into doing $180^{\circ}-(\angle QPM+\angle PQM)=\angle PMQ,$ because tangent lines have lots of potential for angle chasing and it's the only place to go.

Now the rest of the problem will just come naturally by just trying things.

\subsection{Orthocenter}
Sometimes a problem will ask you to prove that $AH\perp BC$ for some point $H$ not on $BC.$ This is generally difficult to do directly, and one of the more elementary methods used is to show that $H$ is the orthocenter of $\triangle ABC,$ or $BH\perp CA$ and $CH\perp AB.$

This is obviously not always going to be true, so make sure that this actually seems true before you try too hard to prove it.

\begin{exam}[Swiss Math Olympiad 2007/4]
Let $ABC$ be an acute-angled triangle with $AB> AC$ and orthocenter $H$. Let $D$ be the projection of $A$ on $BC$. Let $E$ be the reflection of $C$ about $D$. The lines $AE$ and $BH$ intersect at point $S$. Let $N$ be the midpoint of $AE$ and let $M$ be the midpoint of $BH$. Prove that $MN$ is perpendicular to $DS$.
\end{exam}

\begin{sol}
We claim $S$ is the orthocenter of $\triangle DEM.$ To do this, it suffices to show that $SN\perp DM$ and $SM\perp DN.$ Let $H'$ be the second intersection of $AH$ with $(ABC).$

Note that $DM\parallel BH'$ by a \hyperref[section:homothety]{homothety} about $H,$ $\angle MAE=\angle DAC=90^{\circ}-\angle C,$ and $\angle AMB=\angle C,$ proving $SN\perp DM.$

Now note that $DN\parallel AC$ by a \hyperref[section:homothety]{homothety} about $E,$ proving $SM\perp DN.$
\end{sol}

\pagebreak

\section{Problems}

\minpt{40}

\psetquote{How arrogant. The life of each human is worth one, that's it. Nothing more, nothing less.}{Fullmetal Alchemist: Brotherhood}

\begin{prob}[AMC 10A 2020/12]{2}
Triangle $AMC$ is isosceles with $AM = AC$. Medians $\overline{MV}$ and $\overline{CU}$ are perpendicular to each other, and $MV=CU=12$. What is the area of $\triangle AMC?$
\end{prob}

\begin{center}
\begin{asy}
import olympiad;
size(4cm);
draw((-4,0)--(4,0)--(0,12)--cycle);
draw((-2,6)--(4,0));
draw((2,6)--(-4,0));
label("M", (-4,0), W);
label("C", (4,0), E);
label("A", (0, 12), N);
label("V", (2, 6), NE);
label("U", (-2, 6), NW);
label("P", (0, 3.6), S);
\end{asy}
\end{center}

\begin{prob}[Brazil 2004]{2}
In the figure, $ABC$ and $DAE$ are isosceles triangles ($AB = AC = AD = DE$) and the angles $BAC$ and $ADE$ have measures $36^{\circ}$.
\end{prob}
\begin{enumerate}
    \item Using geometric properties, calculate the measure of angle $\angle EDC$.
    \item Knowing that $BC = 2$, calculate the length of segment $DC$.
    \item Calculate the length of segment $AC$.
\end{enumerate}
\begin{center}
\begin{asy}
import olympiad;
size(4cm);
pair A=(0,0),B=dir(0),C=dir(36),D=dir(72),E=(2cos(0.4pi),0);
dot(A^^B^^C^^D^^E);
draw(A--B--C--cycle);
draw(A--D--E);
draw(D--C,dotted);
label("A",A,dir(-90));
label("B",B,dir(-90));
label("C",C,dir(40));
label("D",D,dir(90));
label("E",E,dir(-90));
\end{asy}
\end{center}

\begin{prob}[]{2}
Let circles $\omega_1$ and $\omega_2$ intersect at $X,Y.$ Let line $\ell_1$ passing through $X$ intersect $\omega_1$ at $A$ and $\omega_2$ at $C,$ and let line $\ell_2$ passing through $Y$ intersect $\omega_1$ at $B$ and $\omega_2$ at $D.$ If $\ell_1$ intersects $\ell_2$ at $P,$ prove that $\triangle PAB\sim \triangle PCD.$
\end{prob}
    
\begin{prob}[AMC 10B 2011/17]{2}
In the given circle, the diameter $\overline{EB}$ is parallel to $\overline{DC}$, and $\overline{AB}$ is parallel to $\overline{ED}$. The angles $AEB$ and $ABE$ are in the ratio $4 : 5$. What is the degree measure of angle $BCD?$
\end{prob}
    
\begin{center}
\begin{asy}
import olympiad;

size(4cm);

real r=3;
pair A=(-3cos(80),-3sin(80));
pair D=(3cos(80),3sin(80)), C=(-3cos(80),3sin(80));
pair O=(0,0), E=(-3,0), B=(3,0);
path outer=Circle(O,r);
draw(outer);
draw(E--B);
draw(E--A);
draw(B--A);
draw(E--D);
draw(C--D);
draw(B--C);

pair[] ps={A,B,C,D,E,O};
dot(ps);

label("$A$",A,N);
label("$B$",B,NE);
label("$C$",C,S);
label("$D$",D,S);
label("$E$",E,NW);
\end{asy}
\end{center}

\begin{prob}[Dennis Chen]{3}
Consider rectangle $ABCD$ with $AB = 6,$ $BC = 8.$ Let $M$ be the midpoint of $AD$ and let $N$ be the midpoint of $CD.$ Let $BM$ and $BN$ intersect $AC$ at $X$ and $Y$ respectively. Find $XY.$
\end{prob}
    
\begin{prob}[AMC 10A 2019/13]{3}
Let $\triangle ABC$ be an isosceles triangle with $BC = AC$ and $\angle ACB = 40^{\circ}$. Construct the circle with diameter $\overline{BC}$, and let $D$ and $E$ be the other intersection points of the circle with the sides $\overline{AC}$ and $\overline{AB}$, respectively. Let $F$ be the intersection of the diagonals of the quadrilateral $BCDE$. What is the degree measure of $\angle BFC?$
\end{prob}
    
\begin{req}[Miquel's Theorem]{3}
Consider $\triangle ABC$ with $D$ on $BC,$ $E$ on $CA,$ and $F$ on $AB.$ Prove that $(AEF),$ $(BFD),$ and $(CDE)$ concur.
\end{req}

\begin{prob}[]{2}
Consider $\triangle ABC$ with $D$ on segment $BC,$ $E$ on segment $CA,$ and $F$ on segment $AB.$ Let the circumcircles of $\triangle FBD$ and $\triangle DCE$ intersect at $P\neq D.$ If $\angle A=50^{\circ},\angle B=35^{\circ},$ find $\angle DPE.$
\end{prob}

\begin{prob}[AIME II 2018/4]{3}
In equiangular octagon $CAROLINE$, $CA = RO = LI = NE =$ $\sqrt{2}$ and $AR = OL = IN = EC = 1$. The self-intersecting octagon $CORNELIA$ encloses six non-overlapping triangular regions. Let $K$ be the area enclosed by $CORNELIA$, that is, the total area of the six triangular regions. Then $K =$ $\dfrac{a}{b}$, where $a$ and $b$ are relatively prime positive integers. Find $a + b$.
\end{prob}

\begin{req}[Brazil 2007]{4}
Let $ABC$ be a triangle with circumcenter $O$. Let $P$ be the intersection of straight lines $BO$ and $AC$ and $\omega$ be the circumcircle of triangle $AOP$. Suppose that $BO = AP$ and that the measure of the arc $OP$ in $\omega$, that does not contain $A$, is $40^{\circ}$. Determine the measure of the angle $\angle OBC$.
\end{req}

\begin{prob}[]{4}
Consider square $ABCD$ and some point $P$ outside $ABCD$ such that $\angle APB=90^{\circ}.$ Prove that the angle bisector of $\angle APB$ also bisects the area of $ABCD.$
\end{prob}
 
\begin{req}[AMC 10B 2018/12]{4}
Line segment $\overline{AB}$ is a diameter of a circle with $AB=24$. Point $C$, not equal to $A$ or $B$, lies on the circle. As point $C$ moves around the circle, the centroid (center of mass) of $\triangle{ABC}$ traces out a closed curve missing two points. To the nearest positive integer, what is the area of the region bounded by this curve?
\end{req}

\begin{prob}[Formula of Unity 2018]{6}
A point $O$ is the center of an equilateral triangle $ABC.$ A circle that passes through points $A$ and $O$ intersects the sides $AB$ and $AC$ at points $M$ and $N$ respectively. Prove that $AN = BM.$
\end{prob}

\begin{req}[AMC 10A 2021/17]{6}
Trapezoid $ABCD$ has $\overline{AB} \parallel \overline{CD}$, $BC = CD = 43$, and $\overline{AD} \perp \overline{BD}$. Let $O$ be the intersection of the diagonals $\overline{AC}$ and $\overline{BD}$, and let $P$ be the midpoint of $\overline{BD}$. GIven that $OP = 11$, the length $AD$ can be written in the form $m\sqrt{n}$, where $m$ and $n$ are positive integers and $n$ is not divisible by the square of any prime. What is $m + n$?
\end{req}
    
\begin{prob}[Memorial Day Mock AMC 10 2018/21]{6}
In the following diagram, $m\angle BAC=m\angle BFC=40^{\circ}$, $m\angle ABF=80^{\circ}$, and $m\angle FEB=2m\angle DBE=2m\angle FBE$. What is $m\angle ADB$?
\end{prob}
    
    \begin{center}
    \begin{asy}
    size(4cm);
    draw((0,0)--(-14,0)--(2,8)--(0,0)--(-9,2.5)--(-5.5,0)--(-6,4)--cycle);
    draw((-5.5,0)--(2,8));
    label("A", (2,8), NE);
    label("B", (0,0), SE);
    label("C", (-5.5,0), S);
    label("D", (-14,0), SW);
    label("E", (-9,2.5), NNW);
    label("F", (-6,4), NNW);
    \end{asy}
    \end{center}
    
\begin{prob}[FARML 2012/6]{6}
In triangle $ABC,$ $AB=7,$ $AC=8,$ and $BC=10.$ $D$ is on $AC$ and $E$ is on $BC$ such that $\angle AEC=\angle BED=\angle B+\angle C.$ Compute the length $AD.$
\end{prob}

\begin{prob}[USAJMO 2020/4]{9}
Let $ABCD$ be a convex quadrilateral inscribed in a circle and satisfying $DA < AB = BC < CD$. Points $E$ and $F$ are chosen on sides $CD$ and $AB$ such that $BE \perp AC$ and $EF \parallel BC$. Prove that $FB = FD$.
\end{prob}
    
\begin{prob}[MAST Diagnostic 2020]{13}
Consider $\triangle ABC$ with $D$ on line $BC.$ Let the circumcenters of $\triangle ABD$ and $\triangle ACD$ be $M,N,$ respectively. Let the circumcircle of $\triangle MND$ intersect the circumcircle of $\triangle ACD$ again at $H\neq D.$ Prove that $A,M,H$ are collinear.
\end{prob}

\end{document}