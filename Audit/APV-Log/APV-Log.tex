\section{Theory}
We commonly have functions of the form $f(x)=x^n,$ and to find $f^{-1}(x),$ we just take the $n$th root of both sides to get $\sqrt[n]{x}=f^{-1}(x).$ But how would we find the inverse of a function like $f(x)=n^x?$ To do this, we create an inverse function known as a logarithm, where $n^{\log_nx}=x.$ Note that the base of a logarithm, which is $b$ in the case of $\log_b(x)$, must be in the interval $(0, 1)$ or greater than $1$.  In particular, it is important to check for cases where $b = 1$ in problems because not doing so may lead to extraneous solutions.

Here are two examples to get you up to speed.

\begin{exam}
Find $\log_28.$
\end{exam}

\begin{sol}
Notice that $2^{\log_28}=8=2^3$ by the definition of $\log,$ so $\log_28=\ansbold{3.}$
\end{sol}

\begin{exam}
Simplify $\frac{\log_5x}{\log_{25}x}.$
\end{exam}

\begin{sol}
Let $25^a=x.$ Then notice $5^{2a}=x.$ Substituting yields $\frac{2a}{a}=\ansbold{2.}$
\end{sol}

Here's a motivating exercise for what's going to come next.

\begin{exer}
Evaluate $\log_2{16}+\log_2{32},$ and then evaluate $\log_2{16\cdot 32}.$
\end{exer}

\subsection{Fundamental Rules}
The fundamental two rules of logarithms are the addition and subtraction rules. Notice that addition outside becomes multiplication inside, and similarly, subtraction outside becomes division inside. This is a consequence of the way exponents behave: $x^{a+b}=x^a\cdot x^b.$

\begin{theo}[Logarithm Addition]
Given positive reals $a,b,c$ with $a > 0$ and $a \neq 1$, $\log_{a}b+\log_{a}c=\log_{a}{bc}.$
\end{theo}

\begin{pro}
Notice that $a^{\log_ab+\log_ac}=a^{\log_ab}\cdot a^{\log_ac}=bc=a^{\log_abc}.$ Since the bases are the same, it follows the exponents are the same.
\end{pro}

\begin{theo}[Logarithm Subtraction]
Given positive reals $a,b,c$ with $a > 0$ and $a \neq 1$, $\log_{a}b-\log_{a}c=\log_{a}{\frac{b}{c}}.$
\end{theo}

\begin{pro}
This is a repeat of logarithm addition. Notice that $a^{\log_ab-\log_ac}=\frac{a^{\log_ab}}{ a^{\log_ac}}=\frac{b}{c}=a^{\log_a\frac{b}{c}}.$
\end{pro}

Notice that we're exploiting the properties of logarithms, and we're expressing everything without using logarithms as soon as possible. This trend will continue in AIME problems; once the logs have been removed, there's not much underneath to solve.

\subsection{Base Change}
The base change rule allows you to express all logarithms in the same base; this is extremely powerful, even if it doesn't look like much.

\begin{theo}[Base Change]
Given positive reals $a,b,c$ with $a > 0$ and $a \neq 1$, $\frac{\log_{a}b}{\log_{a}c}=\log_{c}b.$
\end{theo}

\begin{pro}
Have $x=\log_ab,$ $y=\log_ac,$ and $z=\log_cb.$ Notice that $a^x=b,a^y=c,c^z=b.$ Then $(a^y)^z=a^x,$ implying $yz=x$ or $\frac{x}{y}=z.$
\end{pro}

This is one of the fundamental manipulation techniques for logarithmic manipulations. This is very convenient because it can be used to make all the logarithms share an arbitrary common base. Specifically,
\[\log_ba=\frac{\log a}{\log b}.\]

\begin{exam}[AMC 12B 2021/9]
What is the value of \[\frac{\log_2 80}{\log_{40}2}-\frac{\log_2 160}{\log_{20}2}?\]
\end{exam}

\begin{sol}
First we convert everything to base $10$ with the base change theorem. Note that our expression is equivalent to
\[\frac{\log 80\log 40 - \log 160\log 20}{(\log 2)^2}.\]
Here is where things get a little tricky. Note that $\log 160 = \log 80 + \log 2$ and $\log 20 = \log 40 - \log 2$, so our expression becomes
\begin{align*}
\frac{\log 80\log 40 - (\log 80+\log 2)(\log 40-\log 2)}{(\log 2)^2}&=\frac{\log 2(\log 80-\log 40)+(\log 2)^2}{(\log 2)^2} \\
&=\frac{\log 2(\log 2)+(\log 2)^2}{(\log 2)^2} \\
&=\ansbold{2.}
\end{align*}
\end{sol}

The motivation for expressing $\log 160$ as $\log 80 + \log 2$ is twofold. The general reason is that we want things to simplify with an SFFT-esque expression, and the reason we use $\log 2$ specifically is because it is in the denominator.

We present the so-called logarithm chain rule as an exercise. (It's pretty useless and is only being presented as a check-up.)

\begin{exer}[Logarithm Chain Rule]
Given positive reals $a,b,c,d$ with $a, c > 0$ and $a, c \neq 1$, $\log_{a}b\log_{c}d=\log_{a}d\log_{c}b.$
\end{exer}

\section{Examples}
Here are some examples of AIME logarithm problems. I want to re-iterate the following with these two problems: usually, \dbold{interpreting the log condition is the entire problem}.

\begin{exam}[AIME II 2009/2]
Suppose that $a$, $b$, and $c$ are positive real numbers such that $a^{\log_3 7} = 27$, $b^{\log_7 11} = 49$, and $c^{\log_{11}25} = \sqrt{11}$. Find
\[a^{(\log_3 7)^2} + b^{(\log_7 11)^2} + c^{(\log_{11} 25)^2}.\]
\end{exam}

\begin{sol} We notice that $a^{(\log_37)^2}=(a^{\log_37})^{\log_37}.$ Similar expressions hold for $b,c.$

We then substitute $a^{\log_37}=27$ as defined in the problem statement, and we do the same for $b,c$. This becomes $27^{\log_37}+49^{\log_711}+\sqrt{11}^{\log_{11}25}=3^{3\log_37}+7^{2\log_7{11}}+11^{\frac{1}{2}\log_{11}25}=7^3+11^2+25^{\frac{1}{2}}.$ This simplifies to $\ansbold{469},$ which is our answer.\end{sol}

\begin{exam}[AIME I 2011/9]
Suppose $x$ is in the interval $[0, \pi/2]$ and $\log_{24\sin x} (24\cos x)=\frac{3}{2}$. Find $24\cot^2 x$.
\end{exam}

\begin{sol}
We can rewrite this as $(24\sin x)^3=(24\cos x)^2,$ which implies $24\sin^3 x=\cos^2 x=1-\sin^2 x.$ Thus we want to find the positive root of $24\sin^3 x+\sin^2 x-1=0.$ Using the Rational Root Theorem (aka guessing), we see that $\frac{1}{3}$ is a root. Thus $\cot x=2\sqrt{2}$ and our answer is $\ansbold{192.}$\end{sol}
