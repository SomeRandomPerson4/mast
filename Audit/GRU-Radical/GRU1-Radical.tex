\documentclass[mast]{lucky}



\title{Radical Axes}
\author{Dennis Chen}
\date{GRU}

\begin{document}

\maketitle

\section{Power of a Point}
We first extend the power of a point theorem to a definition.
\begin{defi}[Power of a Point]
The power of a point $P$ with respect to circle $\omega$ with center $O$ and radius $r$ as $OP^2-r^2.$ We will denote this as $\pow(P,\omega)=OP^2-r^2.$
\end{defi}
This yields the following corollary.
\begin{fact}[Square of Tangent Line]
The power of a point is the square of the length of the tangent line.
\end{fact}
\section{Radical Axes}
\begin{defi}[Radical Axis]
The radical axis of a pair of circles $\omega_1,\omega_2$ as the locus of points such that $\pow(P,\omega_1)=\pow(P,\omega_2).$
\end{defi}

\begin{theo}[Radical Axis Theorem]
The radical axis of a pair of circles is a line.
\end{theo}

\begin{pro}
We use coordinates to prove this.

Without loss of generality, let the center of $\omega_1$ be $(0,0)$ and let the center of $\omega_2$ be $(x_{0},0).$ Denote the radii of $\omega_1,\omega_2$ as $r_1,r_2,$ respectively. If the coordinates of $P$ are $(x,y),$ then $(x^2+y^2)-r_1^2=([x-x_{0}]^2+y^2)-r_2^2.$ Rearranging, this yields $-r_1^2=-2x_{0}x+x^2_{0}-r_2^2.$ This is the equation of a line, as desired. 
\end{pro}

A corollary that arises: if $\omega_1,\omega_2$ intersect at $X,Y,$ then the radical axis is $XY.$ This is because $\pow(X,\omega_1)=0=\pow(X,\omega_2)$ and $\pow(Y,\omega_1)=0=\pow(Y,\omega_2).$ Since two points are needed to determine a line, the proof is done.

\begin{theo}[Radical Center Theorem]
Consider three circles $\omega_1,\omega_2,\omega_3.$ Then their pairwise radical axes either concur or are all parallel.
\end{theo}

\begin{pro}
Without loss of generality, let the radical axis of $\omega_1,\omega_3$ and the radical axis of $\omega_1,\omega_2$ intersect at $P.$\footnote{If they do not intersect, then all three radical axes are parallel; this is because the proof works regardless of which pair of radical axes you intersect as it is symmetric.} Then notice that $\pow(P,\omega_1)=\pow(P,\omega_3)$ and $\pow(P,\omega_1)=\pow(P,\omega_2),$ so $\pow(P,\omega_2)=\pow(P,\omega_3),$ implying that $P$ lies on the radical axis of $\omega_2,\omega_3,$ as desired.
\end{pro}

\section{Techniques}

\subsection{Basic Techniques}
Keeping this simple result in mind kills problems involving common chords and external tangents.

\begin{fact}[Radical Axis Bisects External Tangent]
Consider two circles $\omega_1, \omega_2$ that intersect at $X,Y.$ Let one of their common external tangents intersect $\omega_1$ at $A$ and $\omega_2$ at $B.$ Then $XY$ bisects $AB.$
\end{fact}

\begin{pro}
Note that $XY$ is the radical axis. Let $XY$ intersect $AB$ at $M.$ Since $M$ lies on the radical axis, $AM=BM.$
\end{pro}

\subsection{Advanced Techniques}
We introduce two powerful techniques - Linearity of Power and circles with radius $0.$
\begin{theo}[Linearity of Power]
The function $f(P)=\pow(P,\omega_1)-\pow(P,\omega_2)$ changes at a linear rate as $P$ moves along a fixed line $\ell.$
\end{theo}

\begin{pro}
Without loss of generality, let $\ell$ be the $x$ axis, let the equation of $\omega_1$ be $(x-h_1)^2+(y-k_1)^2=r_1^2,$ and let the equation of $\omega_2$ be $(x-h_2)^2+(y-k_2)^2=r_2^2.$

Let the coordinates of $P$ be $(x,0).$ Then
$$f(P)=(r_1^2-((h_1-x)^2+k_1^2))-(r_2^2-((h_2-x)^2+k_2^2))$$
$$f(P)=r_1^2-r_2^2-(x^2-2h_1x+h_1^2+k_1^2)+(x^2-2h_2x+h_2^2+k_2^2)$$
$$f(P)=r_1^2-r_2^2+h_2^2-h_1^2+k_2^2-k_1^2+x(2h_2-2h_1).$$

Since all variables except for $x$ are constant, $f(P)$ varies linearly.
\end{pro}

Here's a really silly example of it.
\begin{exam}[MAST Diagnostic 2021/12]
In $\triangle ABC,$ let the foot of $B$ to $AC$ be $E$ and the foot of $C$ to $AB$ be $F.$ Suppose that the circle through $F$ centered at $B$ is externally tangent to the circle through $E$ centered at $C$ at some point $D.$ Let $G$ be the midpoint of $EF.$ Prove that $DG$ is perpendicular to $BC.$
\end{exam}

\begin{sol}
Let the circle centered at $B$ be $\omega_1$ and the circle centered at $C$ be $\omega_2.$, and let $f(P)=\pow(P,\omega_1)-\pow(P,\omega_2).$ Then by Linearity of Power, we want to show that $f(F)+f(E)=0.$ Note that $f(F)=\pow(F,\omega_1)-\pow(F,\omega_2)=-\pow(F,\omega_2)=-(FC^2-CE^2)=-(BC^2-BF^2-CE^2)=BF^2+CE^2-BC^2.$ Also note that $f(E)=\pow(E,\omega_1)-\pow(E,\omega_2)=\pow(E,\omega_1)=EB^2-BF^2=BC^2-CE^2-BF^2.$

Thus by Linearity of Power, $f(G)=0,$ implying that $G$ lies on the radical axis of $\omega_1,\omega_2.$ Since the radical axis is the line through $D$ perpendicular to $BC,$ we are done.
\end{sol}

This is all motivated by the fact that $G$ should lie on the radical axis of $\omega_1,\omega_2.$ The linearity of power solution comes quite naturally because of the right triangles created by the altitudes.

And here's an example of a problem solved using circles with radius $0.$

\begin{exam}[Iran TST 2011/1]
In acute triangle $ABC$, $\angle B$ is greater than $\angle C$. Let $M$ be the midpoint of $BC$ and let $D$ and $E$ be the feet of the altitudes from $C$ and $B,$ respectively. Let $K$ and $L$ be the midpoints of $ME$ and $MD$. If $KL$ intersects the line through $A$ parallel to $BC$ in $T$, prove that $TA=TM$.
\end{exam}

\begin{sol}
We claim that the line through $A$ parallel to $BC,$ $MD,$ and $ME$ are tangent to $(ADE).$ (This is known as the Three Tangent Lemma.)
\\

Proof: Let $H$ be the orthocenter of $\triangle ABC.$ Note that $AH$ is the diameter of $(ADE)$ as $\angle ADH=\angle AEH=90^{\circ}.$

Since $AH$ is perpendicular to $BC$ and the line through $A$ is parallel to $BC,$ it is a tangent.

To show $ME$ is tangent, we show $\angle DEM=\angle DAE=\angle A.$ Notice that $MB=MC=MD=ME,$ since $M$ is the center of $(BCDE).$

Notice that
\[\angle DEM=\angle DEB+\angle BEM=\angle DEB+\angle EBM=\angle DEB+\angle EBC\]
\[\angle DEB+\angle EBC=90^{\circ}-\angle C+90^{\circ}-\angle B=\angle A,\]
proving that $ME$ is tangent to $(AEF)$ as desired. $\blacksquare$
\\

Thus $KL$ is the radical axis of $(ADE)$ and the circle centered at $M$ with radius $0.$ Since $T$ lies on the radical axis and $TA$ is tangent to $(ADE),$ $TA=TM.$

\begin{center}
    \begin{asy}
    import olympiad;
    size(6cm);
    pair A=dir(125), B=dir(210), C=dir(-30),M,D,E,H,K,L,T,P,X;
    M=(B+C)/2;
    D=foot(B,A,C);
    E=foot(C,A,B);
    H=extension(B,D,C,E);
    K=(M+E)/2;
    L=(M+D)/2;
    P=foot(A,B,C);
    X=A+P-B;
    T=extension(A,X,K,L);
    
    draw(A--B--C--A);
    draw(D--M--E);
    draw(A--T--K);
    draw(A--H,blue+dotted);
    draw(circumcircle(A,D,E));
    draw(circumcircle(D,E,B),blue+dotted);
    dot("$A$", A, N);
dot("$B$", B, dir(200));
dot("$C$", C, dir(-20));
dot("$D$", D, NE);
dot("$E$", E, dir(210));
dot("$M$", M, S);
dot("$K$", K, N);
dot("$L$", L, dir(-30));
dot("$T$", T, NE);
dot("$H$", H, N);
    \end{asy}
\end{center}
\end{sol}

\pagebreak
\section{Problems}

\minpt{32}

\begin{prob}[]{1}
If circle $\omega$ with center $O$ has radius $3$ and $OP=5,$ find $\pow(P,\omega).$
\end{prob}

\begin{prob}[]{1}
Consider two externally tangent circles $\omega_1,\omega_2.$ Let them have common external tangents $AC,BD$ such that $A,B$ are on $\omega_1$ and $C,D$ are on $\omega_2.$ Let $AC$ intersect $BD$ at $P,$ and let the common internal tangent intersect $AC$ and $BD$ at $X$ and $Y$. If $\frac{[PCD]}{[PAB]}=\frac{1}{25},$ find $\frac{[PCD]}{[PXY]}.$
\end{prob}

\begin{prob}[Mandelbrot 2012]{1}
Let $A$ and $B$ be points on the lines $y=3$ and $y=12,$ respectively. There are two circles passing through $A$ and $B$ that are also tangent to the $x$ axis, say at $P$ and $Q.$ Suppose that $PQ=2012.$ Find $AB.$
\end{prob}

\begin{prob}[HMMT 2020/T3]{2}
Let $ABC$ be a triangle inscribed in a circle $\omega$ and $\ell$ be the tangent to $\omega$ at $A$. The line through $B$ parallel to $AC$ meets $\ell$ at $P$, and the line through $C$ parallel to $AB$ meets $\ell$ at $Q$. The circumcircles of $ABP$ and $ACQ$ meet at $S\neq A$. Show that $AS$ bisects $BC$.
\end{prob}

\begin{prob}[Geometry Bee 2019]{2}
Circles $O_1$ and $O_2$ are constructed with $O_1$ having radius of $2,$ $O_2$ having radius of $4,$ and $O_2$ passing through the point $O_1.$ Lines $\ell_1$ and $\ell_2$ are drawn so they are tangent to both $O_1$ and $O_2.$ Let $O_1$ and $O_2$ intersect at points $P$ and $Q.$ Segment $\overline{EF}$ is drawn through $P$ and $Q$ such that $E$ lies on $\ell_1$ and $F$ lies on $\ell_2.$ What is the length of $\overline{EF}?$
\end{prob}

\begin{req}[USAJMO 2012/1]{3}
Given a triangle $ABC$, let $P$ and $Q$ be points on segments $\overline{AB}$ and $\overline{AC}$, respectively, such that $AP=AQ$. Let $S$ and $R$ be distinct points on segment $\overline{BC}$ such that $S$ lies between $B$ and $R$, $\angle BPS=\angle PRS$, and $\angle CQR=\angle QSR$. Prove that $P,Q,R,S$ are concyclic (in other words, these four points lie on a circle).
\end{req}

\begin{prob}[MAST Diagnostic 2020]{3}
Consider $\triangle ABC,$ and let the feet of the $B$ and $C$ altitudes of the triangle be $X,Y.$ Let $XY$ intersect $BC$ at $P.$ Then prove that the circumcircles of $\triangle PBY$ and $\triangle PCX$ concur with $AP.$
\end{prob} 

\begin{prob}[GOTEEM 1]{4}
Let $ABC$ be a scalene triangle. The incircle of $\triangle ABC$ is tangent to sides $BC,$ $CA,$ $AB$ at $D,$ $E,$ and $F,$ respectively. Let $G$ be a point on the incircle of $\triangle ABC$ such that $\angle AGD = 90^{\circ}.$ If lines $DG$ and $EF$ intersect at $P,$ prove that $AP$ is parallel to $BC.$
\end{prob}

\begin{prob}[USAMO 2009/1]{4}
Given circles $\omega_1$ and $\omega_2$ intersecting at points $X$ and $Y$, let $\ell_1$ be a line through the center of $\omega_1$ intersecting $\omega_2$ at points $P$ and $Q$ and let $\ell_2$ be a line through the center of $\omega_2$ intersecting $\omega_1$ at points $R$ and $S$. Prove that if $P, Q, R$ and $S$ lie on a circle then the center of this circle lies on line $XY$.
\end{prob}

\begin{prob}[IMO 1995/1]{4}
Let $A,B,C,D$ be four distinct points on a line, in that order. The circles with diameters $AC$ and $BD$ intersect at $X$ and $Y$. The line $XY$ meets $BC$ at $Z$. Let $P$ be a point on the line $XY$ other than $Z$. The line $CP$ intersects the circle with diameter $AC$ at $C$ and $M$, and the line $BP$ intersects the circle with diameter $BD$ at $B$ and $N$. Prove that the lines $AM,DN,XY$ are concurrent.
\end{prob}

\begin{prob}[]{6}
Consider scalene $\triangle ABC$ with incenter $I.$ Let the $A$ excircle of $\triangle ABC$ intersect the circumcircle of $\triangle ABC$ at $X,Y.$ Let $XY$ intersect $BC$ at $Z.$ Then choose $M,N$ on the $A$ excircle of $\triangle ABC$ such that $ZM,ZN$ are tangent to the $A$ excircle of $\triangle ABC.$ Prove $I,M,N$ are collinear.
\end{prob}

\begin{prob}[AIME II 2010/15]{6}
In triangle $ABC$, $AC = 13$, $BC = 14$, and $AB=15$. Points $M$ and $D$ lie on $AC$ with $AM=MC$ and $\angle ABD = \angle DBC$. Points $N$ and $E$ lie on $AB$ with $AN=NB$ and $\angle ACE = \angle ECB$. Let $P$ be the point, other than $A$, of intersection of the circumcircles of $\triangle AMN$ and $\triangle ADE$. Ray $AP$ meets $BC$ at $Q$. The ratio $\frac{BQ}{CQ}$ can be written in the form $\frac{m}{n}$, where $m$ and $n$ are relatively prime positive integers. Find $m-n.$
\end{prob}

\begin{prob}[AIME I 2016/15]{6}
Circles $\omega_1$ and $\omega_2$ intersect at points $X$ and $Y$. Line $\ell$ is tangent to $\omega_1$ and $\omega_2$ at $A$ and $B$, respectively, with line $AB$ closer to point $X$ than to $Y$. Circle $\omega$ passes through $A$ and $B$ intersecting $\omega_1$ again at $D \neq A$ and intersecting $\omega_2$ again at $C \neq B$. The three points $C$, $Y$, $D$ are collinear, $XC = 67$, $XY = 47$, and $XD = 37$. Find $AB^2$.
\end{prob}

\begin{prob}[PUMaC 2017]{9}
Triangle $ABC$ has incenter $I$. The line through $I$ perpendicular to $AI$ meets the circumcircle of $ABC$ at points $P$ and $Q$, where $P$ and $B$ are on the same side of $AI$. Let $X$ be the point such that $PX \parallel CI$ and $QX \parallel BI$. Show that $PB, QC$, and $IX$ intersect at a common point.
\end{prob}

\begin{prob}[USAMTS 2018]{13}
Acute scalene triangle $\triangle ABC$ has circumcenter $O$ and orthocenter $H$. Points $X$ and $Y$, distinct from $B$ and $C$, lie on the circumcircle of $\triangle ABC$ such that $\angle BXH = \angle CYH = 90^\circ$. Show that if lines $XY$, $AH$, and $BC$ are concurrent, then $\overline{OH}$ is parallel to $\overline{BC}$.
\end{prob}
\end{document}