\documentclass[mast]{lucky}



\title{Radical Axes}
\author{Dennis Chen}
\date{GRU}

\begin{document}

\maketitle

\section{Power of a Point}
We first extend the power of a point theorem to a definition.
\begin{defi}[Power of a Point]
The power of a point $P$ with respect to circle $\omega$ with center $O$ and radius $r$ as $OP^2-r^2.$ We will denote this as $\pow(P,\omega)=OP^2-r^2.$
\end{defi}
This yields the following corollary.
\begin{fact}[Square of Tangent Line]
The power of a point is the square of the length of the tangent line.
\end{fact}
\section{Radical Axes}
\begin{defi}[Radical Axis]
The radical axis of a pair of circles $\omega_1,\omega_2$ as the locus of points such that $\pow(P,\omega_1)=\pow(P,\omega_2).$
\end{defi}

\begin{theo}[Radical Axis Theorem]
The radical axis of a pair of circles is a line.
\end{theo}

\begin{pro}
We use coordinates to prove this.

Without loss of generality, let the center of $\omega_1$ be $(0,0)$ and let the center of $\omega_2$ be $(x_{0},0).$ Denote the radii of $\omega_1,\omega_2$ as $r_1,r_2,$ respectively. If the coordinates of $P$ are $(x,y),$ then $(x^2+y^2)-r_1^2=([x-x_{0}]^2+y^2)-r_2^2.$ Rearranging, this yields $-r_1^2=-2x_{0}x+x^2_{0}-r_2^2.$ This is the equation of a line, as desired. 
\end{pro}

A corollary that arises: if $\omega_1,\omega_2$ intersect at $X,Y,$ then the radical axis is $XY.$ This is because $\pow(X,\omega_1)=0=\pow(X,\omega_2)$ and $\pow(Y,\omega_1)=0=\pow(Y,\omega_2).$ Since two points are needed to determine a line, the proof is done.

\begin{theo}[Radical Center Theorem]
Consider three circles $\omega_1,\omega_2,\omega_3.$ Then their pairwise radical axes either concur or are all parallel.
\end{theo}

\begin{pro}
Without loss of generality, let the radical axis of $\omega_1,\omega_3$ and the radical axis of $\omega_1,\omega_2$ intersect at $P.$\footnote{If they do not intersect, then all three radical axes are parallel; this is because the proof works regardless of which pair of radical axes you intersect as it is symmetric.} Then notice that $\pow(P,\omega_1)=\pow(P,\omega_3)$ and $\pow(P,\omega_1)=\pow(P,\omega_2),$ so $\pow(P,\omega_2)=\pow(P,\omega_3),$ implying that $P$ lies on the radical axis of $\omega_2,\omega_3,$ as desired.
\end{pro}

\section{Techniques}

\subsection{Basic Techniques}
Keeping this simple result in mind kills problems involving common chords and external tangents.

\begin{fact}[Radical Axis Bisects External Tangent]
Consider two circles $\omega_1, \omega_2$ that intersect at $X,Y.$ Let one of their common external tangents intersect $\omega_1$ at $A$ and $\omega_2$ at $B.$ Then $XY$ bisects $AB.$
\end{fact}

\begin{pro}
Note that $XY$ is the radical axis. Let $XY$ intersect $AB$ at $M.$ Since $M$ lies on the radical axis, $AM=BM.$
\end{pro}

\subsection{Advanced Techniques}
We introduce two powerful techniques - Linearity of Power and circles with radius $0.$
\begin{theo}[Linearity of Power]
The function $f(P)=\pow(P,\omega_1)-\pow(P,\omega_2)$ changes at a linear rate as $P$ moves along a fixed line $\ell.$
\end{theo}

\begin{pro}
Without loss of generality, let $\ell$ be the $x$ axis, let the equation of $\omega_1$ be $(x-h_1)^2+(y-k_1)^2=r_1^2,$ and let the equation of $\omega_2$ be $(x-h_2)^2+(y-k_2)^2=r_2^2.$

Let the coordinates of $P$ be $(x,0).$ Then
$$f(P)=(r_1^2-((h_1-x)^2+k_1^2))-(r_2^2-((h_2-x)^2+k_2^2))$$
$$f(P)=r_1^2-r_2^2-(x^2-2h_1x+h_1^2+k_1^2)+(x^2-2h_2x+h_2^2+k_2^2)$$
$$f(P)=r_1^2-r_2^2+h_2^2-h_1^2+k_2^2-k_1^2+x(2h_2-2h_1).$$

Since all variables except for $x$ are constant, $f(P)$ varies linearly.
\end{pro}

Here's a really silly example of it.
\begin{exam}[MAST Diagnostic 2021/12]
In $\triangle ABC,$ let the foot of $B$ to $AC$ be $E$ and the foot of $C$ to $AB$ be $F.$ Suppose that the circle through $F$ centered at $B$ is externally tangent to the circle through $E$ centered at $C$ at some point $D.$ Let $G$ be the midpoint of $EF.$ Prove that $DG$ is perpendicular to $BC.$
\end{exam}

\begin{sol}
Let the circle centered at $B$ be $\omega_1$ and the circle centered at $C$ be $\omega_2.$, and let $f(P)=\pow(P,\omega_1)-\pow(P,\omega_2).$ Then by Linearity of Power, we want to show that $f(F)+f(E)=0.$ Note that $f(F)=\pow(F,\omega_1)-\pow(F,\omega_2)=-\pow(F,\omega_2)=-(FC^2-CE^2)=-(BC^2-BF^2-CE^2)=BF^2+CE^2-BC^2.$ Also note that $f(E)=\pow(E,\omega_1)-\pow(E,\omega_2)=\pow(E,\omega_1)=EB^2-BF^2=BC^2-CE^2-BF^2.$

Thus by Linearity of Power, $f(G)=0,$ implying that $G$ lies on the radical axis of $\omega_1,\omega_2.$ Since the radical axis is the line through $D$ perpendicular to $BC,$ we are done.
\end{sol}

This is all motivated by the fact that $G$ should lie on the radical axis of $\omega_1,\omega_2.$ The linearity of power solution comes quite naturally because of the right triangles created by the altitudes.

And here's an example of a problem solved using circles with radius $0.$

\begin{exam}[Iran TST 2011/1]
In acute triangle $ABC$, $\angle B$ is greater than $\angle C$. Let $M$ be the midpoint of $BC$ and let $D$ and $E$ be the feet of the altitudes from $C$ and $B,$ respectively. Let $K$ and $L$ be the midpoints of $ME$ and $MD$. If $KL$ intersects the line through $A$ parallel to $BC$ in $T$, prove that $TA=TM$.
\end{exam}

\begin{sol}
We claim that the line through $A$ parallel to $BC,$ $MD,$ and $ME$ are tangent to $(ADE).$ (This is known as the Three Tangent Lemma.)
\\

Proof: Let $H$ be the orthocenter of $\triangle ABC.$ Note that $AH$ is the diameter of $(ADE)$ as $\angle ADH=\angle AEH=90^{\circ}.$

Since $AH$ is perpendicular to $BC$ and the line through $A$ is parallel to $BC,$ it is a tangent.

To show $ME$ is tangent, we show $\angle DEM=\angle DAE=\angle A.$ Notice that $MB=MC=MD=ME,$ since $M$ is the center of $(BCDE).$

Notice that
\[\angle DEM=\angle DEB+\angle BEM=\angle DEB+\angle EBM=\angle DEB+\angle EBC\]
\[\angle DEB+\angle EBC=90^{\circ}-\angle C+90^{\circ}-\angle B=\angle A,\]
proving that $ME$ is tangent to $(AEF)$ as desired. $\blacksquare$
\\

Thus $KL$ is the radical axis of $(ADE)$ and the circle centered at $M$ with radius $0.$ Since $T$ lies on the radical axis and $TA$ is tangent to $(ADE),$ $TA=TM.$

\begin{center}
    \begin{asy}
    import olympiad;
    size(6cm);
    pair A=dir(125), B=dir(210), C=dir(-30),M,D,E,H,K,L,T,P,X;
    M=(B+C)/2;
    D=foot(B,A,C);
    E=foot(C,A,B);
    H=extension(B,D,C,E);
    K=(M+E)/2;
    L=(M+D)/2;
    P=foot(A,B,C);
    X=A+P-B;
    T=extension(A,X,K,L);
    
    draw(A--B--C--A);
    draw(D--M--E);
    draw(A--T--K);
    draw(A--H,blue+dotted);
    draw(circumcircle(A,D,E));
    draw(circumcircle(D,E,B),blue+dotted);
    dot("$A$", A, N);
dot("$B$", B, dir(200));
dot("$C$", C, dir(-20));
dot("$D$", D, NE);
dot("$E$", E, dir(210));
dot("$M$", M, S);
dot("$K$", K, N);
dot("$L$", L, dir(-30));
dot("$T$", T, NE);
dot("$H$", H, N);
    \end{asy}
\end{center}
\end{sol}

\pagebreak
\section{Problems}
\minpt{TBD}

\begin{prob}[]{1}
Consider a point $P$ with power $36$ respective to a circle with center $O.$ If $PO=10,$ find the radius of the circle.
\end{prob}

\begin{prob}[]{2}
Consider a coordinate plane with two circles tangent to the $x$ axis at $X,Y,$ respectively. If the circles intersect at $P,Q,$ and $XY=8,$ is it possible for $P$ to lie on $y=3$ and $Q$ to lie on $y=12?$
\end{prob}

\begin{prob}[APMO 2020/1]{3}
Let $\Gamma$ be the circumcircle of $\triangle ABC$. Let $D$ be a point on the side $BC$. The tangent to $\Gamma$ at $A$ intersects the parallel line to $BA$ through $D$ at point $E$. The segment $CE$ intersects $\Gamma$ again at $F$. Suppose $B$, $D$, $F$, $E$ are concyclic. Prove that $AC$, $BF$, $DE$ are concurrent.

\end{prob}

\begin{prob}[EGMO 2019/4]{3}
Let $ABC$ be a triangle with incentre $I$. The circle through $B$ tangent to $AI$ at $I$ meets side $AB$ again at $P$. The circle through $C$ tangent to $AI$ at $I$ meets side $AC$ again at $Q$. Prove that $PQ$ is tangent to the incircle of $ABC.$
\end{prob}

\begin{prob}[IMO 2004/1]{4}
Let $ABC$ be an acute-angled triangle with $AB\neq AC$. The circle with diameter $BC$ intersects the sides $AB$ and $AC$ at $M$ and $N$ respectively. Denote by $O$ the midpoint of the side $BC$. The bisectors of the angles $\angle BAC$ and $\angle MON$ intersect at $R$. Prove that the circumcircles of the triangles $BMR$ and $CNR$ have a common point lying on the side $BC$.
\end{prob}

\begin{prob}{4}
Let $P_1P_2P_3\dots P_{2n}$ be a regular $2n$-gon, and let $X$ be a point. Prove \[(P_1P_{n+1}X), (P_2P_{n+2}X), (P_3P_{n+3}X),\dots (P_nP_{2n}X)\] are concurrent.
\end{prob}

\begin{prob}[ISL 2017/G1]{6}
Let $ABCDE$ be a convex pentagon such that $AB=BC=CD$, $\angle{EAB}=\angle{BCD}$, and $\angle{EDC}=\angle{CBA}$. Prove that the perpendicular line from $E$ to $BC$ and the line segments $AC$ and $BD$ are concurrent.
\end{prob}

\begin{prob}[Swiss TST 2018/3]{6}
Let $ABC$ be a triangle with $\angle  ABC \ne \angle  BCA$. The inscribed circle $k$ of the triangle $ABC$ is tangent to the sides $BC, CA,AB$ at points $D, E , F$ respectively. The segment $AD$ intersects $k$ again at $P$. Let $Q$ be the point of intersection of $EF$ with the line perpendicular to $AD$ passing through $P$. Let $X,Y$ be the points of intersection of $AQ$ with $DE,DF$ respectively. Prove that $A$ is the midpoint of the segment $XY$.
\end{prob}

\begin{prob}[ISL 2017 G5]{9}
Let $ABCC_1B_1A_1$ be a convex hexagon such that $AB=BC$, and suppose that the line segments $AA_1, BB_1$, and $CC_1$ have the same perpendicular bisector. Let the diagonals $AC_1$ and $A_1C$ meet at $D$, and denote by $\omega$ the circle $ABC$. Let $\omega$ intersect the circle $A_1BC_1$ again at $E \neq B$. Prove that the lines $BB_1$ and $DE$ intersect on $\omega$.
\end{prob}

\end{document}