\documentclass{article}
\usepackage[mast]{lucky}

\title{Just Do It}
\author{Dennis Chen}
\date{MQV}

\begin{document}
\maketitle
There are two categories of problems in lower level math contests, particularly the AMCs; one of them is where you solve the problem by reducing it to something easy and solving said easy problem. The other type is where you ``just do it.'' It is the second which we analyze.

Some hard-seeming problems can die to a technique known as ``just do it,'' where you stop being scared by the problem and just do it. These problems are not by any means easy, otherwise they would seem more approachable, but the purpose is to show that some hard-seeming problems are actually approachable. Alternatively, there are some easy problems that you just have to buckle down and do; out of the first 10 on the AMCs, most of them fall into ``Just Do It.''

Most of these problems tend to be sequence problems or counting problems, and often they will be sequence problems that are also counting problems.

\section{Examples}
In the interest of making you just do it, I will only be outlining what you should do to get started.

\begin{exam}[CIME I 2021/3]
For any positive integer $n$, let $r(n)$ denote the "reverse" of $n$. For example, $r(123)=321$ and $r(2020)=0202=202.$ It is given that there exists a unique set of 3-digit positive integers $\{p, r(p)\} \neq \{273,372\}$ for which $p \cdot r(p) = 273 \cdot 372.$ Find $p+r(p).$
\end{exam}

\begin{walk}
\begin{enumerate}
\item Let the numbers be $100a+10b+c$ and $100c+10b+a.$

\item Note that $ac\equiv 6\pmod{10}$ because $273\cdot 372\equiv 6\pmod{10}.$

\item By size reasons, conclude that $ac=6.$

\item Find $a$ and $c.$

\item Take mod $100$ to find $b.$
\end{enumerate}
\end{walk}

\begin{exam}[MAST Diagnostic 2020]
A secret spy organization needs to spread some secret knowledge to all of its members. In the beginning, only $1$ member is \textit{informed}. Every informed spy will call an uninformed spy such that every informed spy is calling a different uninformed spy. After being called, an uninformed spy becomes informed. The call takes $1$ minute, but since the spies are running low on time, they call the next spy directly afterward. However, to avoid being caught, after the third call an informed spy makes, the spy stops calling. How many minutes will it take for every spy to be informed, provided that the organization has $600$ spies?
\end{exam}

\begin{walk}
Keep track of the number of spies that have made no calls, one call, and two calls.
\end{walk}

\begin{exam}[NARML/9]
Let $\mathcal{H}(x)$ be a function on the positive integers such that
\begin{itemize}
\Item $\mathcal{H}(1)=1,$ and
\Item for integers $n>1,$ $2\mathcal{H}(n)=\sum\limits_{i\mid n}\mathcal{H}(i).$
\end{itemize}
Compute the smallest positive integer $n>1$ that satisfies $\mathcal{H}(n)=n.$
\end{exam}
\begin{walk}
This function does not distinguish between primes. Use this symmetric structure to start making some general observations.
\end{walk}

\begin{exam}[CMC 10A 2021/20]
Pyramid $ABCDE$ with square base $BCDE$ and apex $A$ has edge lengths all equal to 1 with the square base sitting atop a table. Some of the edges of the pyramid are cut so that it is opened out flat to its net (in one piece) on the table without any folds. Two possibilities for the net are shown below, and nets are considered the same if they differ by a rotation but different if they differ by a reflection. How many distinct nets are possible?

\begin{center}
\begin{asy}
pen thmred=RGB(230, 190, 190);

size(6cm);
pair A=(1,1),B=(-1,1),C=(-1,-1),D=(1,-1);
pair E=(1+sqrt(3),0),F=(0,1+sqrt(3)),G=(-1-sqrt(3),0),H=(0,-1-sqrt(3));
filldraw(A--B--C--D--cycle, thmred);
draw(E--A--F--B--G--C--H--D--cycle);

pair W=(8,1),X=(6,1),Y=(6,-1),Z=(8,-1);
pair P=(9,1+sqrt(3)),Q=(7,1+sqrt(3)),R=(6-sqrt(3),0),S=(7,-1-sqrt(3));
filldraw(W--X--Y--Z--cycle, thmred);
draw(P--W--Q--X--R--Y--S--Z);
draw(P--Q);
\end{asy}
\end{center}
\end{exam}
\begin{walk}
Notice that of the $8$ edges of the pyramid, we want to keep $4$ of them such that all the planes are connected to each other. The rotation but no reflection condition suggests that it's best to take cases based on how many of these edges belong to the square; now just do the counting.
\end{walk}

\pagebreak\section{Problems}

\minpt{TBD}

\psetquote{If you don’t act yourself, that chance won’t come!}{Code Geass}

\begin{prob}[AMC 10A 2021/9]{2}
What is the least possible value of $(xy-1)^2+(x+y)^2$ for real numbers $x$ and $y$?
\end{prob}

\begin{prob}[AIME II 2020/6]{4}
Define a sequence recursively by $t_1 = 20$, $t_2 = 21$, and$$t_n = \frac{5t_{n-1}+1}{25t_{n-2}}$$for all $n \ge 3$. Then $t_{2020}$ can be written as $\frac{p}{q}$, where $p$ and $q$ are relatively prime positive integers. Find $p+q$.
\end{prob}

\begin{prob}[CMC 10A 2021/13]{6}
A natural number is called \textit{in-2-itive} if it is a perfect square and each of its digits is chosen from the set $\{1,2,4,8\}.$ What is the sum of the digits of the largest \textit{in-2-itive} number less than $10,000?$
\end{prob}

\begin{req}[AMC 10A 2021/22]{6}
Hiram's algebra notes are $50$ pages long and are printed on $25$ sheets of paper; the first sheet contains pages $1$ and $2$, the second sheet contains pages $3$ and $4$, and so on. One day he leaves his notes on the table before leaving for lunch, and his roommate decides to borrow some pages from the middle of the notes. When Hiram comes back, he discovers that his roommate has taken a consecutive set of sheets from the notes and that the average (mean) of the page numbers on all remaining sheets is exactly $19$. How many sheets were borrowed?
\end{req}

\begin{prob}[CMC 10A 2021/19]{6}
A positive integer is called $\textit{condescending}$ if each digit is nonzero and strictly larger than the sum of all of the digits to its right. For example, $8421$, $95$, and $6$ are condescending, but $5321$ is not. How many positive integers are condescending?
\end{prob}

\begin{prob}[MAST Diagnostic 2021/12]{9}
Let $f(x)=x^2-12x+36.$ In terms of $k$, for $k\geq 2,$ find the sum of all real $n$ such that $f^{k-1}(n)=f^k(n).$
\end{prob}
\end{document}