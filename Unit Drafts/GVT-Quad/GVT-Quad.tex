\documentclass{article}
\usepackage[mast]{lucky}

\title{Cyclic Quadrilaterals}
\author{Dennis Chen}
\date{GVT}

\begin{document}
\maketitle

\section{Theory}

\section{Examples}

\pagebreak

\section{Problems}

\minpt{TBD}

\begin{prob}[AIME II 2011/10]{3}
A circle with center $O$ has radius 25. Chord $\overline{AB}$ of length 30 and chord $\overline{CD}$ of length 14 intersect at point $P$. The distance between the midpoints of the two chords is 12. The quantity $OP^2$ can be represented as $\frac{m}{n}$, where $m$ and $n$ are relatively prime positive integers. Find the remainder where $m+n$ is divided by 1000.
\end{prob}

\begin{prob}[NIMO Janurary 2013/8]{4}
Let $AXYZB$ be a convex pentagon inscribed in a semicircle with diameter $AB$. Suppose that $AZ-AX=6$, $BX-BZ=9$, $AY=12$, and $BY=5$. Find the greatest integer not exceeding the perimeter of quadrilateral $OXYZ$, where $O$ is the midpoint of $AB$.
\end{prob}

\begin{prob}[AMC 12A 2017/24]{6}
Quadrilateral $ABCD$ is inscribed in circle $O$ and has sides $AB = 3$, $BC = 2$, $CD = 6$, and $DA = 8$. Let $X$ and $Y$ be points on $\overline{BD}$ such that
\[\frac{DX}{BD} = \frac{1}{4} \quad \text{and} \quad \frac{BY}{BD} = \frac{11}{36}.\]Let $E$ be the intersection of intersection of line $AX$ and the line through $Y$ parallel to $\overline{AD}$. Let $F$ be the intersection of line $CX$ and the line through $E$ parallel to $\overline{AC}$. Let $G$ be the point on circle $O$ other than $C$ that lies on line $CX$. What is $XF \cdot XG$?
\end{prob}

\end{document}