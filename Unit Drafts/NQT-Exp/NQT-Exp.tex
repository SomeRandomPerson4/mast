\documentclass[mast]{lucky}

\title{Exponents in modular arithmetic}
\author{Dennis Chen, Kelin Zhu}
\date{NQT}

\begin{document}
\maketitle
\section{Fermat's Little Theorem}

Often in number theory problems we will want to take some number to some large power and find its remainder when divided by another number. Fermat's Little Theorem provides a way to make this calculation much easier.

\begin{theo}[Fermat's Little Theorem]
Consider a prime $p.$ For any integer $a$, $a^{p}\equiv a \pmod {p}.$
\end{theo}

If we add the restriction that $a$ and $p$ are relatively prime, we can divide by $a$ on both sides to achieve the following.

\begin{theo}[Fermat's Little Theorem, alternative]
Consider a prime $p.$ For relatively prime $a, p$, $a^{p-1}\equiv 1 \pmod {p}.$
\end{theo}

There are two proofs for this theorem. We present the induction proof first because it requires the least amount of ingenuity.
\begin{pro}[1 (Induction)]
For the inductive proof, we prove that $a^p\equiv a\pmod{p}$.

This is obviously true for the base case $a=1.$

Now assume that this is true for $a=n.$ Then
\[(n+1)^p\equiv n^p+\binom{p}{1}n^{p-1}+\binom{p}{2}n^{p-2}+\cdots+1\pmod{p}.\]
But notice that $\binom{p}{1},\binom{p}{2}\dots\binom{p}{p-1}$ are all divisible by $p,$ so
\[n^p+\binom{p}{1}n^{p-1}+\binom{p}{2}n^{p-2}+\cdots+1\equiv n^p+1\equiv n+1 \pmod{p},\]
as desired.
\end{pro}
The rearrangement proof requires a little bit more creativity and is more aesthetic. It can also be generalized to Euler's Theorem, where the first proof cannot.
\begin{pro}[2 (Rearrangement)]
Suppose that $a$ and $p$ are relatively prime. We claim that $a,2a,3a\dots a(p-1)$ is a rearrangement of $1,2,3\dots p-1$ taken mod $p.$ We prove this by contradiction. Assume that there are two integers such that $ax\equiv ay \pmod{p}$ with $0 < x, y < p$ and $x \neq y$. Since $\gcd(a,p)=1,$ we can divide both sides by $a$ to yield $x\equiv y.$ But this is obviously not possible. Thus, contradiction.

This implies that $(p-1)!\equiv a^{p-1}(p-1)! \pmod{p}.$ As $\gcd(p,(p-1)!)=1,$ we can divide both sides by $(p-1)!$ to get $1\equiv a^{p-1}\pmod{p},$ as desired.
\end{pro}

Here are some basic examples of Fermat's Little Theorem.

\begin{exam}
Find the remainder of $2^{20} + 3^{30} + 4^{40} + 5^{50} + 6^{60}$ when divided by $7.$
\end{exam}

\begin{sol}
Note that
\begin{align*}
2^{20} + 3^{30} + 4^{40} + 5^{50} + 6^{60}&\equiv 2^2+3^0+4^4+5^2+6^0\\
&\equiv 4+1+256+25+1 \\
&\equiv 0\pmod{7}.
\end{align*}
\end{sol}

\begin{exam}[AMC 12A 2008/15]
Let $k={2008}^{2}+{2}^{2008}$. What is the units digit of $k^2+2^k$?
\end{exam}

\begin{sol}
This is a mixture of Chinese Remainder Theorem and Fermat's Little Theorem.

Obviously $2\mid k^2+2^k,$ so we can consider its remainder when divided by $5.$ Now note that
\[k\equiv {2008}^{2}+{2}^{2008}\equiv 3^2+2^0\equiv 0\pmod{5}.\]
Since $5\mid k$ and $4\mid k,$
\[k^2+2^k\equiv 2^k\equiv 1\pmod{5}.\]
Now by CRT Congruences,
\[0\equiv 6\pmod{2}\]
\[1\equiv 6\pmod{5},\]
so the remainder when divided by $10$ is $6$ as well.
\end{sol}

Here's a pedagogical example of Fermat's and Chinese Remainder Theorem.

\begin{exam}[42 PMO Level 3 2020/1]
Let $p_1,p_2,\ldots,p_n$ be distinct prime numbers and $P$ be their product. Let the number $S$ be defined as
\[S=\sum\limits_{i=1}^{n}\left(\frac{P}{p_i}\right)^{p_i-1}.\]
Show that $S-1$ is a multiple of $P.$
\end{exam}

\begin{sol}
Note that $S\equiv \left(\frac{P}{p_k}\right)^{p_k-1}\equiv 1\pmod{p_k}$ by Fermat's Little Theorem, so $S\equiv 1\pmod{P}$ by CRT.
\end{sol}

\subsection{Monic Monomials mod n}
The important idea is that monic monomials \textbf{don't always cover all residues} mod $n$ for certain $n.$

We begin with perhaps the most famous and often used example.

\begin{exam}[Quadratic Residues mod 4]
The only positive integers $0\leq k<4$ such that $n^2\equiv k\pmod{4}$ has a solution are $k=0$ and $k=1.$
\end{exam}

In other words, the remainder of a square divided by $4$ is always $0$ or $1.$

\begin{sol}
Just plug in $0,1,2,3$ and check only $0,1$ are outputted.
\end{sol}

\begin{exam}[Macedonia JBMO TST 2016/1]
Solve
\[x_{1}^4 + x_{2}^4 +...+ x_{14}^4=2016^3 - 1\]
over the integers.
\end{exam}

\begin{sol}
There are no solutions. Take mod $16$ and note that $x^4\in \{0,1\}\pmod{16}$ and $2016^3-1\equiv 15\pmod{16}.$
\end{sol}

With practice and experience, these patterns will become clearer over time. For instance, fourth powers strongly suggest taking mod $16,$ and squares strongly suggest taking mod $4,$ mod $3,$ or mod $8,$ depending on the context.

Speaking of squares and mod $3\ldots$

\begin{exam}[USAJMO 2011/1]
Find, with proof, all positive integers $n$ for which $2^n + 12^n + 2011^n$ is a perfect square.
\end{exam}

\begin{sol}
We claim the only solution is $n=1,$ since $2+12+2011=2025=45^2.$

Now we claim that no solutions $n>1$ work. Assume otherwise. Then note that for mod $3$ reasons, we must have $n$ be odd since $2^n+2011^n\equiv 2^n+1^n\equiv 2^n+1\pmod{3},$ and we must have it be congruent to $0$ mod $3.$ But also note for mod $4$ reasons, we must have $n$ be even since $2011^n\equiv (-1)^n\pmod{4},$ and we must have it be congruent to $1$ mod $4,$ contradiction.
\end{sol}

\subsection{Exponential Functions mod n}
You will also want to consider exponential functions mod $n.$ The idea is you can use Fermat's Little Theorem to figure out the residues the function is restricted to.

We start with an easier problem as an example.

\begin{exam}[April USAJMO 2020/1]
Determine, with proof, whether there exists a positive integer $n$ such that $4^n-1$ divides $5^n-1.$
\end{exam}

\begin{sol}
Clearly $3\mid 4^n-1$ so we must have $3\mid 5^n-1,$ implying that $2\mid n.$ But also note that $2\mid n$ implies $5\mid 4^n-1,$ contradiction.
\end{sol}
Exponential functions usually go hand in hand with bounding and size arguments.

\begin{exam}[TurtleKing123]
Find all ordered triplets of natural numbers $(a, b, c)$ such that
\[6^a+7^b=13^c.\]
\end{exam}

The proof of the $a=1$ case is better suited for the Prime Factorization unit, but the rest is quite informative for this unit.

\begin{sol}
We claim the answer is just $a=b=c=1,$ and it is easy to check that it works.

Assume $a>2.$ The cases $a=1$ and $a=2$ will be shown later.

Taking mod $4$ gives us that $b$ is even, and then taking mod $8$ gives $c$ is even as well. Let $2x=b$ and $2y=c.$ Then
\[(13^x-7^y)(13^x+7^y)=6^a.\]
Note that for mod $3$ reasons all powers of $3$ are concentrated in $13^x-7^y,$ so $13^x+7^y$ must be of the form $2^k.$ But this implies that $3^k\mid 13^x-7^y,$ absurd for size reasons.

For $a=2$ note that taking mod $14$ gives that $b$ is even and $c$ is odd, but mod $14$ gives that $b$ is odd and $c$ is even, contradiction.

For $a=1,$ subtract $13$ to get
\[7^b-7=13^c-13.\]

The first main claim is that $43\mid 7^b-7.$ Note that $12\mid b-1$ since the smallest number $k$ such that $7^{k}\equiv 1\pmod{13}$ is $12,$ so $43\mid 7^{12}-1\mid 7^b-7.$ 

The second main step is proving $14\mid c-1.$ Note that $7\mid k$ where $k$ is the smallest number such that $13^k\equiv 1\pmod{43},$ since $13^6\equiv 6\pmod{43},$ implying $7\mid c-1.$ Also for mod $7$ reasons, $2\mid c-1,$ which implies that $14\mid c-1.$ Then note that by LTE, $49\mid 13^{14}-1,$ so $49\mid 13^c-13.$ Thus $49\mid 7^b-7,$ which is only possible if $b=1.$
\end{sol}
Now we present a much harder example. If you don't know what the order of a number modulo a prime is, the idea is that $\ord_{p}n$ is the smallest integer $k$ such that $n^k\equiv 1\pmod{p}.$\footnote{I also recommend you skip this example for now if you don't know what order is.}

\begin{exam}[SJMO 2020/1]
Find all positive integers $k\geq 2$ for which there exists some positive integer $n$ such that the last $k$ digits of the decimal representation of $10^{10^n}-9^{9^n}$ are the same.
\end{exam}

\begin{sol}
We claim only $2\leq k\leq 4$ works. It suffices to just provide a construction for $k=4,$ and said construction is $n=25.$ We verify this works later in the proof.

We show that the only way to have the last two digits be the same is for them to both be $11.$ Note
\[10^{10^n}-9^{9^n}\equiv -9^{9^n}\equiv -1\pmod{4}\]
\[10^{10^n}-9^{9^n}\equiv -9^{9^n}\pmod{25}.\]
Now note that $9^{10}\equiv 1\pmod{25}$ by Euler's Theorem, so $9^{9^n}\equiv 9^{(-1)^n}\equiv 9,14\pmod{25}.$ Thus the only possible remainders of $-9^{9^n}$ divided by $100$ are $-9,-89,$ or $91,11.$ So $11$ is the only possible ending $2$ digits.

Now this implies we want to solve $10^{10^n}-9^{9^n}\equiv -9^{9^n}\equiv 10^0+10^1+\cdots+10^{k-1}\pmod{10^k}.$ This is valid because if $10^{10^n}\leq 10^{k-1}$ this is obviously not true. In other words,
\[9^{9^n}\equiv 8(10^0+10^1+\cdots+10^{k-1})+1\equiv \frac{8(10^k-1)}{9}+1\equiv \frac{1}{9}\pmod {10^k},\]
or $9^{9^n+1}\equiv 1\pmod{10^k}.$

Now we show that $n=25$ works for $k\geq 4.$ Note that the smallest number $k$ such that $3^k\equiv 1\pmod{2^4}$ is $k=4,$ and $4\mid 2(9^{25}+1)$ and $\phi(5^4)=4\cdot 5^3,$ and $4\cdot 5^3\mid 2(9^{25}+1),$ where $4\mid 2(9^{25}+1$ is obvious and $5^3\mid 9^{25}+1$ follows from LTE as $\nu_5(9^{25}+1)=\nu_5(9+1)+\nu_5(25)=3.$

Now we prove $k>4$ doesn't work. Note that the minimal $k$ such that $3^k\equiv 1\pmod{2^5}$ is $k=8,$ and $8\nmid 2(9^n+1)$ as $9^n+1\equiv 2\pmod{4}.$ Thus there are no solutions to $9^{9^n+1}\equiv 1\pmod{2^k}$ for $k>4.$
\end{sol}

\section{Euler's Totient Function}

Now we take a look at Euler's Totient Function.

\begin{defi}[Euler's Totient Function]
We define $\phi(n)$ to be the number of positive integers less than or equal to $n$ that are also relatively prime to $n.$
\end{defi}

\begin{theo}[Multiplicativity]
For relatively prime $m,n,$ $\phi(m)\cdot\phi(n)=\phi(mn).$
\end{theo}

The basic idea of the proof is just using CRT to find all possible congruences.

\begin{pro}
Note that we can have
\[x\equiv j_{1},j_{2},\ldots,j_{\phi(m)}\pmod{m}\]
\[x\equiv k_{1},k_{2},\ldots,k_{\phi(n)}\pmod{n},\]
where $j_i$ encompasses the numbers between $1$ and $m$ relatively prime to $m$ and $k_i$ encompasses the number between $1$ and $n$ relatively prime to $n.$ Then note that this system has $\phi(m)\phi(n)$ solutions when taken mod $mn.$
\end{pro}

\begin{theo}[Product Formula]
Say the prime factorization of positive integer $n$ is $p_1^{e_1}\cdot p_2^{e_2}\dots p_k^{e_k}.$ Then $\phi(n)=n\frac{p_1-1}{p_1}\cdot \frac{p_2-1}{p_2}\dots \frac{p_k-1}{p_k}.$
\end{theo}

\begin{pro}
Instead of actually proving this fully, we outline it for the reader to work through themselves.
\begin{enumerate}
\item Find $\phi(p_1^{e_1}).$
\item Use multiplicity to find $\phi(p_1^{e_1})\phi(p_2^{e_2})\cdots\phi(p_k^{e_k}).$
\end{enumerate}
\end{pro}

\begin{theo}[Euler's Totient Theorem]
For relatively prime $a,n,$ $a^{\phi(n)}\equiv 1\pmod {n}.$
\end{theo}

\begin{pro}
This is very similar to the rearrangement proof for Fermat's Little Theorem.\footnotemark

Let the set of positive integers less than and relatively prime to $n$ be $x_1,x_2,\ldots,x_{\phi(n)}.$ Then note that $ax_1,ax_2,\ldots,ax_{\phi(n)}$ is a rearrangement of $x_1,x_2,\ldots,x_{\phi(n)}.$

We proceed by contradiction. Assume that there are two integers such that $ax\equiv ay \pmod{n}.$ Since $\gcd(a,n)=1,$ we can divide both sides by $a$ to yield $x\equiv y.$ But this is obviously not possible. Thus, contradiction.

This implies that $x_1x_2\ldots x_{\phi(n)}\equiv (x_1x_2\ldots x_{\phi(n)})a^{\phi(n)}\pmod{n}.$ Dividing both sides by $x_1x_2\ldots x_{\phi(n)}$ yields $a^{\phi(n)}\equiv 1\pmod{n},$ as desired.
\end{pro}

\footnotetext{So similar, in fact, that I copy-pasted the proof for Fermat's and made minor adjustments.}

Also, notice that Fermat's is just a special case of Euler's.
\section{Problems}
\minpt{TBD}

\psetquote{I have a truly marvelous demonstration of this proposition which this margin is too narrow to contain.}{Pierre de Fermat}
\subsection{Period of a Repeating Decimal}
\begin{prob}[]{2}
The expansion of $\frac{1}{7}$ is $0.\overline{142857},$ which is a repeating decimal with a $6$ digit long sequence. How many digits long is the expansion of $\frac{1}{13}?$
\end{prob}

\begin{prob}[]{2}
We define the cycle of a repeating fraction $\tfrac{m}{n}$ as the minimum number $i$ such that $\tfrac{m}{n} = 0.\overline{a_1a_2a_3\dots a_i}$. Find the cycle of $\tfrac{1}{23}$.
\end{prob}

\begin{prob}[AMC 10A 2019/18]{3}
For some positive integer $k$, the repeating base-$k$ representation of the (base-ten) fraction $\frac{7}{51}$ is $0.\overline{23}_k = 0.232323\ldots_k$. What is $k$?
\end{prob}

\begin{req}[e-dchen Mock MATHCOUNTS]{4}
What is the sum of all odd $n$ such that $\frac{1}{n}$ expressed in base $8$ is a repeating decimal with period $4?$
\end{req}

\begin{prob}[AMC 12A 2014/23]{6}
The fraction\[\dfrac1{99^2}=0.\overline{b_{n-1}b_{n-2}\ldots b_2b_1b_0},\]where $n$ is the length of the period of the repeating decimal expansion. What is the sum $b_0+b_1+\cdots+b_{n-1}$?
\end{prob}

\begin{prob}[AMC 12B 2016/22]{6}
For a certain positive integer $n$ less than $1000$, the decimal equivalent of $\frac{1}{n}$ is $0.\overline{abcdef}$, a repeating decimal of period $6$, and the decimal equivalent of $\frac{1}{n+6}$ is $0.\overline{wxyz}$, a repeating decimal of period $4$. Find $n.$
\end{prob}
\end{document}