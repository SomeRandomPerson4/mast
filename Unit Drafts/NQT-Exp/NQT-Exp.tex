\documentclass[mast]{lucky}

\title{Exponents in modular arithmetic}
\author{Dennis Chen, Kelin Zhu}
\date{NQT}

\begin{document}
\maketitle

\section{Problems}
\subsection{Period of a Repeating Decimal}
\begin{prob}[]{2}
The expansion of $\frac{1}{7}$ is $0.\overline{142857},$ which is a repeating decimal with a $6$ digit long sequence. How many digits long is the expansion of $\frac{1}{13}?$
\end{prob}

\begin{prob}[]{2}
We define the cycle of a repeating fraction $\tfrac{m}{n}$ as the minimum number $i$ such that $\tfrac{m}{n} = 0.\overline{a_1a_2a_3\dots a_i}$. Find the cycle of $\tfrac{1}{23}$.
\end{prob}

\begin{prob}[AMC 10A 2019/18]{3}
For some positive integer $k$, the repeating base-$k$ representation of the (base-ten) fraction $\frac{7}{51}$ is $0.\overline{23}_k = 0.232323\ldots_k$. What is $k$?
\end{prob}

\begin{req}[e-dchen Mock MATHCOUNTS]{4}
What is the sum of all odd $n$ such that $\frac{1}{n}$ expressed in base $8$ is a repeating decimal with period $4?$
\end{req}

\begin{prob}[AMC 12A 2014/23]{6}
The fraction\[\dfrac1{99^2}=0.\overline{b_{n-1}b_{n-2}\ldots b_2b_1b_0},\]where $n$ is the length of the period of the repeating decimal expansion. What is the sum $b_0+b_1+\cdots+b_{n-1}$?
\end{prob}

\begin{prob}[AMC 12B 2016/22]{6}
For a certain positive integer $n$ less than $1000$, the decimal equivalent of $\frac{1}{n}$ is $0.\overline{abcdef}$, a repeating decimal of period $6$, and the decimal equivalent of $\frac{1}{n+6}$ is $0.\overline{wxyz}$, a repeating decimal of period $4$. Find $n.$
\end{prob}
\end{document}