\documentclass[mast]{lucky}

\usepackage{graphicx}

\title{Applications of Calculus}
\author{Kelin Zhu}
\date{ART}

\begin{document}
\maketitle
Here, we will discuss the common uses of single-variable differentiaton and integration in contest problems that are not explicitly about Calculus.
\section{Jensen's inequality and Tangent Line Trick}
(This section is more useful for Olympiad math rather than computational, but its importance demands for it to be included.)

A part of why ``Trivial by Jensen's'' is often said as a meme regarding problems that may or may not involve inequalities is its true power.
\begin{theo}[Jensen's Inequality]
In an interval, a function is \textbf{convex} if and only if its second derivative is nonnegative throughout the interval and \textbf{concave} if and only if its second derivative is nonpositive throughout the interval.

For real numbers $x_1,x_2\ldots x_i$ in a convex function $f$'s domain and positive real weights $w_1,w_2\ldots w_n$, we have
$$f\left(\frac{\sum w_i x_i}{\sum w_i}\right)\le \frac{\sum w_i f(x_i)}{\sum w_i}$$
When the function is concave, we have an analogous inequality
$$f\left(\frac{\sum w_i x_i}{\sum w_i}\right)\ge \frac{\sum w_i f(x_i)}{\sum w_i}$$
as $-f$ is convex.
\end{theo}

It is useful to memorize the convexities of the most common functions, and below are some of them:
\begin{exer}[List of functions to evaluate]
$x^2, x, \frac{1}{x}, \sqrt{x}, \log(x)$ over the real numbers

(Answers: convex, convex and concave, convex, concave, concave)
\end{exer}

The last function (log with respect to an arbitrary base), despite seemingly the least common, can be used to simplify a multitude of inequalities.
\section{Local maximas and minimas}
(This section can be thought of a follow-up to the graphing unit in a way, as it 
Here is the first problem I have ever solved in any contest using differentiation, which is a prime example of how the roots of derivatives can give us critical information on a function.
\begin{exam}[Stormersyle mock AMC 10/25]
An ordered pair $(a,b)$ is \textit{spicy} if there exists real $c$ such that the polynomial $f(x)=x^3+ax^2+bx+c$ has all real roots. For how many ordered pairs $(a,b)$ of integers with $1\le a,b\le 20$ is $(a,b)$ spicy?
\end{exam}
\begin{sol}
The key claim is the following: such a real $c$ exists iff $f$ has a local minima and maxima.

Since nonreal roots of a real-coefficient polynomial come in complex conjugate pairs, $f'$, which has degree 2, has either 2 distinct zeroes, no zeroes or a double root.

If it has two distinct roots, then we can draw a horizontal line between the local minima and maxima; since the polynomial is continuous, the line will intersect $f$ between the two critical points, once as $x\rightarrow -\infty$ and once as $x\rightarrow \infty$.

if it has a double root, then we can shift $f$ so that the inflection point is a triple root.

Otherwise, $f$ strictly increases (as $3>0$), and it's obviously impossible to choose a $c$ such that $f$ has 3 roots.

Therefore, we just need to calculate the number of pairs $(a,b)$ with $4a^2-12b\ge 0$, which can easily be computed to be $\ansbold{305}$.
\end{sol}

Here is a much more difficult example that still utilizes the properties of local minimas and maximas.
\begin{exam}[2021 HMMT Feb. AlgNT/9]
Find all monic cubic polynomials $f$ that have the following properties:
\begin{itemize}
\item $f$ is odd, and
\item over all reals $c, f(f(x))-c$ has either $1, 5$ or $9$ roots.
\end{itemize}
\end{exam}
\begin{walk}
\begin{enumerate}
\item Don't be scared by the problem number!
\item $f$ is of the form $x^3+ax$. Using simple reasoning, arrive at that $a<0$.
\item Consider moving a horizontal line from a large $y$ value (intersecting $f(f(x))$ once) downwards. What does it hopping from intersecting $f$ once to five times tell us?
\item Using the chain rule, solve for the local maximas of $f(f(x))$. (It might be helpful to make the substitution $a=-3b^2$.)
\item Use the fact that the local maximas have equal $y$ values to find $b$.
\end{enumerate}
\end{walk}
\section{Estimating series}
We can often estimate infinite sums with integrals, which solves many problems asking for the rough value (floor/ceil or rounded) of infinite sums.
\begin{exam}
Find the floor of $\sum_{n=1}^{1000000}\frac{1}{\sqrt{n}}$.
\end{exam}
\begin{sol}
Trying to look for smart telescopes or cancellations would be pointless, but fortunately, the sum is close to an easily evaluable integral.

The sum is a lower bound to $\int_{x=0}^{1000000}\frac{1}{\sqrt{x}}dx=2000$ and an upper bound to $\int_{x=1}^{1000001}\frac{1}{\sqrt{x}}dx>1998$. Therefore, the answer is either 1998 or 1999. 

We can see that $\int_{x=0}^{1}(1-\frac{1}{\sqrt{x}})dx=1$, and therefore $\int_{x=0}^{1000000}\left\lceil\frac{1}{\sqrt{x}}\right\rceil-\frac{1}{\sqrt{x}}>1$, making our answer $\ansbold{1998}$.
\end{sol}

\section{Calculating area}

\section{Problems}
\noindent\minpt{TBD}

\begin{prob}[SMT 2021]{2}
Farley the frog starts at the first lily pad in an infinite row of lily pads. If she is currently on the $n$th lily pad, she has a $\frac{1}{n}$ probability of jumping to the $n+1$th lilypad. Find the expected number of lily pads that she will ever reach.
\end{prob}
\end{document}
