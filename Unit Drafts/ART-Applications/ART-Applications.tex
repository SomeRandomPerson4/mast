\documentclass[mast]{lucky}

\usepackage{graphicx}

\title{Applications of Calculus}
\author{Kelin Zhu}
\date{ART}

\begin{document}
\maketitle
Here, we will discuss the common uses of single-variable differentiaton and integration in contest problems that are not explicitly about Calculus.
\section{Jensen's inequality and Tangent Line Trick}
(This section is more useful for Olympiad math rather than computational, but its importance demands for it to be included.)

A part of why ``Trivial by Jensen's'' is often said as a meme regarding problems that may or may not involve inequalities is its true power.
\begin{theo}[Jensen's Inequality]
In an interval, a function is \textbf{convex} if and only if its second derivative is nonnegative throughout the interval and \textbf{concave} if and only if its second derivative is nonpositive throughout the interval.

For real numbers $x_1,x_2\ldots x_i$ in a convex function $f$'s domain and positive real weights $w_1,w_2\ldots w_n$, we have
$$f\left(\frac{\sum w_i x_i}{\sum w_i}\right)\le \frac{\sum w_i f(x_i)}{\sum w_i}$$
When the function is concave, we have an analogous inequality
$$f\left(\frac{\sum w_i x_i}{\sum w_i}\right)\ge \frac{\sum w_i f(x_i)}{\sum w_i}$$
as $-f$ is convex.
\end{theo}

It is useful to memorize the convexities of the most common functions, and below are some of them:
\begin{exer}[List of functions to evaluate]
$x^2, x, \frac{1}{x}, \sqrt{x}, \ln(x)$ over the real numbers

(Answers: convex, convex and concave, convex, concave, concave)
\end{exer}

The last function (natural logarithm), despite seemingly the least common, can be used to simplify a multitude of inequalities.
\begin{exam}[AM-GM]
Prove that for a set of nonnegative real numbers $a_1,a_2,\ldots,a_n$, the following always holds:\[\frac{a_1+a_2+\ldots+a_n}{n}\geq\sqrt[n]{a_1a_2\cdots a_n}\]
\end{exam}
\begin{walk}
\begin{enumerate}
\item Take the logs of both sides.
\item Use the fact that log is concave, and finish immediately with Jensen's.
\end{enumerate}
\end{walk}
Because of log's properties (turns multiplication into addition, exponentiation into multiplication, division into subtraction), the above trick is often applicable on inequalities involving a fixed function.
\begin{exer}[Taiwan TST 2014]
Let $a_i > 0$ for $i=1,2,\dots,n$ and suppose $a_1 + a_2 + \dots + a_n = 1$. Prove that for any positive integer $k$,
\[ \left( a_1^k + \frac{1}{a_1^k} \right) \left( a_2^k + \frac{1}{a_2^k} \right) \dots \left( a_n^k + \frac{1}{a_n^k} \right) \ge \left( n^k + \frac{1}{n^k} \right)^n. \]
\end{exer}

Sometimes, even when Jensen seems to be hopeless, we can still prove the following pseudo-Jensen inequality:
$$f(x)\ge f(a)+f'(a)(x-a)$$
where $a$ is the average of all variables. This is the tangent line trick.
\begin{exam}[Poland 1996]
$a,b,c \geq-3/4$ and $a+b+c=1$. Show that: $\frac{a}{1+a^{2}}+\frac{b}{1+b^{2}}+\frac{c}{1+c^{2}}\leq \frac{9}{10}$
\end{exam}
\begin{sol}
By TLT, notice the inequality 
$$\frac{x}{x^2+1}-\frac{18}{25}(x-\frac{1}{3})-\frac{3}{10}\le 0 \iff -\frac{(4x+3)(3x-1)^2}{50(x^2+1)}\le 0$$
Summing this over all variables yields the desired result.
\end{sol}
\section{Local maximas and minimas}
(This section can be thought of a follow-up to the graphing unit in a way, as it 
Here is the first problem I have ever solved in any contest using differentiation, which is a prime example of how the roots of derivatives can give us critical information on a function.
\begin{exam}[Stormersyle mock AMC 10/25]
An ordered pair $(a,b)$ is \textit{spicy} if there exists real $c$ such that the polynomial $f(x)=x^3+ax^2+bx+c$ has all real roots. For how many ordered pairs $(a,b)$ of integers with $1\le a,b\le 20$ is $(a,b)$ spicy?
\end{exam}
\begin{sol}
The key claim is the following: such a real $c$ exists iff $f$ has a local minima and maxima.

Since nonreal roots of a real-coefficient polynomial come in complex conjugate pairs, $f'$, which has degree 2, has either 2 distinct zeroes, no zeroes or a double root.

If it has two distinct roots, then we can draw a horizontal line between the local minima and maxima; since the polynomial is continuous, the line will intersect $f$ between the two critical points, once as $x\rightarrow -\infty$ and once as $x\rightarrow \infty$.

if it has a double root, then we can shift $f$ so that the inflection point is a triple root.

Otherwise, $f$ strictly increases (as $3>0$), and it's obviously impossible to choose a $c$ such that $f$ has 3 roots.

Therefore, we just need to calculate the number of pairs $(a,b)$ with $4a^2-12b\ge 0$, which can easily be computed to be $\ansbold{305}$.
\end{sol}

Here is a much more difficult example that still utilizes the properties of local minimas and maximas.
\begin{exam}[2021 HMMT Feb. AlgNT/9]
Find all monic cubic polynomials $f$ that have the following properties:
\begin{itemize}
\item $f$ is odd, and
\item over all reals $c, f(f(x))-c$ has either $1, 5$ or $9$ roots.
\end{itemize}
\end{exam}
\begin{walk}
\begin{enumerate}
\item Don't be scared by the problem number!
\item $f$ is of the form $x^3+ax$. Using simple reasoning, arrive at that $a<0$.
\item Consider moving a horizontal line from a large $y$ value (intersecting $f(f(x))$ once) downwards. What does it hopping from intersecting $f$ once to five times tell us?
\item Using the chain rule, solve for the local maximas of $f(f(x))$. (It might be helpful to make the substitution $a=-3b^2$.)
\item Use the fact that the local maximas have equal $y$ values to find $b$.
\end{enumerate}
\end{walk}
\section{Derivatives Trick}

\begin{theo}[Derivatives Trick]
Let $f(x)=(x-r_{1})(x-r_{2})\ldots(x-r_{k})$. Then,
$$\frac{f'(x)}{f(x)} = \sum_{i=1}^{k} \frac{1}{x-r_{i}}$$
\end{theo}

\begin{pro}
We take the natural log of both sides.
$$\ln (f(x)) = \sum_{i=1}^{k} \ln(x-r_{i})$$
We take the derivative of both sides using the chain rule.
$$\frac{f'(x)}{f(x)} = \sum_{i=1}^{k} \frac{1}{x-r_{i}}$$
\end{pro}

We can extend this further by taking the derivative again as much as we need to do to find $\sum_{i=1}^{k} \frac{1}{(x-r_{k})^n}$ for $n$ in general.
\begin{theo}[Derivatives Trick Extended]
$$\frac{f''(x)f(x)-f'(x)^2}{f(x)^2} = -\sum_{i=1}^{k} \frac{1}{(x-r_{i})^2}$$
\end{theo}

\begin{pro}
Use the Quotient Rule.
\end{pro}
\section{Estimating series}
We can often estimate infinite sums with integrals, which solves many problems asking for the rough value (floor/ceil or rounded) of infinite sums.
\begin{exam}
Find the floor of $\sum_{n=1}^{1000000}\frac{1}{\sqrt{n}}$.
\end{exam}
\begin{sol}
Trying to look for smart telescopes or cancellations would be pointless, but fortunately, the sum is close to an easily evaluable integral.

The sum is a lower bound to $\int_{x=0}^{1000000}\frac{1}{\sqrt{x}}dx=2000$ and an upper bound to $\int_{x=1}^{1000001}\frac{1}{\sqrt{x}}dx>1998$. Therefore, the answer is either 1998 or 1999. 

We can see that $\int_{x=0}^{1}(1-\frac{1}{\sqrt{x}})dx=1$, and therefore $\int_{x=0}^{1000000}\left\lceil\frac{1}{\sqrt{x}}\right\rceil-\frac{1}{\sqrt{x}}>1$, making our answer $\ansbold{1998}$.
\end{sol}
Here is a more difficult AIME problem 15 that instead utilizes trigonometric integrals:
\begin{exam}[AIME II 2000/15]
Find the least positive integer $n$ such that\[ \frac 1{\sin 45^\circ\sin 46^\circ}+\frac 1{\sin 47^\circ\sin 48^\circ}+\cdots+\frac 1{\sin 133^\circ\sin 134^\circ}=\frac 1{\sin n^\circ}. \]
\end{exam}
\begin{walk}
This is perhaps more naturally solved with telescoping, but we present an integration solution due to AoPS user \textbf{nkim9005}.
\begin{enumerate}
    \item Convert the cosecants to secants.
    \item Estimate each term in a way that makes the sequence easily integrable.
    \item Using this estimate, integrate and find the approximate value of $\sin(x)$.
    \item Finish the problem using the fact that $sin(x)\approx x$ for small values.
\end{enumerate}
\end{walk}
As you probably have noticed, this method does rely on praying that the errors won't be significant.
In the case of our last example, our 3 sources of error happened to cancel each other out.
However, if you are unlucky, inaccuracies have the potential to render a calculation useless.
Therefore, this technique should be used as more of a last resort for when you see no other way out.

\section{Calculating area}
(this section veers towards other subjects, and is only included because it is also an application of integration; feel free to skip it)
Most commonly in Geometric probability, to solve the problem or as an intermediary stage in solving the problem, it is necessary to evaluate the area bounded by multiple functions. That is a textbook application of integration.

Here is an easy example to give a general feel of these problems:
\begin{exam}[AMC 10A 2015/25]
Let $S$ be a square of side length $1$. Two points are chosen independently at random on the sides of $S$. The probability that the straight-line distance between the points is at least $\tfrac12$ is $\tfrac{a-b\pi}c$, where $a$, $b$, and $c$ are positive integers and $\gcd(a,b,c)=1$. What is $a+b+c$?
\end{exam}
\begin{sol}
First, scale up everything by a factor of 2. Select the first point $A$; WLOG assume that it is on the bottom side and closer to the bottom-left vertex.

If the second point is also on the bottom side, it is well-known that there is a $\frac{1}{4}$ probability that they are not within 1 of each other.

If the second point is on the top side or right side, they must have distance $>1$.

The tricky part is if the second point is on the left side. Assume that $A$ is $x$ away from the bottom-left vertex; the probability that $B$ is at least $\sqrt{1-x^2}$ away from the bottom-left vertex is equal to $\frac{\int_{x=0}^{2}(2-\sqrt{1-x^2})dx}{2}=\frac{2-\frac{\pi}{4}}{2}$.

The answer is therefore $\frac{2+\frac{1}{4}+1-\frac{\pi}{8}}{4}=\frac{26-\pi}{32}$, making the requested sum $\ansbold{59}$.
\end{sol}

Here is a more difficult example, where the setup is perhaps more difficult to see but ultimately spiritually similar:
\begin{exam}[PUMAC Combo 2016/7A]
The Dinky is a train connecting Princeton to the outside world. It runs on an odd schedule:
the train arrives once every one-hour block at some uniformly random time (once at a random
time between 9am and 10am, once at a random time between 10am and 11am, and so on).
One day, Emilia arrives at the station, at some uniformly random time, and does not know
the time. She expects to wait for $y$ minutes for the next train to arrive. After waiting for an
hour, a train has still not come. She now expects to wait for $z$ minutes. Find $yz$.
\end{exam}
\begin{sol}
Suppose that Emilia arrives after $x$ of the current hour is over for some nonnegative $x<1$.

If she arrives after the train for the hour (with probability $x$), she is expected to wait $1-x+\frac{1}{2}$ more hours.

Otherwise, she is expected to wait $\frac{1-x}{2}$ more hours. From here, we can calculate $y=60\int_{x=0}^{1}[(x)(1-x+\frac{1}{2})+(1-x)\frac{1-x}{2}]dx=35$.

After 1 hour, there is a $x(1-x)$ chance that no trains have arrived. In which case, Emilia is expected to wait for $\frac{x}{2}$ more hours. We can calculate $z=60\frac{\int_{x=0}^{1}(x^2)(1-x)}{2\int_{x=0}^{1}x(1-x)}=15$, making our answer $35\cdot 15=\ansbold{525}.$
\end{sol}



\section{Problems}
\noindent\minpt{TBD}

\begin{prob}[SMT 2021]{2}
Farley the frog starts at the first lily pad in an infinite row of lily pads. If she is currently on the $n$th lily pad, she has a $\frac{1}{n}$ probability of jumping to the $n+1$th lilypad. Find the expected number of lily pads that she will ever reach.
\end{prob}

\begin{prob}[OMO Spring 2020/24]{13}
Let $A$, $B$ be opposite vertices of a unit square with circumcircle $\Gamma$. Let $C$ be a variable point on $\Gamma$. If $C\not\in\{A, B\}$, then let $\omega$ be the incircle of triangle $ABC$, and let $I$ be the center of $\omega$. Let $C_1$ be the point at which $\omega$ meets $\overline{AB}$, and let $D$ be the reflection of $C_1$ over line $CI$. If $C \in\{A, B\}$, let $D = C$. As $C$ varies on $\Gamma$, $D$ traces out a curve $\mathfrak C$ enclosing a region of area $\mathcal A$. Compute $\lfloor 10^4 \mathcal A\rfloor$.
\end{prob}
\end{document}