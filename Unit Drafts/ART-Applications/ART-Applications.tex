\documentclass{article}
\usepackage[mast]{lucky}
\usepackage{graphicx}

\title{Applications of Differentiation}
\author{Kelin Zhu}
\date{ART}

\begin{document}
\maketitle
Here, we will discuss the common uses of differentiation in contest problems that are not explicitly about Calculus.
\section{Jensen's inequality and Tangent Line Trick}
A part of why ''Trvial by Jensen's'' is often said as a meme regarding problems that may or may not involve inequalities is its true power.
\section{Local maximas and minimas}
Here is the first problem I have ever solved in any contest using differentiation, which is a prime example of how the roots of derivatives can give us critical information on a function.
\begin{exam}[Stormersyle mock AMC 10/25]
An ordered pair $(a,b)$ is \textit{spicy} if there exists real $c$ such that the polynomial $f(x)=x^3+ax^2+bx+c$ has all real roots. For how many ordered pairs $(a,b)$ of integers with $1\le a,b\le 20$ is $(a,b)$ spicy?
\end{exam}
\begin{sol}
The key claim is the following: such a real $c$ exists iff $f$ has a local minima and maxima.
Since nonreal roots of a real-coefficient polynomial come in complex conjugate pairs, $f'$, which has degree 2, has either 2 zeroes, no zeroes or a double root.
If it has no zeroes, then $f$ strictly increases (as $3>0$), and it's obviously impossible to choose a $c$ such that $f$ has 3 roots.
if it has a double root, then we can shift $f$ so that the inflection point is a triple root.
Otherwise, we can draw a horizontal line between the local minima and maxima; since the polynomial is continuous, the line will intersect $f$ between the two critical points, once as $x\rightarrow -\infty$ and once as $x\rightarrow \infty$.
Therefore, we just need to calculate the number of pairs $(a,b)$ with $4a^2-12b\ge 0$, which can easily be computed to be $\ansbold{305}$.
\end{sol}
Here is a much more difficult example that still utilizes the properties of local minimas and maximas.
\begin{exam}[2021 HMMT Feb. AlgNT/9]
Find all monic cubic polynomials $f$ following the following properties:
\begin{itemize}
\item $f$ is odd, and
\item over all reals $c, f(f(x))-c$ has either $1, 5$ or $9$ roots.
\end{itemize}
\end{exam}
\section{Problems}
\noindent\minpt{TBD}

\begin{prob}[SMT 2021]{2}
Farley the frog starts at the first lily pad in an infinite row of lily pads. If she is currently on the $n$th lily pad, she has a $\frac{1}{n}$ probability of jumping to the $n+1$th lilypad. Find the expected number of lily pads that she will ever reach.
\end{prob}
\end{document}