\documentclass{article}
\usepackage[mast]{dennis}

\title{Perspectives}
\author{Dennis}
\date{CRV}

\begin{document}
\maketitle

%theory

\section{Pointers}

In the easier version of this handout (\db{CQV-Perspectives}), the bijections you need to make are mostly studied and the aim is to familiarize the reader with basic combinatorial theorems, tying them together with the pedagogy of ``perspectives.'' Aside from Freedom problems, the techniques you needed to use were mostly handed to you on a platter. But there are a large section of perspectives-style problems that do not cleanly fit into a category, and in those cases only intuition and experience will help. You should know everything from \db{CQV-Perspectives} and should be able to feel when a problem or argument is ``perspectives-style,'' even if you can't articulate why.

Because you are expected to know basic theory and it's already covered in CQV anyway, we just present a couple of flagship examples.

\begin{exam}[AMC 12A 2014/19]
There are exactly $N$ distinct rational numbers $k$ such that $|k|<200$ and \[5x^2+kx+12=0\] has at least one integer solution for $x$. What is $N$?
\end{exam}

\begin{sol}
We know basically nothing about $k.$ But we do know a lot about $x$; for instance, it is an integer. Thus we can instead solve for $k$, noting that
\[k=-\left(5x+\frac{12}{x}\right).\]
Things would get a little dicey if there were two integer values of $x$ that gave the same value of $k$, but we do not need to worry about that since by AM-GM, the function is strictly decreasing. (Note the negative sign.)

Since $\frac{12}{x}$ is small for large $|x|$, $x$ can be any integer between $-39$ and $39$ inclusive, except for $0.$ Thus the anwser is $78.$
\end{sol}

This is a flagship Perspectives problem because instead of counting something that's difficult to keep track of, you count something \textit{else} you know a lot about. Make sure in these types of problems you don't miss ``edge cases'' -- there are going to be times when the Perspectives argument doesn't entirely hold, and figuring out these exceptions is going to be part of solving the problem. 

\begin{exam}[AIME 1994/11]
Ninety-four bricks, each measuring $4''\times10''\times19'',$ are to be stacked one on top of another to form a tower 94 bricks tall. Each brick can be oriented so it contributes $4''\,$ or $10''\,$ or $19''\,$ to the total height of the tower. How many different tower heights can be achieved using all ninety-four of the bricks? 
\end{exam}

Though Perspectives may seem like a catch-all for bijection, what makes a problem really feel ``Perspectives-like'' is clever uses of restrictions. For instance, in this next example, you will see that while there are multiple ways to construct a height, you only need to consider a set of constructions that covers each height at least once. This is secretly easier than it sounds.

\begin{sol}
First note we can biject the height of the bricks. Subtracting $4$ from each dimension to get heights of $0, 6, 15$ will reduce the total height by a constant of $4\cdot 94,$ and then dividing each dimension by $3$ to get heights of $0, 2, 5$ will divide the total height by $3$. Thus both processes are bijections, and we can now proceed with dimensions $0, 2, 5$.

Now note that if we have five blocks of height $2$, we can replace it with two blocks of height $5$ and three blocks of height $0$. This implies that if a tower can be constructed, \db{it can be constructed with no more than four blocks of height $2$}, as any extra can just be swapped with blocks of height $5$ and $0$ to achieve the same height.

Thus we only have to consider the cases where there are $0$, $1$, $2$, $3$, or $4$ blocks of height $2$. By mod $5$, none of these cases will overlap. For each case, you just have to consider how many of the remaining blocks have height $5$. (For example, if there are no blocks of height $2$, then there are $94$ blocks left. Since we can have anywhere between $0$ to $94$ blocks of height $5$, this case has $95$ possibilities.) This implies the final answer is
\[95+94+93+92+91=\ansbold{465}.\]
\end{sol}

\pagebreak

\section{Problems}

\minpt{TBD}

\psetquote{Doubt them. Question them, suspect them... and take a good, long look into their hearts. Humans are the kind of beings that can't put their pain into words, after all.}{Liar Game}

\begin{req}[IMO 1987/1]{3}
Let $p_n (k)$ be the number of permutations of the set $\{ 1, \ldots , n \} , \; n \ge 1$, which have exactly $k$ fixed points. Prove that

\[\sum_{k=0}^{n} k \cdot p_n (k) = n!.\]}

\begin{req}[AIME 1994/11]{6}

\end{req
\end{req}

\begin{prob}[ISL 2018/C1]{9}
Let $n\geq 3$ be an integer. Prove that there exists a set $S$ of $2n$ positive integers satisfying the following property: For every $m=2,3,...,n$ the set $S$ can be partitioned into two subsets with equal sums of elements, with one of subsets of cardinality $m.$
\end{prob}

\begin{prob}[IMO 2002/1]{9}
Let $n$ be a positive integer. Each point $(x,y)$ in the plane, where $x$ and $y$ are non-negative integers with $x+y<n$, is coloured red or blue, subject to the following condition: if a point $(x,y)$ is red, then so are all points $(x',y')$ with $x'\leq x$ and $y'\leq y$. Let $A$ be the number of ways to choose $n$ blue points with distinct $x$-coordinates, and let $B$ be the number of ways to choose $n$ blue points with distinct $y$-coordinates. Prove that $A=B$.
\end{prob}

\begin{prob}[CMO 2019/3]{13}
Let $m$ and $n$ be positive integers. A $2m\times 2n$ grid of squares is colored in the usual chessboard fashion. Determine the number of ways to place $mn$ counters on the white squares, at most one counter per square, so that no two counters are diagonally adjacent.
\end{prob}

\begin{prob}[USAMO 2015/4]{13}
Steve is piling $m \geq 1$ indistinguishable stones on the squares of an $n \times n$ grid. Each square can have an arbitrarily high pile of stones. After he finished piling his stones in some manner, he can then perform stone moves, defined as follows. Consider any four grid squares, which are corners of a rectangle, i.e. in positions $(i, k),(i, l),(j, k),(j, l)$ for some $1 \leq i, j, k, l \leq n,$ such that $i<j$ and $k<l .$ A stone move consists of either removing one stone from each of $(i, k)$ and $(j, l)$ and moving them to $(i, l)$ and $(j, k)$ respectively, or removing one stone from each of $(i, l)$ and $(j, k)$ and moving them to $(i, k)$ and $(j, l)$ respectively.

Two ways of piling the stones are equivalent if they can be obtained from one another by a sequence of stone moves. How many different non-equivalent ways can Steve pile the stones on the grid?
\end{prob}
\end{document}