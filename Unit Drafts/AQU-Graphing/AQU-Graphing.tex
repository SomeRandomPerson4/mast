\documentclass[mast]{lucky}
\usepackage{asymptote}

\title{Graphing}
\author{math1}
\date{AQU}

\begin{document}

\maketitle

By itself, graphing does not provide a rigorous solution, but allows us to gain intuition about the polynomial, function, or other condition given in the problem.

We begin with the following example.

\begin{exam}[AIME I 2015/10]
    Let $f(x)$ be a third-degree polynomial with real coefficients satisfying
    \begin{align*}
        |f(1)|=|f(2)|=|f(3)|=|f(5)|=|f(6)|=|f(7)|=12.
    \end{align*}
    Find $|f(0)|$.
\end{exam}

\begin{sol}
    A purely algebraic approach does not seem nice here due to the absolute values. Instead, we consider a graphical approach. By symmetry, we may assume that $f(1) = 12$. Red points are points that the cubic passes through.
    \begin{center}
        \begin{asy}
            size(5cm);
            draw((0,0)--(85,0),arrow=Arrow(TeXHead));
            draw((0,0)--(0,-50),arrow=Arrow(TeXHead));
            draw((0,0)--(0,50),arrow=Arrow(TeXHead));

            draw((-1,30)--(1,30));
            draw((-1,-30)--(1,-30));

            label("$12$",(-7,30));
            label("$-12$",(-10,-30));

            for(int i = 1; i < 8; ++i) {     
                int n = 10*i;
                if(i == 4) {
                    continue;
                }
                draw((n, 1)--(n, -1));
                dot((n,30), blue);
                dot((n,-30), blue);
            }
            dot((10,30), red);
        \end{asy}
    \end{center}
    To continue, we would like to use a strong property of cubics that would uniquely determine the rest of the points. One such property is that a cubic has exactly two turning points. 

    Attempting to draw a curve with exactly two turning points that passes through one blue point per tick gives the following unique (up to symmetry) set of red points that the cubic passes through.
    \begin{center}
        \begin{asy}
            size(5cm);
            draw((0,0)--(85,0),arrow=Arrow(TeXHead));
            draw((0,0)--(0,-50),arrow=Arrow(TeXHead));
            draw((0,0)--(0,50),arrow=Arrow(TeXHead));

            draw((-1,30)--(1,30));
            draw((-1,-30)--(1,-30));

            label("$12$",(-7,30));
            label("$-12$",(-10,-30));

            for(int i = 1; i < 8; ++i) {
                int n = 10*i;
                if(i == 4) {
                    continue;
                }
                draw((n, 1)--(n, -1));
                dot((n,30), blue);
                dot((n,-30), blue);
            }
            dot((10,30), red);
            dot((20,-30), red);
            dot((30,-30), red);
            dot((50,30), red);
            dot((60,30), red);
            dot((70,-30), red);
        \end{asy}
    \end{center}
    There are many ways to finish the problem from here. A direct way is to write a system of equations and solve for the constant term of $f(x)$, but it is more elegant to note that
    \begin{align*}
        f(x) = a(x - 1)(x - 5)(x - 6) + 12
    \end{align*}
    for some constant $a$. In particular, $-12 = f(2) = 12a + 12$, so $a = -2$. Then, $|f(0)| = f(0) = -30a + 12 = -30(-2) + 12 = \ansbold{72}$.
\end{sol}

\begin{remark}
    Notice that the solution is not rigorous (although similar ideas are likely involved in a rigorous solution to the problem, only formalized), but interpreting the given function graphically allowed us to easily express informal ideas on paper.
\end{remark}

\end{document}