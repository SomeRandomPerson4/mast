\documentclass{article}
\usepackage[mast]{lucky}

\title{Algebraic Manipulation}
\author{Dennis Chen, skyscraper, William Dai}
\date{APV}

\begin{document}
\maketitle

Here we discuss some clever algebra manipulations. These basic tricks are fairly common in competition questions, and knowledge of these tricks generally reduces the time needed to solve algebra questions.

\section{Symmetric Systems}

The most basic of algebraic manipulations involves systems of equations. The gist is that when you are given $x\circ y,y\circ z,$ and $x\circ z$ (for some operation $\circ$), you can find $x \circ y \circ z$ to determine $x,y,$ and $z.$
\begin{exam}[2018 AMC 10B/4]
A three-dimensional rectangular box with dimensions $X$, $Y$, and $Z$ has faces whose surface areas are $24$, $24$, $48$, $48$, $72$, and $72$ square units. What is $X$ + $Y$ + $Z$?
\end{exam}
\begin{sol}
Manipulation does the trick better than just bashing out possible values. Note that the surface areas can be expressed as $XY,XZ,$ and $YZ.$ Therefore, we can let $XY = 24, XZ = 48, YZ = 72.$

A good way to solve for $X,Y,$ and $Z$ individually is by finding $XYZ$ and then dividing this quantity by $XY, XZ,$ and $YZ.$ Multiplying all equations together yields $XY \cdot XZ \cdot YZ = (XYZ)^2 = 24 \cdot 48 \cdot 72$, and taking the square root gives $XYZ = 288,$ so it follows that $X = \frac{XYZ}{YZ} = \frac{288}{72} = 4.$ Similarly, $Y = 6$ and $Z=12,$ so our answer is $4+6+12= 22.$
\end{sol}
\begin{exam}[Classic]
Find positive integers $a,b,$ and $c$ that satisfy
\begin{align*}
(a+b)(a+c)&=96(b+c) \\
(a+b)(b+c)&=54(a+c) \\
(a+c)(b+c)&=24(a+b)
\end{align*}
\end{exam}

\begin{sol}
The idea of this problem is similar to the previous example, in that there is a way to "manipulate" the product. Multiplying the equations gives
\[(a+b)^2(a+c)^2(b+c)^2 = 96 \cdot 54 \cdot 24(a+b)(a+c)(b+c)\]
\[(a+b)(a+c)(b+c)= 96 \cdot 54 \cdot 24\]
We can divide by $(a+b)(a+c)$ to obtain $b+c = \frac{96 \cdot 54 \cdot 24}{96} \cdot (b+c),$ or $b+c = 36,$ and then apply the same method with $a+b$ and $a+c.$ But we're not done, we still need to resolve:
\begin{align*}
b+c &= 36 \\
a+c &= 72\\
a+b &= 48
\end{align*}
Applying the manipulation trick again, we have: $2a +2b+2c = 156$, implying that $a+b+c = 78.$ Subtracting $a+b+c$ from $b+c,a+c,$ and $a+b$ leads to our desired answer: $\ansbold{a = 42, b = 6, c = 30}.$
\end{sol}

\section{Fraction Tricks}
There's no ``official'' name for this manipulation, but it is often very useful in manipulating fractions with ugly denominators and numerators. 
\begin{theo}[Fraction Trick]
If $\frac{a}{b}=\frac{c}{d}$, then
\[\frac{a}{b}=\frac{c}{d}=\frac{a+c}{b+d}.\]
\end{theo}

\begin{pro}
Let $\frac{a}{b}=\frac{c}{d}=k$. Then $\frac{a+c}{b+d}=\frac{bk+dk}{b+d}=k$.
\end{pro}

This may seem obvious but a large amount of people often forget that this exists when it comes up.

A similarly ``obvious'' trick is flipping the fraction when the denominator is complicated and the numerator is simple.

\begin{exam}[Classic]
If $\frac{x}{x^2+x+1}=\frac{1}{3}$, find $x$.
\end{exam}

\begin{sol}
We reciprocate both sides to get $\frac{x^2+x+1}{x}=3$. Then, $x+1+\frac{1}{x}=3$, or $\implies x-2+\frac{1}{x}=0$. We multiply by $x$ to get $x^2-2x+1=(x-1)^2=0$, so $x=\ansbold{1}$.
\end{sol}

\section{Numbers as Variables}
Like the title implies, we treat certain numbers in the problems as variables. Then, we can apply more general algebraic identities or manipulate them easier.
\begin{exam}[SMT 2021/N5]
There are exactly four distinct positive integers $n$ for which $15380 - n^2$ is a perfect square. Noting that $13^2 + 37^2 = 1538$, compute the sum of the four possible values of $n$.
\end{exam}

This problem is quite difficult compared to the rest of the handout. Even then, I believe the problem is worth presenting because it's such a good example of how numbers can obscure the intent of an algebraic problem, and how knowing that these numbers are used solely to obscure can give you the courage to push through and find the hidden identity.

\begin{sol}
The solution hinges on the fact that
\[(a^2+b^2)(c^2+d^2)=(ac+bd)^2+(ad-bc)^2.\]
Note that
\[15380=(1^2+3^2)(13^2+37^2)=(13+111)^2+(36-27)^2=124^2+9^2\]
and
\[15380=(1^2+3^2)(37^2+13^2)=(37+39)^2+(13-74)^2=76^2+61^2.\]
Since we know there are only four values of $n,$ the answer is
\[124+9+76+61=270.\]
\end{sol}

This $(a^2+b^2)(c^2+d^2)=(ac+bd)^2+(ad-bc)^2$ identity is known as the Brahmagupta-Fibonacci Identity. It is a quite popular identity used in non-AMC style contests, such as ARML, college contests, etc.

\begin{exam}[2017-2018 Mandelbrot]
Let us say that a decade is \emph{primeval} if it contains four prime numbers. For instance, the decade from $1480$ to $1490$ was primeval, since $1481, 1483, 1487$ and $1489$ are all primes. Let $p_1,p _2, p_3$ and $p_4$ be the primes (in order) in the  next primeval decade after $2020.$ Compute the value of $p_2 p_3-p_1 p_4$.
\end{exam}

\begin{sol}
Finding the next primeval decade is too time consuming! We instead use two observations:
\begin{itemize}
\item Prime numbers greater than $5$ must end in $1,3,7,$ or $9.$
\item If the oldest year of a primeval decade is $x,$ then the four prime years are $x+1,x+3,x+7,$ and $x+9.$
\end{itemize}
Now the problem becomes standard algebra fare; our answer is $(x+3)(x+7)-(x+1)(x+9) = 12.$ We avoided unnecessary computation here by treating large numbers as variables instead.
\end{sol}

\begin{exam}[Math Prizes For Girls 2015]{6}
Let $S$ be the sum of all distinct real solutions of the equation 
\[\sqrt{x + 2015} = x^2 - 2015.\]
Compute $\lfloor 1/S \rfloor$.  Recall that if $r$ is a real number, then $\lfloor r \rfloor$ (the floor of $r$) is the greatest integer that is less than or equal to $r$.
\end{exam}

\begin{sol}
Let $2015=y$. Then, we have $\sqrt{x+y}=x^2-y$, which implies that $x+y=x^4-2x^2y+y^2\implies y^2+(-2x^2-1)y+x^4-x=0$. Thus $y=\frac{2x^2+1 \pm (2x+1)}{2}$. Now, we have $2015=x^2+x+1$ or $2015=x^2-x$. These give $x=\frac{-1 \pm \sqrt{8057}}{2}$ and $x=\frac{1 \pm \sqrt{8061}}{2}$. Now, note that we have
\[x+y\ge 0 \implies x \ge -2015~~\text{and}~~x^2-y\ge 0\implies |x|\ge \sqrt{2015}.\]
We can see that $\frac{-1 + \sqrt{8057}}{2}>\sqrt{2015}$ and that $\frac{-1-\sqrt{8057}}{2} < -\sqrt{2015}$. Also, $\frac{1-\sqrt{8061}}{2} > - \sqrt{2015}$ and $\frac{1+\sqrt{8061}}{2} > \sqrt{2015}$. So our only two solutions are $\frac{-1-\sqrt{8057}}{2}$ and $\frac{1+\sqrt{8061}}{2}$.

Then $\frac{1}{S}=\frac{2}{\sqrt{8061}-\sqrt{8057}}=\frac{\sqrt{8061}+\sqrt{8057}}{2}$. Since $89 < \frac{1}{S} < 90$ so our answer is $\ansbold{89}$.
\end{sol}

\section{Symmetric and Cyclic Expressions}
We can sometimes use the fact that a sum is symmetric along with combinatorical thinking to quickly determine it.

\begin{exam}[AMC 10A 2021/14]
All the roots of the polynomial $z^6-10z^5+Az^4+Bz^3+Cz^2+Dz+16$ are positive integers, possibly repeated. What is the value of $B$?
\end{exam}

\section{Substitution}

\begin{exam}[NEMO 2017]
The value of the expression
\[\sqrt{1+\sqrt{\sqrt[3]{32}-\sqrt[3]{16}}} + \sqrt{1-\sqrt{\sqrt[3]{32}-\sqrt[3]{16}}}\]
can be written as $\sqrt[m]{n}$, where $m$ and $n$ are positive integers. Compute the smallest possible value of
$m + n$.
\end{exam}

All of the following involve the very common substitution of letting $x+y=a$ and $xy=b$ and rewriting the terms you know in terms of these expressions.
\begin{exam}[HMMT February 2013]
Let $x$ and $y$ be real numbers with $x > y$ such that $x^2 y^2 + x^2 + y^2 + 2xy = 40$ and $xy + x + y = 8$. Find
the value of $x$.
\end{exam}

\begin{exam}[HMMT February 2013]
Let $x,y$ be complex numbers such that $\frac{x^2+y^2}{x+y} = 4$ and $\frac{x^4+y^4}{x^3+y^3} = 2$. Find all possible values of 
$\frac{x^6+y^6}{x^5+y^5}$.
\end{exam}

\begin{exam}[HMMT February 2013]
Let $a$ and $b$ be real numbers such that $\frac{ab}{a^2+b^2}=\frac{1}{4}$. Find all possible values of $\frac{|a^2-b^2|}{a^2+b^2}$.
\end{exam}

\begin{sol}
$\frac{\sqrt{3}}{2}$
\end{sol}


%%10 pm 2\sqrt(17), sub x+y & xy
\pagebreak\section{Problems}

\minpt{}

\psetquote{I can only say that the human curiosity is something completely irrational.}{Puella Magi Madoka Magica}

\begin{prob}[AMC 10A 2018/14]{2}
What is the greatest integer less than or equal to\[\frac{3^{100}+2^{100}}{3^{96}+2^{96}}?\]
\end{prob}

\begin{prob}[2008-2009 Mandelbrot]{3}
Austin currently owns some shirts, pants, and pairs of shoes; he chooses one of each to create an outfit. If he were to obtain one more shirt, his total number of outfits would increase by $48.$ Similarly, if he bought another pair of pants he would have $90$ more outfits, while an extra pair of shoes would result in $120$ more outfits. How many outfits can he currently create?
\end{prob}
  

\begin{prob}[OMO Fall 2014/12]{4}
Let $a$, $b$, $c$ be positive real numbers for which
\[
  \frac{5}{a} = b+c, \quad
  \frac{10}{b} = c+a, \quad \text{and} \quad
  \frac{13}{c} = a+b.
\]
If $a+b+c = \frac mn$ for relatively prime positive integers $m$ and $n$, compute $m+n$.
\end{prob}
  
  
\begin{prob}[JMC 10 2021/22]{5}
Let $r_1,r_2,r_3,r_4$ be the roots of $P(x)= x^4+4x^3-3x^2+2x-1.$ Suppose $Q(x)$ is the monic polynomial with all six roots in the form $r_{i}+r_{j}$ for integers $1\le i < j \le 4.$ What is the coefficient of the $x^4$ term in the polynomial $Q(x)?$
\end{prob}


\begin{prob}[2019 hARMLess Mock ARML/4]{5}
Fran has two congruent rectangular prisms, each with volume $V$. There are three distinct ways for Fran to glue the two prisms together along congruent faces to form a larger rectangular prism. The three prisms that can be created in this way have surface areas $20,18,$ and $17$. Compute $V$.
\end{prob}


\begin{prob}[vvluo]{6}
Determine all solutions, if any exist, to the system
\begin{align*}
\frac{1}{xy}&=\frac{x}{z}+1 
\frac{1}{yz}&=\frac{y}{x}+1 
\frac{1}{zx}&=\frac{z}{y}+1.
\end{align*}
\end{prob}


\begin{prob}[BMT 2020]{6}
Let $a,b$ and $c$ be real numbers such that $a+b+c=\frac{1}{a}+\frac{1}{b}+\frac{1}{c}$ and $abc=5$. The value of
\[(a-\frac{1}{b})^3+(b-\frac{1}{c})^3+(c-\frac{1}{a})^3\]
can be written in the form $\frac{m}{n}$, where $m$ and $n$ are relatively prime positive integeres. Compute $m+n$.
\end{prob}

\begin{prob}[NICE Spring 2021/13]{6}
Suppose $x$ and $y$ are nonzero real numbers satisfying the system of equations
\begin{align*}
    3x^2 + y^2 &= 13x,
    x^2 + 3y^2 &= 14y.
\end{align*}
Find $x+y$.
\end{prob}



\begin{req}[JMC 10 2021/18]{9}
If $x,y,$ and $z$ are positive real numbers that satisfy the equation
\[xy+yz+zx=96(y+z)-x^2 = 24(x+z) -y^2 = 54(x+y) -z^2,\]
what is the value of $xyz$?
\end{req}

\begin{prob}[CARML 2019/8]{13}
There exist unique positive integers $1<a<b$ satisfying
\[a^2+b^2=2^{20}+1.\]
Compute $a+b$.
\end{prob}
\end{document}