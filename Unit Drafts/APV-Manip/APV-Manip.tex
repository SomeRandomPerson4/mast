\documentclass{article}
\usepackage[mast]{dennis}

\title{Algebraic Manipulation}
\author{Dennis Chen, skyscraper}
\date{APV}

\begin{document}
\maketitle

Here, we talk about some clever algebra manipulations. These basic tricks are fairly common in competition questions, and knowledge of these tricks generally reduces the time needed to solve algebra questions.

\section{Plain Algebraic Manipulation}

The most basic of algebraic manipulations involves systems of equations. The gist is that when you are given $x\circ y,y\circ z,$ and $x\circ z$ (for some operation $\circ$), you can find $x \circ y \circ z$ to determine $x,y,$ and $z.$
\begin{exam}[2018 AMC 10B/4]
A three-dimensional rectangular box with dimensions $X$, $Y$, and $Z$ has faces whose surface areas are $24$, $24$, $48$, $48$, $72$, and $72$ square units. What is $X$ + $Y$ + $Z$?
\end{exam}
\begin{sol}
Manipulation does the trick better than just bashing out possible values. Note that the surface areas can be expressed as $XY,XZ,$ and $YZ.$ Therefore, we can let $XY = 24, XZ = 48, YZ = 72.$ A good way to solve for $X,Y,$ and $Z$ individually is finding $XYZ$ and then dividing by each of $XY, XZ,$ and $YZ.$ Multiplying all equations together yields $XY \cdot XZ \cdot YZ = (XYZ)^2 = 24 \cdot 48 \cdot 72 \implies XYZ = 288,$ so it follows that $X = \frac{XYZ}{YZ} = \frac{288}{72} = 4.$ Similarly, $Y = 6$ and $Z=12,$ so our answer is $4+6+12= 22.$
\end{sol}
\begin{exam}[Classic]
Find positive integers $a,b,$ and $c$ that satisfy
\begin{align*}
(a+b)(a+c)=96(b+c) \\
(a+b)(b+c)=54(a+c) \\
(a+c)(b+c)=24(a+b)
\end{align*}
\end{exam}

\begin{sol}
The idea of this problem is similar to the previous example, in that there is a way to ``manipulate" the product. Multiplying the equations gives: 
\[(a+b)^2(a+c)^2(b+c)^2 = 96 \cdot 54 \cdot 24(a+b)(a+c)(b+c)\implies (a+b)(a+c)(b+c)= 96 \cdot 54 \cdot 24\]
We can divide by $(a+b)(a+c)$ to obtain $b+c = \frac{96 \cdot 54 \cdot 24}{96} \cdot (b+c),$ or $b+c = 36,$ and then apply the same method with $a+b$ and $a+c.$ But we're not done, we still need to resolve:
\begin{align*}
b+c = 36 \\
a+c = 72\\
a+b = 48
\end{align*}
Applying the manipulation trick again, we have: $2a +2b+2c = 156 \implies a+b+c = 78.$ Subtracting $a+b+c$ from $b+c,a+c,$ and $a+b$ leads to our desired answer: $a = 42, b = 6, c = 30.$
\end{sol}

\section{Numbers as Variables}

\begin{exam}[SMT 2021/N5]
There are exactly four distinct positive integers $n$ for which $15380 - n^2$ is a perfect square. Noting that $13^2 + 37^2 = 1538$, compute the sum of the four possible values of $n$.
\end{exam}

This problem is quite difficult compared to the rest of the handout. Even then, I believe the problem is worth presenting because it's such a good example of how numbers can obscure the intent of an algebraic problem, and how knowing that these numbers are used solely to obscure can give you the courage to push through and find the hidden identity.

\begin{sol}
The solution hinges on the fact that
\[(a^2+b^2)(c^2+d^2)=(ac+bd)^2+(ad-bc)^2.\]
Note that
\[15380=(1^2+3^2)(13^2+37^2)=(13+111)^2+(36-27)^2=124^2+9^2\]
and
\[15380=(1^2+3^2)(37^2+13^2)=(37+39)^2+(13-74)^2=76^2+61^2.\]
Since we know there are only four values of $n,$ the answer is
\[124+9+76+61=270.\]
\end{sol}

This $(a^2+b^2)(c^2+d^2)=(ac+bd)^2+(ad-bc)^2$ identity is known as the Brahmagupta-Fibonacci Identity. It is a quite popular identity used in non-AMC style contests, such as ARML, college contests, etc.

\begin{exam}[2017-2018 Mandelbrot]
Let us say that a decade is \emph{primeval} if it contains four prime numbers. For instance, the decade from $1480$ to $1490$ was primeval, since $1481, 1483, 1487$ and $1489$ are all primes. Let $p_1,p _2, p_3$ and $p_4$ be the primes (in order) in the  next primeval decade after $2020.$ Compute the value of $p_2p_3 - p_1p_4.$
\end{exam}

\begin{sol}
Finding the next primeval decade is too time consuming! We instead use two observations:
\begin{itemize}
\item Prime numbers greater than $7$ must end in $1,3,7,$ or $9.$
\item If the oldest year of a primeval decade is $x,$ then the four prime years are $x+1,x+3,x+7,$ and $x+9.$
\end{itemize}
Now the problem becomes standard algebra fare; our answer is $(x+3)(x+7)-(x+1)(x+9) = 12.$ We avoided unnecessary computation here, by treating large numbers as simple variables.
\end{sol}

\section{Symmetric and Cyclic Expressions}

\pagebreak\section{Problems}

\minpt{}

\psetquote{I can only say that the human curiosity is something completely irrational.}{Puella Magi Madoka Magica}

\begin{prob}[2008-2009 Mandelbrot]{3}
Austin currently owns some shirts, pants, and pairs of shoes; he chooses one of each to create an outfit. If he were to obtain one more shirt, his total number of outfits would increase by $48.$ Similarly, if he bought another pair of pants he would have $90$ more outfits, while an extra pair of shoes would result in $120$ more outfits. How many outfits can he currently create?
\end{prob}
  \\

\begin{prob}[OMO Fall 2014/12]{4}
Let $a$, $b$, $c$ be positive real numbers for which \[
  \frac{5}{a} = b+c, \quad
  \frac{10}{b} = c+a, \quad \text{and} \quad
  \frac{13}{c} = a+b. \] If $a+b+c = \frac mn$ for relatively prime positive integers $m$ and $n$, compute $m+n$.
\end{prob}
  \\
  
\begin{prob}[2019 hARMLess Mock ARML/4]{5}
Fran has two congruent rectangular prisms, each with volume $V$. There are three distinct ways for Fran to glue the two prisms together along congruent faces to form a larger rectangular prism. The three prisms that can be created in this way have surface areas $20,18,$ and $17$. Compute $V$.
\end{prob}
\\

\begin{prob}[2014 November HMMT]{5}
Let $a, b, c, x$ be reals with $(a + b)(b + c)(c + a) \neq 0$ that satisfy
$$\frac{a^2}{a+b}=\frac{a^2}{a+c}+20, \frac{b^2}{b+c}=\frac{b^2}{b+a} + 14 \text{, and } \frac{c^2}{c+a}=\frac{c^2}{c+b}+x$$
Compute $x$.
\end{prob}
\\

\begin{prob}[vvluo]{6}
Determine all solutions, if any exist, to the system
\begin{align*}
\frac{1}{xy}&=\frac{x}{z}+1 \\
\frac{1}{yz}&=\frac{y}{x}+1 \\
\frac{1}{zx}&=\frac{z}{y}+1.
\end{align*}
\end{prob}
\\

\begin{prob}[NICE Spring 2021/13]{6}
Suppose $x$ and $y$ are nonzero real numbers satisfying the system of equations
\begin{align*}
    3x^2 + y^2 &= 13x,\\
    x^2 + 3y^2 &= 14y.
\end{align*}
Find $x+y$.
\end{prob}
\\

\begin{req}[JMC 10 2021/18]{9}
If $x,y,$ and $z$ are positive real numbers that satisfy the equation
\[xy+yz+zx=96(y+z)-x^2 = 24(x+z) -y^2 = 54(x+y) -z^2,\]
what is the value of $xyz$?
\end{req}
\\


\begin{prob}[CARML 2019/8]{13}
There exist unique positive integers $1<a<b$ satisfying
\[a^2+b^2=2^{20}+1.\]
Compute $a+b$.
\end{prob}
\end{document}