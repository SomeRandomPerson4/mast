\documentclass{article}
\usepackage[mast]{lucky}
\usepackage{graphicx}

\title{Barycentric Coordinates}
\author{Kelin Zhu}
\date{GRV}

\begin{document}
\maketitle
Barycentric coordinates is a powerful technique of bashing that can be used to solve many Olympiad and late-AIME difficulty problems. Its strength comes from that it sets the sides of a triangle as axes, rather than two perpendicular lines. In a sense, it can be thought of as a sequel to mass points due to the emphasis on ratios present in both techniques.
\end{document}