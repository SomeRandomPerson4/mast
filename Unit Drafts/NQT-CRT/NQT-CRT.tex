\documentclass[mast]{lucky}

\title{Chinese Remainder Theorem}
\author{Dennis Chen, Kelin Zhu}
\date{NQT}

\begin{document}
\maketitle

The Chinese Remainder Theorem is a centerpiece of AIME and AMC Number Theory; many problems are unsolvable without invoking it and many more can be greatly simplified with it.

\begin{theo}[Chinese Remainder Theorem (abbreviated as CRT)]
For pairwise relatively prime positive integers $n_1,n_2,\ldots,n_k,$ $a$ mod $n_1n_2\cdots n_k$ uniquely determines $a$ mod $n_i$ for $1\leq i\leq k,$ and vice versa.
\end{theo}

\section{Fundamental Theorem of Arithmetic}
This will be explored in greater depth in the NQV-Prime unit, but it goes along really well with CRT and complements our knowledge from NPU-Mods, so I will include it.

\begin{theo}[Fundamental Theorem of Arithmetic]
Every number greater than $1$ is either a prime or can be uniquely, up to order, expressed as a product of primes.

(You have probably heard of it or even discovered it on your own; the formal theorem reassures that you can assume such thing.)
\end{theo}

If you are curious about the proof, you may check out \url{https://gowers.wordpress.com/2011/11/18/proving-the-fundamental-theorem-of-arithmetic/}.

\begin{fact}[Canonical Representation]
The \textbf{canonical representation} of a positive integer $n$ is of the form $p_1^{e_1}p_2^{e_2}\ldots p_m^{e_m}$ for $p_1<p_2<\ldots <p_m, e_1,e_2\ldots e_m>1$ 
\end{fact}

In conjunction with CRT, this means that we can evaluate any integer mod $n$ by evaluating it mod $p_1^{e_1},p_2^{e_2}\ldots$
\section{Two Main Applications}
CRT tells us that we can:
\begin{enumerate}
\item given an independent system of linear congruences, you can ``stitch them together'' to one linear congruence that encompasses all of the conditions. This is often used when you obtain two mod conditions that are disjoint into one big condition that is easier to work with. In particular, finding values satisfying two linear congruences is hard, but finding values satistfying one modular congruence means one simply has to work with an arithmetic sequence.
\item given a linear congruence, you can ``take it apart'' into an independent system of linear congruence that encompasses the original congruence, and solve them independently. This is often used to split up a residue mod a composite number into residues mod prime powers, that are usually much more tolerable, and then using the previous to ``put them back together" to find a exact value. For example, one might want to find the last three digits of a large number $N$. What one would do is find $N \pmod{8}$ and $N \pmod{125}$, then combine them to find $N\pmod{1000}$.
\end{enumerate}


\end{document}