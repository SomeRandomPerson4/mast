\documentclass[blue,onecol]{shooting}

\setcounter{tocdepth}{0}

\title{The MAST Compendium}
\author{Dennis Chen}
\date{\today}

\begin{document}
\maketitle

\section*{Dedication}
\begin{center}
{\alegreyafont To the students who made this possible,

\alegreyafont thank you.}
\end{center}

\pagebreak

\section*{Foreword}

\pagebreak

\section*{Preface}
This book is a collection of the materials that I've used in my program, which I have fortuitously named MAST. The first name I came up with was Mid-AIME Training Program (MATP). I did not initially intend on coming up with a funny acronym for my program, but after seeing the letters ``MAT'' appear in succession, I knew that I could potentially score some extra points by making my acronym happen to be a word. Thus ``Mid-AIME Self-Training'' (MAST) was born. I wanted to make the base of the program wider and give it more potential to spread without rebranding, and the current name felt lacking somehow, so ``Mid-AIME'' became ``Math Advancement (by).'' And thus the final name ``Math Advancement by Self-Training'' was born.

MAST is not the progenitor of these materials. In the fall of 2017, during my middle school years, I ran a program for a small number of my classmates with the abbreviation MPP, though I presently cannot recall what the letters stand for. Looking back, it is a miracle that the program happened in the first place -- indeed, it is odd that my classmates got their parents to agree to let them stay in a run-down garage for two hours a week -- but I was not satisfied with the shape it had taken. As high school started in the fall of 2019, few of my classmates returned to the program during the third year as they expected high school to keep them busy, and the gap was being filled up by younger students. During the March of 2020, a variety of factors pushed and emboldened me to start MAST: I had almost been guaranteed to qualify for the upcoming USAJMO (it had not been cancelled at the time, and though in retrospect I was embarrassingly close to missing the qualification, I was unaware that it would be so hard to qualify this year), and I believed that the dwindling participation in MPP which so dispirited me would be fixed if only I had more dedicated students. These materials and this program had already been fundamentally transformed once, and not for the better. The ``spirit'' of the old program was no longer worth preserving, and with that, I made the decision to start the program and advertise it on the Art of Problem Solving website.

Initially I was extremely apprehensive of how the program was going to turn out. I did not envision, for better of for worse, that the program would garner as much interest as it had. The rolling application model I had worked to the program's favor, keeping the growth constant -- quite impressive given the way most programs fizzled out after a few weeks. I quickly realized through my growing inability to reply to emails that keeping track of everything manually was impossible, so I put the program on hiatus and had Amol Rama design a website which could do this for me. This turned out to be a major improvement for keeping track of things, particularly in comparison to my habitual failure to respond to emails without having to be reminded that they exist. I thank all my students who took the pains to apply to my program and stuck with it despite the inconveniences that I may have put them through.

In parallel with the growth of the student base was the formation of the staff team. A few students noticed early on, when the direction of MAST was still very much up in the air, that they would be better suited on the staff team and were thus invited to join it. Still others sent me emails or messages asking if they could help somehow with the program. At the time growing the staff team so fast seemed like quite a gamble, but in retrospect the atmosphere I had built around the program made it so that all who asked to help out were qualified to do so (and consequently, nearly every staff request was accepted). I am very glad that I ended up making the right call here, and I hope the staff are also glad that they decided to participate in this program. My gratitude goes it to the staff team, for without them I could not have taken on as many students as I can now, and without them I would not have found my footing or have a clear vision for my program.

As the program continues to grow and my busywork does as well, my efforts to keep up with the work I need to do have been in vain. I think the most appropriate remark to end the preface with must be ``Don't worry! The microphone won't catch anything,'' which I said in response to the general hubbub around my house, unaware that this had ironically been caught on the microphone. Though many of my overconfident assertations have not held true and embarrassing mistakes have been made (and will continue to be made), I thank my students for staying through the program and recommending it to their friends despite this, the staff team for supporting me through my many incompetencies, and you, the reader, for making this compendium worth compiling.

\pagebreak

\section*{How to Use}

\pagebreak

\toc

\part{Vital Topics}

\part{Useful Topics}

\part{Tangential Topics}

\part{Solutions}

\setcounter{chapter}{0}

\end{document}